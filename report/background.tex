\chapter{Background}\label{C:background}
\comment{
The background should cover any important terminology and/or
concepts used in the remainder of the report, and should demonstrate an
understanding of previous works which are relevant.

Remember: A good related work section does not just provide a list of previous works,
accompanied with short summaries.
Wherever possible it must extract real insight from these works, painting a
picture of how they relate to each other and the project.
}


This project draws from research in the field of formal software verification.
The project is based on the Whiley language which supports formal software
verification.
The concept of loop invariants are a necessity is a key to this project.

Formal methods is an application of mathematics to prove software and hardware
systems against a given specification. \todo{cite this}
Formal software verification is the specific application to software with
the intention to prove its correctness.

\todo{fix this stuff}

\section{Formal Software Verification Languages}
\comment{
 * Formal Software Verification \\
 * goals of class of languages (could be covered in Whiley)
}

Formal software verification is a field of software engineering with the
purpose to create software that is able to be mathematically verified to be
correct \cite{survey-formal-soft}.
Prof. Sir Tony Hoare issued a grand challenge to create a \textit{verifying compiler}
that guarantees the correctness of the program before running it \cite{Hoare-grand}.
There has been tool sets that expand languages to achieve this, such as
the SPARK tool set for the Ada \cite{spark-ada}\cite{spark-high-integ}.
Though this generally restricts the functionality of the language, for example
using Ada with SPARK disallows the use of pointers which makes general purpose
programming more challenging relative to C \cite{spark-ada}.
There are some languages that are designed and built from the ground up to
be a formal software language and have a \textit{verifying compiler}.
These languages include such as Whiley~\cite{whiley-origin} and Dafny~\cite{dafny-lang}.

\section{Whiley}
\comment{
Whiley \\
 * goals of the lang \\
 * samples of code \\
 * whiley design paper \\
}

The Whiley language is a general purpose language that has first class
support for formal software verification through its syntax and compiler
tool set~\cite{whiley-origin}.
The design goal for Whiley is to make a platform for formal verification~\cite{whiley-origin}.
The tool set includes the compiler and an accompanying theorem prover.
The compiler translates the source code to an intermediate language which
can then be used with the theorem prover or transforms it into a target
language such as JVM byte code or JavaScript~\cite{whiley-design}
\cite{wyil}.

The structure of a Whiley program is similar to most imperative languages
such as Java or Python.
In Whiley there are functions and methods with the point of difference being
that functions are pure and must return a result,
where methods can be impure and can optionally return a result
\cite{whiley-spec}\cite{whiley-design}.
Both functions and methods can have pre- and post-conditions using
\code{requires} and \code{ensures} respectfully as shown for the \code{min}
function in figure~\ref{lst:whiley-exp}.
Types can be declared as aliases of structures and can include type invariants,
such as the \code{nat} type in figure~\ref{lst:whiley-exp}.

\begin{figure}[ht]
\begin{lstlisting}
function min(int a, int b) -> (int m)
requires true
ensures m == a || m == b
ensures m <= a && m <= b:
    if a < b:
        return a
    else:
        return b

type nat is (int x) where x >= 0

method main():
    nat i = 0
    while i < 10:
        assert min(i, 0) == 0
        assert min(100, i) == i
        i = i + 1
\end{lstlisting}
\caption{Example of a programs structure in Whiley}
\label{lst:whiley-exp}
\end{figure}

\subsection{Whiley Features}
\comment{
Cover the key features used in the loop invariants generators \\
These are features differ languages so will be covered \\
* while loops where clause - covered in While and loop invs \\
* array copy \\
* array generator
}

This section explains two features of Whiley leveraged to generate loop invariants.
These are: copy semantics; and array generators~\cite{whiley-design}.
The Whiley language specification book is available for the full
specification of the language~\cite{whiley-spec}.

\paragraph{Copy semantics} is a design decision of Whiley to
make clones of data instead of referencing the data on assignment \cite{whiley-arrays}.
Figure~\ref{lst:whiley-array-copy} shows an example of the variable
\code{b} making a copy of \code{a} and that mutating \code{a} later
will have no affect on the value of \code{b}.
This is often utilised to make a copy of an array before mutating it
and use it as a reference point in an invariant~\cite{whiley-arrays}.
This is an important usability feature of Whiley as copying values
instead of taking references allows for simpler verification due to lack
of pointers and disallows accidental mutations by unknown references.
\footnote{
Though it is not related to this project,
Whiley does support explicit references to data.
They also include lifetimes to how long the references can live
to remove null reference errors, similar to Rust's lifetimes~\cite{rust-lang}.
}

\begin{figure}[ht]
\begin{lstlisting}
a = [1, 2, 3]
b = a
assert a == b
assert |a| == |b|

a[0] = 2
assert a != b
assert |a| == |b|
\end{lstlisting}
\caption{Example an array copy in Whiley}
\label{lst:whiley-array-copy}
\end{figure}

\paragraph{Array generator} is a feature in Whiley to generate an array
of a specific element duplicated over a specific size~\cite{whiley-spec}
\cite{whiley-arrays}.
Figure~\ref{lst:whiley-array-gen} shows an example of using an array generator
in Whiley and its associated results.
This is a short hand for creating arrays with a default value which is useful
to fill an array of arbitrary length with a default value.
A result of this feature is with static analysis is possible to infer the
length of the array at runtime by equating the array length to the value
from the generator, for example line 7 of figure~\ref{lst:whiley-array-gen}.

\begin{figure}[ht]
\begin{lstlisting}
a = [0; 3]

assert |a| == 3
assert |a| == [0, 0, 0]

b = [0; |a|]
assert |a| == |b|
\end{lstlisting}
    \caption{Example an array generator in Whiley}
    \label{lst:whiley-array-gen}
\end{figure}

\section{Loop Invariants}
\comment{
Loop invariants - overview of loop invariants
Theorem prover knows nothing about loop invariants - must give all \\
* must hold at entry \\
* must hold each iteration \\
* must hold on exit \\
* implies post-condition \\
* implied by pre-condition \\
}

Sir Prof.\ Tony Hoare described a general loop with an invariant in
the \textit{Rule of iteration} to be
portion of a program ($S$) that is repeated while a condition ($C$)
holds and an assertion ($I$) that holds on every
iteration~\cite{hoare-logic}~\cite{loop-inv-survey}.
Figure~\ref{lst:general-loop} is an example of this general loop with Whiley
syntax.
The loop invariant is then able to be used to prove properties of the
loop~\cite{loop-inv-survey}.
The loop invariant is a part of the three stages of the loop: entry;
iteration; and exit.

\begin{figure}[ht]
\begin{lstlisting}
while C:
    assert I
    S
\end{lstlisting}
    \caption{A general loop with an invariant $I$}
    \label{lst:general-loop}
\end{figure}

At the entry of the loop the loop invariant must hold.
This results in that the pre-condition of the loop ($P$) must imply
the loop invariant ($I$) as shown below.
This is illustrated in figure~\ref{lst:whiley-inv} where pre-condition
before entering the loop is clearly that \code{i} equals 0 and that holds
true for the invariants, in particular that \code{i >= 0}.

$$\text{Entry} \quad P \implies I$$

With each iteration of the loop the invariant is asserted some where
inside the loop body~\cite{hoare-logic}.
This is generally done at the start of the loop for while
statements~\cite{whiley-origin}.
With the invariant holding at the start of the loop the end of 
the loop must then restore the invariant.
Thus what is known in the loop is that the loop condition ($C$)
and the loop invariant ($I$) holds as shown below.

% $$\text{Iteration} \quad C \wedge I $$

On exit of the loop the loop invariant must still hold.
Since the loop has ceased the loop condition no longer
holds. Thus an exit of the loop can be described
as the conjunction of loop invariant and the logical
not of the loop condition.
This results in an implication of the post-condition of the
loop.

$$\text{Exit} \quad I \wedge \neg{C} \implies Q$$

Figure~\ref{lst:whiley-inv} illustrates this as the
post condition of the loop can be reasoned to be the following
from the loop condition and the invariant:
$$\neg{ ( i < |items| ) } \wedge i \leq |items| \wedge i \geq 0$$

This can be simplified to down:

$$ i = |items| \wedge i \geq 0$$

This implies the post-condition of the function that $r = |items|$ with
the variable \code{i} renamed to \code{r}.
Each of the stages of a loop are similar to building an inductive proof \cite{invarints-classifiction}.

\subsection{Loop Invariants in Whiley}

In the Whiley language the concept of loop invariants are a 1\textsuperscript{st} class
construct.
Whiley gives syntax for providing \code{where} clauses to a \code{while}
statement.
Each of the \code{where} is a boolean expression which is the loop
invariant predicate.
Having multiple \code{where} clauses is equivalent to all the clauses are
conjunctively joined.
Figure~\ref{lst:whiley-inv} gives a full example of using a loop invariant
in Whiley. The \code{where} clauses on lines 6 and 7 are Whiley's syntax for
providing the loop invariant in the form of a predicate.

\begin{figure}[ht]
\begin{lstlisting}
function count(int[] items) -> (int r)
ensures r == |items|:
    int i = 0
    //
    while i < |items|
        where i >= 0
        where i <= |items|:
        i = i + 1
    //
    return i
\end{lstlisting}
\caption{Example of loop invariant in Whiley}
\label{lst:whiley-inv}
\end{figure}

\section{Classification of Loop Invariants}
\comment{
classification of loop invariants
 * proves property \\
 * hints to theorem prover \\
 * examples \\
}

There are two classifications of loop invariants.
These are bounding loop invariants and essential invariant~\cite{invarints-classifiction}~\cite{loop-inv-survey}.
With the classification of a loop invariant it helps to make
it clearer what the purpose of the invariant is trying to achieve.

A \textit{bounding loop invariant} is an invariant that is designed
to limit the values of a variable to assert correctness of the loop
body~\cite{loop-inv-survey}.
Figure~\ref{lst:whiley-inv-class} has an example of a bounding invariant on
line 6. This loop invariant does have any impact on finding an index
of the given item but is to assert that the variable \code{i} will index
the array correctly by being inside the valid range of 0 to the length of
the array (exclusively).
The loop condition provides the upper limit of the range of \code{i} but
the lower bound is needed to be provided as an invariant.
The most common bounding invariant observed is enforcing the starting bound
of a range.

An \textit{essential invariant} is an invariant that is an inductive step towards the
postcondition\cite{invarints-classifiction}~\cite{loop-inv-survey}.
Figure~\ref{lst:whiley-inv-class} has an example of an essential invariant on
line 7.
In this case loop invariant is a weakening of the postcondition to only
check that the item is not contained the array previously processed.
This invariant provides the essential inductive reasoning for when the item is
not contained inside the array.

\begin{figure}[ht]
\begin{lstlisting}
function indexOf(int[] items, int item) -> (int r)
ensures r >= 0 && r < |items| ==> items[r] == item
ensures r == |items| ==> all { i in 0..|items| | items[i] != item }:
    int i = 0
    while i < |items|
        where i >= 0
        where all { j in 0..i | items[j] != item }:
        if items[i] == item:
            break
        i = i + 1
    return i
\end{lstlisting}
\caption{Example of loop with bounding and essential invariants}
\label{lst:whiley-inv-class}
\end{figure}

\section{Categories of Generated Loop Invariants}
\comment{
Methods to generate loop invariants \\
Generally generating loop invariants are impossible! \\
}

It has been proven that it is not possible to generate loop invariants
generally \todo{cite this}
so various methods have been developed to generate some invariants.
These methods can be sorted into one of two categories, static and dynamic.
This section will give an overview of static and dynamic methods of loop
invariants.

\subsection{Static Generation of Loop Invariants}

The static generation of loop invariants use the
semantics and analysis of the source code to generate the loop invariants
\cite{java-static-symb}.
Some of the techniques used to statically generate loop invariants are:
weakening postconditions~\cite{infer-postconditions};  symbolic symbolic
execution~\cite{java-static-symb}; and structural
induction~\cite{struct-induction}.
These techniques are strict and conservative in generating loop invariants
to ensure what was generated is a valid invariant without being able to
verify it.

% The key difference between the categories is
% that invariants found statical hold true
% but is restricted to what can be generated.
% \cite{benderfinding}\cite{Leino2005LoopIO}.
% The solution presented in this report uses
% the principles of static methods to generate loop invariants.

\todo{EXAMPLES}

\subsection{Dynamic Generation of Loop Invariants}

Dynamic invariants are similar to dynamic program analysis
tools, they execute the program to find properties.
In the case of dynamically generating loop invariants a
common techniques to generate a set of candidate invariants
and modify the program to include them for verification.
The techniques used to generate the candidate loop invariants
can be derived from less-strict static analysis techniques and
from observing the execution of the program.
This allows a greater range of invariants to be inferred at
the cost of the time taken to test if the candidate invariants are
valid~\cite{infer-dynamic}\cite{infer-postconditions}.


%\subsection{Loop Design Pattern}
%
%A loop design pattern is similar to an architecture pattern.
%In that they can be classified into groups by there intent of
%the pattern and how they are implemented.
%For example, a strategy pattern is implemented to enable changes of
%behaviour dependent on some input by utilising interfaces.
%An example of a loop pattern would be a search through
%an array for an element or processing all elements of an array
%\cite{loop-patterns}.
%A loop pattern can be described in the same terms of an architectural pattern
%by providing the following outline from figure~\ref{l:design-pattern}
%from \textit{A Pattern Language} \cite{pattern-lang}.
%
%\begin{figure}[ht]
%\begin{itemize}
%    \item{\textbf{Examples} of pattern in application}
%    \item{\textbf{Context} in which the pattern is used}
%    \item{\textbf{Problem} that the pattern addresses or solves}
%    \item{\textbf{Forces} requirements or trade offs that constrain possible
%        solutions}
%    \item{\textbf{Solution} the pattern}
%    \item{\textbf{Resulting Context} having used the pattern, what is needed
%        now}
%\end{itemize}
%    \caption{Elements of a design pattern\cite{pattern-lang}}
%    \label{l:design-pattern}
%\end{figure}


