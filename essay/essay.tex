\documentclass{CRPITStyle}
\usepackage[authoryear]{natbib}
\renewcommand{\cite}{\citep}
\pagestyle{empty}
\thispagestyle{empty}
\hyphenation{roddick}

\begin{document}

\title{When to use NoSQL}
\author{David Barnett}
\affiliation{School of Engineering and Computer Science \\
Victoria University of Wellington, \\
PO Box 600, Wellington, 6140 \\
Email:~{\tt barentdavi@myvuw.ac.nz}}

\maketitle

\begin{abstract}
    TODO Abstract
\end{abstract}

\vspace{.1in}

\noindent{\em Keywords:} NoSQL, Use cases, Document Store, Column Base, Graph database

\section{Introduction}
% Cover the aim of the essay

\paragraph{}
With growing use of NoSQL databases in production environments it has become more
important to understand when and why to use them over a traditional
relational database.
NoSQL can be broken down to many different types which differ on how
data is represented~\cite{type_nosql}.
For example the traditional relational databases are based upon relational
model and relational algebra~\cite{relational_db}.

\paragraph{}
This report will cover some common types of NoSQL databases.
These are: Document store; Column base; and, Graph databases.
For each of these types this report will give an overview on how each type
operates and a description of its use case with a comparison with a
traditional SQL database.

\cite{whats_new}
\cite{sql_nosql_gap}
\cite{scalable_sql}

\section{What is NoSQL}
% brief overview

\paragraph{}
NoSQL is a class of databases that are not based on
a relational data model~\cite{nosql_db}.
Instead of using relations the data is modeled using a large
variety of models are used, such as graphs, documents or objects.
These allow for application specific optimisations that
a relational database could not provide.
The need for these optimisations are motivated by the growing
volume of data that is being handled by applications~\cite{nosql_db}.
To accommodate these optimisations a shift in the design of the
databases occurred.

\paragraph{}
The design goals for NoSQL database compared to
a relational database are similar but distinct.
They both share the goal of providing a service that
manages the storage and retrieval of data.
However, they differ in what properties
of these actions they guarantee.
For relational databases they are designed to hold
the properties of a relational database~\cite{relational_db,base}.
Figure~\ref{l:acid} provides a brief outline of the ACID properties.

\begin{figure}
\begin{itemize}
    \item \textbf{A}tomic - every transaction will either
        succeed and be applied or fail and commit no modifications.
    \item \textbf{C}onsistent - the state of all relations are
        always in a consistent state.
    \item \textbf{I}solation - transactions are performed as if sequentially.
    \item \textbf{D}urability - after a successful transaction the results
        are persistent over crashes.
\end{itemize}
    \caption{Outline of the ACID properties \cite{relational_db,base}}
    \label{l:acid}
\end{figure}


% compare with relational model

Where relational databases are based on a structured
data model that is strictly enforced~\cite{relational_db}

% give definition of a database? (managed store of data)
% relational is based on set theory and relational algebra

\subsection{BASE}
%% BASE
% What it is

\cite{base}

\begin{itemize}
    \item \textbf{B}asically
    \item \textbf{A}vailable
    \item \textbf{S}oft state
    \item \textbf{E}ventual consistency
\end{itemize}

% How does it compare to ACID

% What does this mean

\subsection{CAP}
%% CAP theory
% what is it

\begin{itemize}
    \item \textbf{C}onsistency
    \item \textbf{A}vailability
    \item \textbf{P}artition tolerance
\end{itemize}

% how it relates to NoSQL

% trade-off between consistency, availability and partition tolerant

% === Document Store ===

\section{Document Store}
% How does it work
%  - key to a structured document
%  - document can be inspected by DBMS
% Relate it back to BASE


\subsection{Use case}
% When are people using this
%  - JSON store
%  - XML store
% Example of use case
%  - Searching

\subsection{When to use it}
% Why are people choosing this
% Compare to relational counterpart

% === Column base ===

\section{Column base}
% How does it work
% Relate it back to BASE

\subsection{Use case}
% When are people using this
% Example of use case

\subsection{When to use it}
% Why are people choosing this
% Compare to relational counterpart

% === graph ===

\section{Graph}
% How does it work
% Relate it back to BASE

\subsection{Use case}
% When are people using this
% Example of use case

\subsection{When to use it}
% Why are people choosing this
% Compare to relational counterpart

\section{Conclusion}


\bibliographystyle{agsm}
\bibliography{essay}

\end{document}
