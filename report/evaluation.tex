\chapter{Evaluation}\label{C:eval}
\comment{
The purpose of the evaluation section is to demonstrate whether you did \\
or did not satisfy the project goals or specifications. \\
If you can tie the performance of your design to some real specification then \\
your evaluation is much stronger. “My code runs in 29 ms” is much weaker than \\
“my code runs within the 30 ms window allowable for real-time performance of the. . . ”. \\


In many cases the evaluation of a project requires significant extra work to design and build test harnesses. \\
These should be explained so that the validity and scope of the evaluation can be understood. \\
Make liberal use of graphs and other figures. \\
They are much more effective at communicating many results than are words. \\
}

\section{Goal}

\section{Evaluation tool}\label{S:eval-tool}
\comment{explanation of the evaluation tool and what each piece does}

\subsection{Breakdown}
\comment{
What this does and how it benefits the evaluation \\
* normalises the data set \\
* removes the difference between the one-liner master and the laid out \\
thinker \\
* breaks up conjunctive expressions in the loop invariants \\
}

\subsection{Minimizing}
\comment{
What this does and how it benefits the evaluation \\
 * finds the smallest subset of loop invariants for the code to compile \\
 * tests with and without  \\
 * tests combinations of  \\
}

\subsection{Reporting}
\comment{
What this does and how it benefits the evaluation \\
  include the eval script here to piece it all together \\
 * collects data about each run \\
 * explain how each data set is tested \\
  + files that do not contain while loops are removed via grep'ing them \\
  + are checked if they will compile and verify before any changes (removes original errors from data set) \\
}

\section{Data Sets}
\comment{
Explain that there is not much data to pick from \\
Discuss the data sets used and what problems they have \\
}

\subsection{Whiley Compiler Tests}

\subsection{Assignments and Labs from SWEN224}

\section{Results}

\subsection{Whiley Compiler Tests}


\begin{center}
\begin{tabular}[c]{r}
    \csvautotabular[]{appendix/summary.csv}{}
\end{tabular}
\end{center}

\subsection{Assignments and Labs from SWEN224}

\section{Discussion}
