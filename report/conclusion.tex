\chapter{Conclusions}\label{C:con}

\comment{
 Goal of chapter: 

  * reflect on the project, \\
  * how it was accomplished, \\
  * what it accomplished, \\
  * what could be done better next time, \\
  * what was learnt. \\
}

\section{Accomplishments}
\comment{talk about how the goal of the project was accomplished, loop invariants are generated for Whiley}

This project has accomplished its primary goal.
The burden of providing repetitive simple loop invariants has been reduced by generating simple loop invariants.
This was achieved by studying simple loop invariants to find a general loop pattern where they arise.
From this pattern a loop invariant generator to discover when these patterns apply
then adding the invariant to the loop.
The success of this approach is evident from the 70\% overall reduction of the number of loop invariants 
that the programmer is required to provide in the Whiley compiler test suite.
From this it is evident that the burden on the programmer to write
these repetitive simple loop invariants have been reduced.

The secondary goal of allowing more programmers and student utilise formally verifiable
languages was not evaluated.
Thus it is undetermined that the introduction of generated loop invariants would achieved this goal.
This is because of a change in direction of the project to focus on generating simple loop
invariants over the study of the implications of generated invariants on programmers (see Section~\ref{s:requirements}).

\section{Related Work}
\comment{Covers the literature review and compare current solution to similar projects}

The problem of loop invariants have been a target for generation for decades.
Over this time other formally variable languages

\todo{THIS SECTION}

\fixme{Should this be in the conclusion section rather than here?}

\subsection{Dafny}
\comment{
	explain how dafny is an alternative to Whiley in the Spec Lang space
	has some features that are missing in Whiley that would be interesting to use
	
	* declaring variables as increasing or decreasing \\
	* defining variables / fields as ghosts, to only be used in specifications
	dafny - other formal language
}

Dafny is language a similar language to Whiley as they both are design with the
goal to be  being formally verifiable.
Dafny is a research language for a verifying compiler built by Microsoft
Research~\cite{dafny-lang}.
Dafny shares a majority of features with Whiley and makes advances in others.

One feature Dafny is to provide a termination metric.
This tells Dafny that the given expression will decrease with each iteration. 
In turn this proves that with each iteration a step towards its termination is achieved \cite{dafny-started}\cite{dafny-lang}.
By default this is inferred from analysis of the loop.
This feature is similar in nature to a key mechanism behind the starting bound and upper bound
generators, being able to determine if a variable increases or decreases
(see Section~\ref{s:sequence-dir}). 

Another feature of Dafny is variables can be annotated to be ghost variables
and the previous value of a variable can be referenced.
In Dafny a ghost variable is one that can only be used for verifications.
This has no impact on the runtime of the program \cite{dafny-started}\cite{dafny-lang}.
The previous value of a variable can use accessed by using a built-in
function, \code{old}, that returns the previous value in specifications.
These features would simplify the equal length and iterative assignment
generators since they could generate zero-cost clones of arrays to use
with their invariants. 


\subsection{Methods to Generate Loop Invariants}
\comment{explain how these methods create loop invariants}

There has been research into automatic inference of loop invariants over
the last three decades.
Over this time a large range of methods has been devised to infer certain
classes of invariants \cite{infer-dynamic}\cite{infer-postconditions}\cite{struct-induction}.
This section will include two tool that utilise these methods,
\text{gin-pink} \cite{infer-postconditions}, and \text{gin-dyn} \cite{infer-dynamic}.

These include logically weakening the post-conditions by back-propagating them
to a loop invariant \cite{infer-postconditions}\cite{infer-dynamic} and
matching a loop to a common pattern\cite{pattern-loop-inv}, similar to what is achieved in this
report.

\todo{this should really be completed at some point}

\subsubsection{Gin-Pink}

In \textit{Inferring Loop Invariants using Postconditions} 
the \textit{gin-pink}'s proposed \cite{infer-postconditions}.
This is a dynamic loop invariant generator similar to the generator
present in this report.
\textit{gin-pink} generates loop invariants by considering the program's code
and the supplied post-conditions.
To achieve this a number of candidate invariants are generated from the 
post-conditions of procedures.
The candidate invariants are generating by applying heuristics to the
post-conditions to form weaker conditions to be used.
Each candidate invariant is tested to see if it holds.
This contrasts with this report as they are both aiming to generate loop
invariants but focus on different foundations.
Where \textit{gin-pink} is aiming for more general invariants 
this report is focused on known repetitive invariants.

\subsubsection{Gin-Dyn}

In \textit{Automating Full Functional Verification of Programs with Loops} the
\textit{gin-dyn} tool is proposed \cite{infer-dynamic}.
Like \textit{gin-pink} this tool generates a range of candidate invariants,
however \textit{gin-dyn} takes it a step further by also using generated tests
as a biases to create loop invariants.


% \cite{infer-dynamic}

% \subsubsection{Variable Aging}

% \cite{infer-postconditions}
% \cite{infer-dynamic}

% \subsection{Pattern-based loop invariant generation}

% \cite{pattern-loop-inv}

% \subsubsection{Coupling}

% \cite{infer-postconditions}
% \cite{infer-dynamic}

% \subsubsection{Term dropping}

% \cite{infer-postconditions}
% \cite{infer-dynamic}

% \cite{struct-induction}

\section{Future work}

This section outlines potential future work from the result of this project.
Each section gives a brief outline of an issue raised or faced during this
project and a possible project that could provide a solution to it.

\subsection{Detection of Duplicate Loop Invariant}
\comment{this could be in the form of lexical, AST, logical equivalence}

With the introduction of generated loop invariants the duplication of loop
invariants become an issue.
A future project could be to detect and inform the programmer of these
duplications.
This could be achieved by equating loop invariants with a range of methods.
The most simple method being lexical equivalence, comparing if code are same.
An increased complexity of attempting to detect equivalent syntax with the same
semantics, e.g. \code{i >= 0} is the same as \code{0 <= i}.
With the most complex being logical equivalence, e.g. \code{i > 0 || i == 0} is the same
as \code{i >= 0}.


\subsection{Expanding Loop Invariant Generators}
\comment{
	There is more loop patterns that could be used to generate loop invariants \\
	Not all possible cases are covered by the current implementations \\
	either find these new invariants by hand, or ML \\
}

Only some of the loop patterns in Whiley were identified and exploited for loop
invariants.
A future work could expand the number of loop invariant generators or reduce
the limitations in the current generators.
The current generators currently have some limitations could be lifted with
additional work, for example static evaluation of branches to reduce the
complexity of the code being generated on.
Since the identification of loop patterns is an intensive task an automated
approach could be identified, such as applying machine learning or other static
analysis techniques.

\subsection{Visitor Pattern for the Whiley Compiler}
\comment{
	Current implementations of Generators have cloned structures \\
	Code duplication could be reduced by using Visitor Pattern \\
	* decreases maintenance cost \\
	* makes adding new syntax / AST nodes easier \\
	* technical debt having so many duplicate AST descents \\
}

A technical issue of extending the Whiley compiler is the architecture forces
the programmer to support all statements and expressions.
This causes a large amount of duplication of code for each component of the
compiler to traverse the abstract syntax tree of any of Whiley or the internal
languages of WyIL and WyAL.
This accumulates technical debt for each component as they all need to be
updated to handle new syntax.
Currently the only method to detect if some syntax is not supported is via 
failing a runtime test that the syntax is handled.
A future work of implementing the visitor pattern for Whiley and the internal
languages.
This would improve the code quality by providing strict interfaces and
classes that handle the traversal of abstract syntax trees.
This removes the issue of each component having to re-implementing traversals
and the interfaces allow the user to implement methods to handle each type of
syntax either strictly for each syntax element (via Java interfaces) or
non-strictly (via extending Java classes).

\subsection{Ghost Variables in Whiley}
\comment{
	See Ada's `'Old`  syntax \\
	Declaration of `ghost` prefix to variable declaration so they are only usable \\
	in program verification \\
}

During the project the issue of lack of ghost variables in Whiley blocked some
invariants and limited others.
They were limited by the design constraint of not wanting to generate local
variables due to the runtime cost (see section~\ref{s:design-create-var} for
more detail ).
This cost can be avoided if Whiley had ghost variables, or syntax to refer to
the value of a variable prior to the loop without copying the value.
The future work would be to add syntax to the Whiley language to support this
feature and teach the specification pipeline how to handle the ghosted
variables.
The ghost variables do not only have to be limited to loops and could also be 
extended to include the values of variables before they enter functions or
cross other boundaries.
This has been achieved in other formal specification languages such as Ada and
Dafny.

\subsection{Generated Loop Invariants in WhileyWeb}
\comment{
	Update WhileyWeb to include generated loop invariants \\
	* show in the same manner as counter-examples \\
	* is the tool most likely used in future course work of the course \\
	* a pleasant UX in a web IDE \\
}

WhileyWeb is Whiley's web developer environment that is mostly used when
developing Whiley.
A future project could be to integrate the generated loop invariants with
the web interface such as the new counter-example feature has been added to
Whiley and WhileyWeb.
Generated loop invariants could be shown as warnings with code snippets of what
was generated and why they were generated.
The challenge of this project is to control the loop invariant
generators and display their results in a natural and user friendly way.
This would also provide an opportunity to do user testing on how generated
loop invariants affect student learning of loop invariants that could not be
completed with this project.

\subsection{Solving verification problems in Whiley}

An issue that arose during this project is a lack of substantially sized code base
of Whiley code.
To mitigate this a future project to implement a large range of 
verification problems in Whiley would be beneficial.
The range of problems could be from simple singular functions such an element
contained in a list and scaling up to implementations of data structures and related
functions.
This project would also provide an opportunity to identify problem areas
in Whiley's programming experience that then could be investigated or improved.