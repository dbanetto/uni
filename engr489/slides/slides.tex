\documentclass[ignorenonframetext,]{beamer}
\setbeamertemplate{caption}[numbered]
\setbeamertemplate{caption label separator}{: }
\setbeamercolor{caption name}{fg=normal text.fg}
\beamertemplatenavigationsymbolsempty
\usepackage{lmodern}
\usepackage{amssymb,amsmath}
\usepackage{ifxetex,ifluatex}
\usepackage{fixltx2e} % provides \textsubscript
\ifnum 0\ifxetex 1\fi\ifluatex 1\fi=0 % if pdftex
  \usepackage[T1]{fontenc}
  \usepackage[utf8]{inputenc}
\else % if luatex or xelatex
  \ifxetex
    \usepackage{mathspec}
  \else
    \usepackage{fontspec}
  \fi
  \defaultfontfeatures{Ligatures=TeX,Scale=MatchLowercase}
\fi
\usetheme[]{Boadilla}
% use upquote if available, for straight quotes in verbatim environments
\IfFileExists{upquote.sty}{\usepackage{upquote}}{}
% use microtype if available
\IfFileExists{microtype.sty}{%
\usepackage{microtype}
\UseMicrotypeSet[protrusion]{basicmath} % disable protrusion for tt fonts
}{}
\newif\ifbibliography
\hypersetup{
            pdftitle={Generating Simple Loop Invariants in Whiley},
            pdfauthor={David Barnett},
            pdfborder={0 0 0},
            breaklinks=true}
\urlstyle{same}  % don't use monospace font for urls
\usepackage{longtable,booktabs}
\usepackage{caption}
% These lines are needed to make table captions work with longtable:
\makeatletter
\def\fnum@table{\tablename~\thetable}
\makeatother

% Prevent slide breaks in the middle of a paragraph:
\widowpenalties 1 10000
\raggedbottom

\AtBeginPart{
  \let\insertpartnumber\relax
  \let\partname\relax
  \frame{\partpage}
}
\AtBeginSection{
  \ifbibliography
  \else
    \let\insertsectionnumber\relax
    \let\sectionname\relax
    \frame{\sectionpage}
  \fi
}
\AtBeginSubsection{
  \let\insertsubsectionnumber\relax
  \let\subsectionname\relax
  \frame{\subsectionpage}
}

\setlength{\parindent}{0pt}
\setlength{\parskip}{6pt plus 2pt minus 1pt}
\setlength{\emergencystretch}{3em}  % prevent overfull lines
\providecommand{\tightlist}{%
  \setlength{\itemsep}{0pt}\setlength{\parskip}{0pt}}
\setcounter{secnumdepth}{0}


\title{Generating Simple Loop Invariants in Whiley \\ ENGR489}
\author{David Barnett}
\date{Supervisor: Lindsay Groves}

\begin{document}
\frame{\titlepage}


\begin{frame}{Motivation}

Goals of the project
\begin{itemize}
\item{To implement generators for simple loop invariants in Whiley}
\item{\st{To be evaluated by SWEN224 Students}}
\end{itemize}

\end{frame}



\begin{frame}{Solution}

Generates 4 loop invariants by adding another analysis pass in the compiler

\begin{itemize}
\item Starting bound
\item Equal array length
\item Upper bound
\item Array iterative assignment
\end{itemize}

\end{frame}

\begin{frame}[fragile]{Starting Bound}

\begin{columns}

\begin{column}{0.5\textwidth}
\begin{itemize}
\item Generally for counting or index variables
\item Made up of 3 components
\begin{itemize}
    \item the variable (\code{i}) - defined outside of the loop
    \item an inequality (\code{>=}) - based on mutations of the variable in the loop
    \item starting value (\code{0}) - known value of variable before the loop
\end{itemize}

\item Handles branches

\end{itemize}
\end{column}

\begin{column}{0.5\textwidth}
\begin{verbatim}
int i = 0
while C
    where i >= 0:
    i = i + 1
\end{verbatim}
\end{column}

\end{columns}

\end{frame}


\begin{frame}[fragile]{Starting Bound - Limitations}

\begin{columns}

\begin{column}{0.5\textwidth}
\begin{itemize}
    \item Cannot determine when a branch will never be executed
    \item Only supports linear monotonic mutations with addition and subtract
        of constants and the counting variable.
\end{itemize}
\end{column}

\begin{column}{0.5\textwidth}
\begin{verbatim}
int i = 0

while i < |items|:
// can't be infer where i >= 0
  apply(items[i])
  if false:
    i = i * -1
  i = i + 1
\end{verbatim}
\end{column}

\end{columns}

\end{frame}


\begin{frame}[fragile]{Equal array length}

\begin{columns}

\begin{column}{0.5\textwidth}
\begin{itemize}
    \item Associates the length of arrays together by looking at assignments
    \item Used when mapping one array to another
    \item Supports common array construction, copying and array generator
\end{itemize}
\end{column}

\begin{column}{0.5\textwidth}
\begin{verbatim}
copy = items
// or
copy = [0; |items|]

while i <= |items|
    where |copy| == |items|:
    copy[i] = f(items[i])
    i = i + 1
\end{verbatim}
\end{column}
\end{columns}

\end{frame}

\begin{frame}[fragile]{Equal array length - Limitations}

\begin{itemize}
    \item Only supports locally assigned variables
    \item Does not support using pre-conditions to associate arrays
    \item Indirect associations, e.g. results of functions, constant size
\end{itemize}

\end{frame}



\begin{frame}[fragile]{Upper bound}

\begin{columns}

\begin{column}{0.5\textwidth}
\begin{itemize}
\item Identifies upper limit of a counting variable
\item Has three components
\begin{itemize}
    \item counting variable (\code{i}) - defined outside of the loop 
    \item mutation (\code{<=}) - determined by values of mutations in the loop
    \item upper limit (\code{C}) - found in the loop condition
\end{itemize}
\item Similar to Starting bound but focuses on amount changed than just
    if positive or negative.
\end{itemize}
\end{column}

\begin{column}{0.5\textwidth}
\begin{verbatim}
int i = 0
while i < C
    where i <= C:
    i = i + 1
\end{verbatim}
\end{column}

\end{columns}

\end{frame}



\begin{frame}[fragile]{Upper bound - Limitations}

\begin{columns}

\begin{column}{0.5\textwidth}
\begin{itemize}
\item Only supports mutations of $\pm 1$ the variable
\item Same as Starting bound
\begin{itemize}
    \item Cannot determine when a branch will never be executed
    \item Only supports linear monotonic mutations with addition and subtract
        of constants and the counting variable.
\end{itemize}
\end{itemize}
\end{column}

\begin{column}{0.5\textwidth}
\begin{verbatim}
int i = 0
while i < C:
    i = i + 2
\end{verbatim}
\end{column}

\end{columns}
\end{frame}



\begin{frame}[fragile]{Array Iterative Assignment}

\begin{columns}

\begin{column}{0.5\textwidth}

\begin{itemize}
    \item Simple iterative assignments to an array
    \item Has two components
    \begin{itemize}
        \item range of quantifier (\code{0..i}) - from the index variable and its
            starting value
        \item assignment to an element (\code{items[i] = f(i)}) - in the loop
            body
    \end{itemize}
    \item All instances of the index variable are replaced with the
        quantifier's variable.
\end{itemize}

\end{column}

\begin{column}{0.5\textwidth}
\begin{verbatim}
int i = 0
while i <= |items|
    where all { k in 0..i |
        items[k] == f(k) }:
    items[i] = f(i)
    i = i + 1
\end{verbatim}
\end{column}

\end{columns}

\end{frame}



\begin{frame}[fragile]{Array Iterative Assignment - Limitations}
\begin{columns}

\begin{column}{0.5\textwidth}
\begin{itemize}
    \item Each iteration needs an assignment
    \item No support for assignments in branches
    \item Destination array cannot be used in assignment
\end{itemize}
\end{column}

\begin{column}{0.5\textwidth}
\begin{verbatim}
while i <= |xs|:
    if i % 2 == 0:
        xs[i] = i - 1
    else:
        xs[i] = i
\end{verbatim}
\end{column}
\end{columns}
\end{frame}

\begin{frame}{Evaluation}

\begin{block}{Goal}
Provide evidence that the generated invariants will reduce the
number of invariants needed to be provided in the source code.
\end{block}

\begin{block}{Data sets}

\begin{itemize}
\item Whiley Compiler test suite
\item Past SWEN224 assignments and labs
\end{itemize}

\end{block}

\end{frame}


\begin{frame}{Evaluation tool}

\begin{block}{Assumptions}

\begin{itemize}
\item Code to be evaluated can be compiled and verified normally.
\item Loop invariant generator can be toggled.
\end{itemize}

\end{block}


\begin{block}{Steps}
\begin{itemize}
\item Breakdown - Normalizes the loop invariants in source code.
\item Minimize - Finds the smallest combonation of invariants to verify
the program.
\item Reporting - Counts and classifies the invariants present.
\end{itemize}
\end{block}

\end{frame}


\begin{frame}{Results}

Evaluation of the Whiley Compiler test suite

\begin{longtable}[]{@{}llll@{}}
\toprule
    Type & Control & Minimized & Generated\tabularnewline
\midrule
\endhead
Starting bound & 0 & 0 & 94\tabularnewline
Arrays Equal & 0 & 0 & 16\tabularnewline
Upper Bound & 0 & 0 & 17\tabularnewline
Iterative Assign & 0 & 0 & 8\tabularnewline
Total Generated & 0 & 0 & 135\tabularnewline
\hline
Source Invariants & 160 & 135 & 41\tabularnewline
\bottomrule
Total & 160 & 135 & 176\tabularnewline
\end{longtable}

\end{frame}



\begin{frame}{Conclusion}

The goal to implement generators of simple loop invariants was achieved.

\begin{itemize}
\item Can generate simple loop invariants
\item Reduced number of invariants programmer provides, by 70\% for
    Whiley Compiler test suite
\end{itemize}

\end{frame}


\begin{frame}

\begin{center}
    \Huge{Questions}
\end{center}

\end{frame}

\begin{frame}[fragile]{Additional examples}

\begin{verbatim}
function sum(int[] items) -> (int r)
requires all { i in 0..|items| | items[i] >= 0 }:
  int i = 0
  int sum = 0
  //
  while i < |items|
    where sum >= 0
    where i >= 0
    where i <= |items|:
    //
    sum = sum + items[i]
    i = i + 1
  //
  return sum
\end{verbatim}

\end{frame}

\begin{frame}[fragile]{Additional examples}

\begin{verbatim}
function instances(int[] items, int x) -> (int c):
  int i = 0
  int c = 0
  while i < |items|
    where i >= 0
    where c >= 0
    where i <= |items|:
    //
    if items[i] == x:
      c = c + 1
    //
    i = i + 1
  //
  return c
\end{verbatim}

\end{frame}

\begin{frame}[fragile]{Additional examples}

\begin{verbatim}
function add(int[] items, int delta) -> (int[] c):
  int i = 0
  int mod = items
  while i < |items|
    where i >= 0
    where i <= |items|
    where |mod| == |items|
    where all { j in 0..i | mod[j] = items[j] + delta }:
    //
    mod[i] = items[j] + delta
    i = i + 1
  //
  return mod
\end{verbatim}

\end{frame}

\begin{frame}[fragile]{Additional examples}

\begin{verbatim}
function min(int[] items) -> (int r)
requires |items| > 0
ensures all { i in 0..|items| | r <= items[i] }
ensures some { i in 0..|items| | r == items[i] }:
  int i = 1
  int m = items[0]
  while i < |items|
  where i >= 1
  where i <= |items|
  where all { j in 0..i | m <= items[j] }
  where some { j in 0..i | m == items[j] }:
    if m > items[i]:
      m = items[i]
    i = i + 1
  return m
\end{verbatim}


\end{frame}

\end{document}
