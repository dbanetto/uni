\begin{appendices}

\chapter{Project Details}\label{A:proj-details}

\section{Original Project Outline}\label{A:proj-outline}

Whiley is a programming language designed to support formal verification, and now using in teacher SWEN224 at VUW.
One of the biggest obstacles to practical use of Whiley,
especially in a teaching context, is the need for  the programmer to provide detailed loop invariants,
many of which are stating trivial or obvious properties.
The aim of this project is to develop techniques to generate simple loop invariants for Whiley,
so as to ease the burden on the programmer,
allowing them to focus on more interesting invariants that relate directly to
the properties the are trying to establish.

It is well that generating loop invariants is very hard, indeed, it is uncomputable in general.
The emphasis in this project is not on generality but on practicality.
The first step will be to look at some of the approaches that have been described in the literature,
and then implement one of them, or some combination, in Whiley, then look at ways in which this can be improved.
Ideally, then implementation will be available in time to be evaluated with SWEN224 students in Trimester 2.

\chapter{Code examples}\label{A:code-examples}

\todo{add more examples}

This appendix contains multiple examples of invariants as code samples.
Some of these samples are fully verified when available where some may
not due to issues present in Whiley. 
These can be run using Whiley's web IDE ( \url{http://whileylabs.com} ) .

% \section{Starting bound}

\begin{figure}[ht]
\begin{lstlisting}
function sum(int[] items) -> (int r)
requires all { i in 0..|items| | items[i] >= 0 }:
  int i = 0
  int sum = 0
  //
  while i < |items|
    where sum >= 0
    where i >= 0: // starting bound
    //
    sum = sum + items[i]
    i = i + 1
  //
  return sum
\end{lstlisting}
\caption{Example of a starting bound invariant.}
\end{figure}

\begin{figure}[ht]
\begin{lstlisting}
function count(int[] items, int x) -> (int c):
  int i = 0
  int c = 0
  while i < |items|
    where i >= 0
    where c >= 0:
    //
    if items[i] == x:
      c = c + 1
    //
    i = i + 1
  //
  return c
\end{lstlisting}
\caption{Example of a starting bound invariant with mutation within a conditional block.}
\end{figure}

% \section{Equal array length}

\begin{figure}[ht]
\begin{lstlisting}
type nat is (int x) where x >= 0

function toNat(int[] items) -> (nat[] r)
ensures |items| == |r|
ensures all { i in 0..|items| | (items[i] >= 0 ==> items[i] == r[i])
                             && (items[i] < 0 ==> r[i] == 0 ) }:
  nat[] result = [0;|items|]
  int i = 0
  while i < |items|
  where |result| == |items| // equal array length
  where i >= 0 // starting bound
  where all { j in i..|result| | result[j] == 0 }
  where all { j in 0..i | (items[j] >= 0 ==> items[j] == result[j]) 
                     && (items[j] < 0 ==> result[j] == 0 ) }:
    if items[i] >= 0:
      result[i] = items[i]
    i = i + 1
  //
  return result
\end{lstlisting}
\caption{Example equal array length with generated arrays.}
\end{figure}

% \section{Upper bound}

\begin{figure}[ht]
\begin{lstlisting}
function min(int[] items) -> (int r)
requires |items| > 0
ensures all { i in 0..|items| | r <= items[i] }
ensures some { i in 0..|items| | r == items[i] }:
  int i = 1
  int m = items[0]
  while i < |items|
  where i >= 1 // starting bound
  where i <= |items| // upper bound
  where all { j in 0..i | m <= items[j] }
  where some { j in 0..i | m == items[j] }:
    if m > items[i]:
      m = items[i]
    i = i + 1
  return m
\end{lstlisting}
\caption{Example of an upper bound invariant.}
\label{lst:upper-bound-min}
\end{figure}

% \section{Array iterative assignment}

\chapter{Evaluation}\label{A:eval}

\section{Failed to Evaluate}\label{A:eval-failed}

\begin{center}
\begin{tabular}{r}
    \csvautotabular[]{appendix/blacklist.csv}{}
\end{tabular}
\end{center}

% \section{Code for evaluation walk through}\label{A:eval-walkthrough}

% \subsection{Original code}

% \subsection{After broken down phase}

% \subsection{After minimized phase}

% \subsubsection{Without loop invariant generation}

% \subsubsection{With loop invariant generation}

% \subsection{Output of reporting phase}

% \todo{code blocks of a walk through of evaluation}

\end{appendices}
