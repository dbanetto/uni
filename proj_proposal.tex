%% $RCSfile: proj_proposal.tex,v $
%% $Revision: 1.3 $
%% $Date: 2016/06/10 03:44:08 $
%% $Author: kevin $

\documentclass[11pt, a4paper, twoside, openright]{report}

\usepackage{float} % lets you have non-floating floats
\usepackage{listings}
\usepackage{color}
\usepackage{url} % for typesetting urls

%  We don't want figures to float so we define
%
\newfloat{fig}{thp}{lof}[chapter]
\floatname{fig}{Figure}

\definecolor{grey}{rgb}{0.95,0.95,0.95}

\lstset{ %
    backgroundcolor=\color{grey},
    frame=single,
    numbers=left
}

%% These are standard LaTeX definitions for the document
%%
\title{DRAFT - Generating simple loop invariants for Whiley}
\author{David Barnett}

%% This file can be used for creating a wide range of reports
%%  across various Schools
%%
%% Set up some things, mostly for the front page, for your specific document
%
% Current options are:
% [ecs|msor|sms]          Which school you are in.
%                         (msor option retained for reproducing old data)
% [bschonscomp|mcompsci]  Which degree you are doing
%                          You can also specify any other degree by name
%                          (see below)
% [font|image]            Use a font or an image for the VUW logo
%                          The font option will only work on ECS systems
%
\usepackage[image,ecs]{vuwproject}

% You should specifiy your supervisor here with
%     \supervisor{Firstname Lastname}
% use \supervisors if there are more than one supervisor
\supervisors{Lindsay Groves}

% Unless you've used the bschonscomp or mcompsci
%  options above use
%   \otherdegree{OTHER DEGREE OR DIPLOMA NAME}
% here to specify degree

% Comment this out if you want the date printed.
\date{}

\begin{document}

% Make the page numbering roman, until after the contents, etc.
\frontmatter

%%%%%%%%%%%%%%%%%%%%%%%%%%%%%%%%%%%%%%%%%%%%%%%%%%%%%%%

\begin{abstract}
    This document gives some ideas about how to write a project
    proposal, and provides a template for a proposal. You should discuss
    your proposal with your supervisor.
\end{abstract}

%%%%%%%%%%%%%%%%%%%%%%%%%%%%%%%%%%%%%%%%%%%%%%%%%%%%%%%

\maketitle

%\tableofcontents

% we want a list of the figures we defined
%\listof{fig}{Figures}

%%%%%%%%%%%%%%%%%%%%%%%%%%%%%%%%%%%%%%%%%%%%%%%%%%%%%%%

\mainmatter

%%%%%%%%%%%%%%%%%%%%%%%%%%%%%%%%%%%%%%%%%%%%%%%%%%%%%%%

% Whiley is a programming language designed to support formal verification, and now using in teacher SWEN224 at VUW.
% One of the biggest obstacles to practical use of Whiley,
% especially in a teaching context, is the need for  the programmer to provide detailed loop invariants,
% many of which are stating trivial or obvious properties.
% The aim of this project is to develop techniques to generate simple loop invariants for Whiley,
% so as to ease the burden on the programmer,
% allowing them to focus on more interesting invariants that relate directly to the properties the are trying to establish.
%
% It is well that generating loop invariants is very hard - indeed, it is uncomputable in general.
% The emphasis in this project is not on generality but on practicality.
% The first step will be to look at some of the approaches that have been described in the literature,
% and then implement one of them, or some combination, in Whiley, then look at ways in which this can be improved.
% Ideally, then implementation will be available in time to be evaluated with SWEN224 students in Trimester 2.

\section*{1. Introduction}

In this section you should include a very brief introduction to the
problem to the problem and the project.

Your project proposal should cover the following points:

\begin{itemize}
    \item the engineering problem that you are going to solve;
    \item how you plan to solve your problem;
    \item how you intend to evaluate your solution; and
    \item any resource requirements for your project such as software,
        hardware or other resources that will be needed in the course of the
        project.
\end{itemize}


Your proposal should be not more than than 4 pages long with about 3 pages of content.

\section*{2. The Problem}

%%% WHAT

% problem: Whiley requires detailed loop invariants to be provided
% some of these are tedious and can be to make, index within bound of array, &
% the reader can see these from inspection (trivial and tedious).

The problem that this project aims to address is that the Whiley programming language
requires too much detail in loop invariants.
Whiley is a programming language built with support to create formally verified
software and aspect of this is proving loop invariants.
The current method of programming with loop invariants is that the programmer will supply
all of the invariants to be proved, example given in figure \ref{fig:whiley-ex-1}.
However this becomes cumbersome as some of the invariants are trivial or obvious
and the programmer has to begrudgingly retype or copy and paste the invariants
from the last loop.


\begin{figure}[h]
    \begin{lstlisting}
    function indexOf(int[] items, int item) -> (int r)
        ensures r == |items| || items[r] == item:

        int i = 0

        while i < |items|
        where i >= 0 && i <= |items|:

            if items[i] == item:
                break
            i = i + 1

        return i
    \end{lstlisting}
    \caption{Example of current Whiley code.}
    \label{fig:whiley-ex-1}
\end{figure}

% Why is this important:
% * Whiley is used in SWEN224 to teach the younglings about verified code & software correctness
% * Obstacle to practical use of Whiley, in a teaching context and industry

The Whiley language has a goal of being used practically.
The main focus for now is being used as a teaching tool in SWEN224.
With the removal of trivial and tedious loop invariants would be one 
less obstacle to this goal.

% Aim:
% * Make it easier to program by reducing the amount of tedious/trivial proofs (demonstrate less code)
% * generate loop invariants based off code using methods described in literature. (implement algorithms from literature)
% * to make it modular and be able to turn off generation to let
%   students think about these easy loop invariants (implement modularity)

This project has three main aims.
The primary aim is to implement a loop invariant generator that can remove
some trivial and tedious invariants, would be demonstrated through reduced required
amount of code. An example of this would being able to remove line 7 of figure \ref{fig:whiley-ex-1}.
The secondary aim is to research methods to generate invariants from literature 
and implement them into Whiley.
A supplementary aim is to make the generator a modular system where specific generators
could be turned on or off depending on the circumstances.

%%% Given text
%% In this section you should give a brief description of the problem
%% itself. You want to briefly explain the problem, why it is important
%% to solve the problem and define your project aims. After reading this
%% section, the reader should understand why it is a problem, believe
%% that it is important to solve and have a clear idea of the aims of
%% your project.
%
%% When describing the aims of the project, you should avoid vague,
%% unmeasurable words like `analyse', `investigate', `describe', and use
%% specific, measurable words like `implement', `demonstrate', `show',
%% `prove'.
%
%% For example:
%
%% \begin{itemize}
%%     \item[\bf Good] The aim of this project is to implement and evaluate a
%%         management system for network switches;
%% \end{itemize}
%% is much better than:
%% \begin{itemize}
%%     \item[\bf Bad] The aim of this project is to investigate management
%%         systems for network switches.
%% \end{itemize}
%
%% In the second case there is no idea of how much work is involved, and
%% you will never know whether you have finished. You and your supervisor
%% (and the markers of your project) may have very different ideas about
%% what such an `investigation' involves. Of course, it is possible that
%% the task you set yourself is not achievable, but if you are clear from
%% the outset this is less likely, and will more easily be corrected.


\section*{3. Proposed Solution}

%%% HOW

% plan:
%  * find papers about generating loop invariants
%  * read and understand them (out: bibliography)
%  * compare different methods of invariant generation (out: section in report) [ 70's vs. Modern techniques ] ( 2 weeks )
%  * read and understand the Whiley compiler ( 1 weel )
%  * write priminaly report ( est: 2 weeks )
%  * choose a small test project to compare the end-results with (e.g. game of tic-tac-toe) ( est: 1 week )
%  * design a modular system to enable loop invariant generation (out: design documents) ( est: 1 week )
%  * Implement the simplest loop invariant generator (out: code) ( est: 2 weeks )
%  * refactor sample project to use new generator (out: code) (est: 1 week)
%  * write report (out: report ) (est: 1 month )
%  * give presentation of work (out: presentation ) (est: 3 weeks)
%
%  note: accompany each invariant generator with a sample of before & after

The proposed solution for this problem has multiple parts.
The first part of the project is background reading with a focus on
methods to invariants generate loop invariants and software
verification. From these readings a bibliography will be produced
as well as a survey of different methods to generate loop invariants.
From the literature identify a selection of possible loop invariant
generators that would be implementable and useful to the aim of the project.
During which it would be reasonable to also be surveying the Whiley
compiler and theorem prover on how these methods could be integrated into the Whiley language
and what would be the most appropriate class of loop invariants to reduce.

% FIXME: I do not like how this part reads
From the methods that were identified and ranked to be most appropriate
will be implemented over the majority of the project.
The number of generators implemented will depend on how reasonable
it is to implement them in the limited time frame.


Below is an itemized list of each section of the project and includes an
estimation of time to complete in chronological ordering.
The time allocated of the twenty-five weeks from proposal hand-in to
final report is due is allocated as such. Note that not all of the
twenty-five weeks are allocated as slip is expected

% TODO: finalize how this looks
\begin{itemize}
    \item Produce a bibliography (est: 2 weeks) 
    \item Survey loop invariant literature for report (est: 1 weeks) 
    \item Survey the Whiley system (est: 1 weeks)
    \item Identify loop invariant generator methods to implement (est: 2 weeks)
    \item Implement loop invariant generators identified (est: 8 weeks)
    \item Write preliminary report (est: 2 weeks)

    \item Write final report (est: 3 weeks)
    \item Write and perform final presentation (est: 3 weeks)
\end{itemize}

%%% Given text
%% In this section you will explain how solve the problem, that is, how
%% you intend to carry the project out. At this early stage you need to
%% be both clear about what you are going to do and flexible enough to
%% adapt to changing circumstances. Making an early plan will not prevent
%% you from running into trouble, but it will help you identify possible
%% problems early. For example, if you intended to run an experiment in
%% HCI, you might realise early on that there would be problems gathering
%% sufficient data to get reliable results, and that you should re-design
%% your experiment.
%
%% Part of the planning process involves producing a timetable for when
%% the work is actually going to be done.
%
%% Each part of the project should produce some output. For example you
%% might plan on spending two weeks on background reading: the output of
%% this will be a bibliography, and a possibly a literature survey for
%% your report. Indeed, if you take the advice given above about having
%% specific, measurable goals, you should describe this part of your
%% project as:
%
%% \begin{itemize}
%%     \item[\bf Good] Produce bibliography (est: 2 weeks)
%% \end{itemize}
%% rather than
%% \begin{itemize}
%%     \item[\bf Bad] Background reading (est: 2 weeks)
%% \end{itemize}
%
%% Note that the methodology you outline here is dependent upon the type
%% of project and engineering area. You must talk to your supervisor
%% about this.

\section*{4. Evaluating your Solution}

%%% WITH

% comparing the code reduction in sample programs to see how much is saved
% with using the generators
% * compare performance ?
% * evaluate how strong / buggy / crashy the new generators are

To evaluate the solution built from the project is to compare code 
examples.
The comparison is focused on the reduction of the need of the programmer 
to provide loop invariants with and without a generator.


%%% Given text
%% In this section you will explain how you will evaluate your solution
%% once you have built it. The method of evaluation will be domain
%% specific. Your supervisor should provide guidance as to what is an
%% appropriate form of evaluation. For example, user testing for a HCI
%% project or performance measurement for a Network Engineering project.

\section*{5. Resource Requirements}

% resources:
%   * a computer
%   * access to food
%   * access to the literature
%   * a line of communication to the main developer of Whiley

It is assumed that the majority of the resources that are required  are already being
provided by Victoria University of Wellington.
This includes the access to a workstation computer and access to literature around the
topics of software verification and generating loop invariants.
An optional resource that would help the project would be communication
with Whiley's core developer, Dr. David J. Pearce.
No other resources have been identified.


%%% Given text
%% In this section you will detail any resource requirements such as
%% hardware, software or access to subjects.

%%%%%%%%%%%%%%%%%%%%%%%%%%%%%%%%%%%%%%%%%%%%%%%%%%%%%%%
\backmatter
%%%%%%%%%%%%%%%%%%%%%%%%%%%%%%%%%%%%%%%%%%%%%%%%%%%%%%%

%\bibliographystyle{ieeetr}
\bibliographystyle{acm}
\bibliography{sample}
\end{document}
