\chapter{Evaluation}\label{C:eval}
\comment{
The purpose of the evaluation section is to demonstrate whether you did \\
or did not satisfy the project goals or specifications. \\
If you can tie the performance of your design to some real specification then \\
your evaluation is much stronger. “My code runs in 29 ms” is much weaker than \\
“my code runs within the 30 ms window allowable for real-time performance of the. . . ”. \\


In many cases the evaluation of a project requires significant extra work to design and build test harnesses. \\
These should be explained so that the validity and scope of the evaluation can be understood. \\
Make liberal use of graphs and other figures. \\
They are much more effective at communicating many results than are words. \\
}

\section{Goal}

The goal of evaluating the loop invariant generators is to test objectively how much the
user provided loop invariants can be reduced.
This is achieved by finding removing redundant invariants until the minimal
set of user provided invariants is found with the rest generated.
From this the number of invariants the user will have to provide
in the source code can be counted against the number needed with 

\todo{include a step by step evaluation of a piece of code}

\section{Evaluation tool}\label{S:eval-tool}
\comment{explanation of the evaluation tool and what each piece does}

This section will give an overview of the evaluation tool used to
determine the types of loop invariants required.
Each \code{where} clauses is assumed to be an atomic invariant for the loop and is treated as a single unit.
An atomic invariant is one that cannot be logically decomposed into a smaller form. 
The tool has three stages to evaluate the code.
There is a pre-evaluation filtering that checks the code being evaluated does
contain a loop and verifies with the vanilla compiler.
The first stage breaks down the loop invariants to make them consistent
across writing styles ;
the second stage finds the minimal set of loop invariants required to verify ;
and, the final stage identifies what invariants are present in the minimal
verified code.


\subsection{Breakdown}
\comment{
What this does and how it benefits the evaluation \\
* normalises the data set \\
* removes the difference between the one-liner master and the laid out \\
thinker \\
* breaks up conjunctive expressions in the loop invariants \\
}

The breakdown step attempts to normalize the loop invariants of coding styles
and break down invariants into atomic invariants.
This is achieved by breaking up all top-level conjunctive invariants into
separate \code{where} clauses.
As this is functionally equivalent as all the \code{where} clauses are
conjunctively joined instead each invariant being separate makes it easier to count
and identify what they are.
An example of this in figure~\ref{lst:breakdown} as it shows much the large
invariant on line 1 is broken into lines 3 and 4.
The limitation of this technique that it only breaks up invariants that are
certainly can be broken up where other invariants that could be logically
re-arranged to be split up are not.
The result of this phase is written to disk to be used in the minimizing stage.

\begin{figure}[ht]
\begin{lstlisting}
where a < b && b < c || (f && g)
// breaks down to
where a < b
where b < c || (f && g)
\end{lstlisting}
\caption{Illustration of how the invariants are broken down}
\label{lst:breakdown}
\end{figure}


\subsection{Minimizing}
\comment{
What this does and how it benefits the evaluation \\
 * finds the smallest subset of loop invariants for the code to compile \\
 * tests with and without  \\
 * tests combinations of  \\
}

This stage of evaluation finds the smallest set of invariants
required for the program to verify.
Since generated loop invariants cannot be minimized their total cannot be
reduce, however the total number of invariants provided in the source code can
be.
This is achieved by trying to verify every combination of the \code{where} clauses.
This is computed from the empty set of invariants to the full set.
Once a single set is found to successfully verify it is flagged as the minimal
set of invariants and computation stops.
It is guaranteed to find some set of invariants that verify as the code being
evaluated is already known to verify with no modifications.
This process can be augmented to have the loop invariant generator enabled or
disabled so that they can be compared with minimal invariants.

\subsection{Reporting}
\comment{
What this does and how it benefits the evaluation \\
  include the eval script here to piece it all together \\
 * collects data about each run \\
 * explain how each data set is tested \\
  + files that do not contain while loops are removed via grep'ing them \\
  + are checked if they will compile and verify before any changes (removes original errors from data set) \\
}

The final stage of the evaluation is tool is to report on what loop invariants
are present in the code.
Each loop is processed and considered individually when counting the
invariants.
This includes what invariants are provided in the source code, what has been
loop invariants are generated and what generators were used.
This is achieved by counting the loop invariants in the AST after the loop
invariants generators have been applied.
The generator that was used to make the loop invariant is identified by
exploiting an implementation detail that generated invariants are tagged
with an attribute that describes why they were made.
From this the data used for the evaluation is generated.

\subsection{Known Issues}

There is some known issues with the evaluation tool and how it handles some
programs.
These issues stem from deeper issues that reside in the version of the compiler
that this project is based off.
One of these issues is that after minimizing the loop invariants the Whiley
code is written out to disk for inspection and to be used in the reporting
stage. There is an issue with Whiley's transformer that converts from AST
back to source where string literals are printed as an array of integers.
When this is written out the original and printed source is now different
and when attempted to compile again it fails.
This is caused by Whiley's internal representation of string types as an array
of integers which has recently been fix in an upstream release of Whiley.

\section{Data Sets}
\comment{
Explain that there is not much data to pick from \\
Discuss the data sets used and what problems they have \\
}

Since Whiley is not a widely used language there is not many large codebases
to test against.
The only choices of codebases large that are verified to test are either the
Whiley compiler test suite and student work from SWEN224.
Each of these data sets have issues with its contents and availability.

\subsection{Whiley Compiler Tests}

The Whiley compiler test suite is built to test the compiler's ability to
compile and verify particular features of Whiley.
This test suite was also use in the development of the loop invariant
generators to ensure that the correctness of the compiler.
The test suite has a large range of code and only a subset of them include
loop and fewer that are fully verified.
There is a total of 102 tests that include both loops and loop invariants to
evaluate.
The issue with this codebase is that it is not truly indicative of
natural coding since each program is for a test.
This however gives a large range of examples and cases to test against
as each program is different due to testing practises.

An implication of using the compiler test suite is that not all test
are pass.
This is due to some of the test presents in the suite are used for marking
known bugs in the compiler or theorem prover.
These programs are listed in the test suites and are removed from
the data set to prevent attempting to evaluate code that is known to
not work even for the unmodified compiler.

Even with filtering out the known bad test some of them failed to be evaluated.
These are listed in Appendix~\ref{A:eval-failed}.
They all manage to pass the checking phase of the evaluation but then later
failed due to other errors. The most common error was \textit{nothing to minimize}
which denotes that the code did not a combination of invariants that would verify, 
even using all provided.

\subsection{Assignments and Labs from SWEN224}

The assignments and labs from SWEN224 is also a large codebase that could be
used with permission.
The main issue with this codebase is its availability.
The advice on the topic was that to use and include the results of
evaluating student work would require the permission of individual students
which is infeasible to get a reasonable quantity.
The other issue with this codebase that each program will follow the same
structure as they are all solving the same problems.
This would result in duplicates of results between students.
In the end this data set was not sought or evaluated.

\section{Results}

\subsection{Whiley Compiler Tests}

Figure~\ref{tb:whiley-summary} shows the summary table of all loop invariants
reported when running the evaluation tool over the Whiley compiler test suite data set.
The results table has 7 categories of loop invariants reported during the
evaluation.
The \textit{total invariants} column shows total number of invariants from
either provided in the source code or generated.
The \textit{source invariants} column shows the number of invariants that had to be
provided in the source code to verify.
The \textit{source invariants} shows the number of invariants that had to be
provided in the source code to verify.
The \textit{generated invariants} shows the number total number of generated
invariants by all of the generators.
The last four columns are specifically each type of generator used to
create the invariant.

\begin{figure}[ht]
\begin{center}
\begin{tabular}[c]{r}
    \csvautotabular[]{appendix/summary.csv}{}
\end{tabular}
\end{center}
\caption{Results of Whiley compiler test suite data set}
\label{tb:whiley-summary}
\end{figure}

\section{Discussion}

The results from evaluating the Whiley compiler suite show that a majority of
the loop invariants in the codebase could be generated.
This is shown by comparing the \textit{source invariants} between the
minimized and generated results, from 135 to 41 invariants (70\% reduction).
From these results it is clear the loop invariant generators accomplish does
reduce amount of repetitive simple invariants that the user had to provide.

The most used loop generator is the starting bound generator.
It achieved a 70\% share of the total generated loop invariants with 94 out of
135 invariants.
This is expected as this invariant is found any time a variable
is used to linearly index an array.
Coupled with iterating arrays is also one of the most common uses of loops
makes the starting bound invariant the most generated invariant.

The total number of invariants present in the generated code is greater
than the minimized and control code.
There is 41 more loop invariants than the minimized code and 16 more than the control.
This was anticipated since the generated loop invariants are not minimized by either
the evaluation tool or when they are generated (see section \ref{s:essential-inv}).
The generated invariants are not tested if they are essential or not and this is
evidence of the generators creating more invariants than required.
An alternative interpretation is that the source invariants were not sufficiently broken down to
atomic invariants and it took multiple generated invariants to replace a single invariant.
This is possible since the method to breakup invariants was simply splitting conjunctive statements
where some invariants could be logically broken up with more complex algorithms.

The evaluation also reveals that not all of the loop invariants in the original
code was required.
This is shown by the decrease from 160 loop invariants to 135 loop invariants
in the minimized code.
This shows another application for the evaluation tool to be used to remove redundant
loop invariants.
As the tool can take minutes to run on complex examples or a large file this
would be best utilised as an invariant quality tool such as code coverage tools
are used in general development.
