%% $RCSfile: proj_report_outline.tex,v $
%% $Revision: 1.3 $
%% $Date: 2016/06/10 03:41:54 $
%% $Author: kevin $

\documentclass[11pt
              , a4paper
              % , twoside
              , openright
              ]{report}

\usepackage{float} % lets you have non-floating floats
\usepackage{listings}
\usepackage[usenames, dvipsnames]{color}
\usepackage{url} % for typesetting urls
\usepackage{mathtools}
\usepackage{pdfpages} % for including PDF's into the document
\usepackage[parfill]{parskip}
\usepackage{hyperref}
\usepackage[toc,page]{appendix}
\usepackage{csvsimple}

% definitions for code
\definecolor{grey}{rgb}{0.95,0.95,0.95}
\lstset{%
    backgroundcolor=\color{grey},
    frame=single,
    numbers=left
}

%
%  We don't want figures to float so we define
%
\newfloat{fig}{thp}{lof}[chapter]
\floatname{fig}{Figure}

%% These are standard LaTeX definitions for the document
%%                            
\title{Generating simple loop invariants for Whiley}
\author{David Barnett}

\usepackage[image,ecs,bschonscomp]{vuwproject}

% You should specifiy your supervisor here with
\supervisor{Lindsay Groves}
% use \supervisors if there is more than one supervisor

% Unless you've used the bschonscomp or mcompsci
%  options above use
\otherdegree{Bachelor of Engineering with Honours in Software Engineering}
% here to specify degree

% Comment this out if you want the date printed.
\date{}

% Personal LateX Commands
\newcommand{\code}[1]{\texttt{#1}}
\newcommand{\sst}{\textsuperscript{st}}
\newcommand{\snd}{\textsuperscript{nd}}
\newcommand{\sth}{\textsuperscript{th}}

\newcommand{\todo}[1]{
    {   \break
        \color{blue} 
        \fbox{\parbox{\textwidth}{
        \textbf{TODO}
        \textcolor{black}{\textit{#1}}
    }}}
}
\newcommand{\fixme}[1]{
    {   \break
        \color{red} 
        \fbox{\parbox{\textwidth}{
        \textbf{FIXME}
        \textit{#1}
    }}}
}
\newcommand{\comment}[1]{
    {  \color{ForestGreen}
       \fbox{\parbox{\textwidth}{#1}}
   }
}

% \newcommand{\fixme}[1]{}
% \newcommand{\todo}[1]{}
% \newcommand{\comment}[1]{}

\begin{document}

% Make the page numbering roman, until after the contents, etc.
\frontmatter

%%%%%%%%%%%%%%%%%%%%%%%%%%%%%%%%%%%%%%%%%%%%%%%%%%%%%%%

%%%%%%%%%%%%%%%%%%%%%%%%%%%%%%%%%%%%%%%%%%%%%%%%%%%%%%%

\begin{abstract}
Whiley is a programming language that can be used for formal software
verification \cite{whiley-origin}.
One aspect of formal software verification is providing loop
invariants.
Some loop invariants are common for verified software and become a burden to
provide the same loop invariant repetitively \cite{whiley-reflection}\cite{spec-usability}.
The primary goal of this project is to relieve some proof obligations from a
Whiley programmer by automatically providing some simple loop invariants.
Instead the loop invariants are generated by the Whiley Compiler itself.
This project does not attempt to generate loop invariants generally, as it is
known to be uncomputable, but generates invariants for specific common cases.

\end{abstract}

%%%%%%%%%%%%%%%%%%%%%%%%%%%%%%%%%%%%%%%%%%%%%%%%%%%%%%%

\maketitle

\chapter*{Acknowledgments}\label{C:ack}

% Any acknowledgments should go in here, between the title page and the table of contents.
% The acknowledgments do not form a proper chapter, and so don't get a number or appear in the table of contents.

This report would not of been possible without the support of Lindsay Groves as my supervisor and David Pearce for his help with Whiley.

\todo{say thanks to Lindsay and DJP, markers?}


\tableofcontents

% we want a list of the figures we defined
\listof{fig}{Figures}

%%%%%%%%%%%%%%%%%%%%%%%%%%%%%%%%%%%%%%%%%%%%%%%%%%%%%%%

\mainmatter

%%%%%%%%%%%%%%%%%%%%%%%%%%%%%%%%%%%%%%%%%%%%%%%%%%%%%%%

% individual chapters included here
\chapter{Introduction}\label{C:intro}
\comment{
The purpose here is to introduce the problem being solved,  to motivate \\
why it is a problem one should care about, and to outline the solution developed \\
during the project. \\


Remember:  the introduction is the first part of the report an examiner will read. \\

If he/she finishes reading it without a proper understanding of the \\
problem being solved or what has been done, then they will almost certainly  \\
struggle with understanding the remainder.  You should attempt to make the project goals \\
and associated specifications as clear and as quantifiable as possible. These goals and \\
specifications should inform everything else that follows, so it is important to establish \\
them in the examiners mind. \\

Should cover \\
* motivation of problem \\
* context of the problem \\
* explain why it is interesting \\
* hint at how the this project solves it \\
* overview   of how it unfolds
}

% outline
\todo{Get a better opening line}
\todo{replace intro paragraph, focus on the teaching aspect of this project}


A common burden of writing formally verified code is providing loop invariants.
This is exacerbated by simple and repetitive loop invariants are present throughout the program.
Whiley shared this problem and this project is to alleviate some of this burden by
generating some simple loop invariants.

\section{Motivation}

% Why
The primary motivation behind this project is to improve the experience writing
verified programs in Whiley.
Whiley is a programming language that supports formal software verification \cite{whiley-origin}.
Currently the process of writing loop invariants can be tedious with 
repetition of common invariants.
This project seeks to improve this by generating some of these tedious loop invariants.
Though it is known that generating loop invariants generally is uncomputable,
the focus is to generate practical loop invariants.
The most practical loop invariants to generate are loop invariants that
are simple in nature and common throughout verified code.
For example, the index variable is within the range of the array being
iterated through~\cite{loop-patterns}.

The secondary motivation for this project is to enable more programmers and
students to utilise formally verified languages.
The hypothesis of David J. Pearce, \textit{Verification is challenging when it requires a creative step},
describes his observations with the difficultly of students with loop invariants~\cite{spec-usability}.
The hypothesis covers the disconnect between loop invariants can have from the
post-condition with invariants about implementation details, such as an index
of an array.
This project seeks to relieve some of these problems meeting some of these
expectations by generating some loop invariants.
An example of a common misconception is that the verifier knows that a copied
array has the same size as its source.

\section{Solution}

% brief overview of solution
The goal of this project is to build an extension to
the Whiley compiler that generates simple loop invariants.
This includes the identification of simple loop invariants and being able
to generate a loop invariant from the source code of a Whiley program.
Since the Whiley compiler is an open source project this project will be
hosted in a fork of the compiler.

The approach of this project is to focus on identifying simple invariants
and implementing generator for them.
The simple invariants are identified by inspecting Whiley code for common
invariants. 
From there a method to recognise the conditions for the invariant and
its components are devised.
This allows a generator to be implemented that will recognise when
the invariant should be applied and how to build it from the components
it has found. 

To ensure that the implemented loop invariant generators do reduce
the number of invariants an experiment was conducted.
This experiment tested if generating of loop invariants 
would reduce the number of invariants needed to be provided.
It was run over a large Whiley code base and found that
it does greatly reduce the number of invariants in the code.

% 'simple'
The guideline used for a simple loop invariants in project is an invariant that is
obvious at inspection that an invariant both exists and would hold.
An example would be that while iterating through an array the index
will not go out of bounds of the array, such as figure~\ref{lst:whiley-ex-1}.
The purpose of having a definition of an simple invariant is to prevent
possible confusion when what looks to be unproven code is verified and to point
the scope of the project away from being too general, such as supporting
a rare scenarios such as indexing an array with a polynomial formula.
This leaves the more interesting invariants for the user to define.

\begin{figure}[ht]
\begin{lstlisting}
function indexOf (int[] items, int item) -> (int r)
ensures r == |items| || items[r] == item:

    int i = 0

    while i < |items|
      where i >= 0 && i <= |items|:

        if items[i] == item:
            break
        i = i + 1

    return i
\end{lstlisting}
\caption{Verifiable Whiley code to find index of an item in an array}
\label{lst:whiley-ex-1}
\end{figure}


\section{Overview}

This report is broken up into 6 chapters with each chapter detailing a separate part of the project.
Chapter~\ref{C:background} explains the background research, concepts and related work to this report.
Chapter~\ref{C:design} discusses the design limitation, goals and choices that went into designing the
solution for this project.
Chapter~\ref{C:impl} provides the details of the mechanisms behind the implemented 
loop invariant generators including examples of application and their limitations.
Chapter~\ref{C:eval} details of the experiment carried out with this project to determine if it is
successful, including the tool and data sets used.
Chapter~\ref{C:con} concludes the report with the conclusions drawn from the project,
related work and future work.  
\chapter{Background}\label{C:background}
\comment{
The background should cover any important terminology and/or
concepts used in the remainder of the report, and should demonstrate an
understanding of previous works which are relevant.

Remember: A good related work section does not just provide a list of previous works,
accompanied with short summaries.
Wherever possible it must extract real insight from these works, painting a
picture of how they relate to each other and the project.
}


This project draws from the research in the field of formal software verification.
The project is based off the Whiley language which supports formal software
verification.
The concept of loop invariants are a necessity is a key to this project.

Formal methods is an application of mathematics to prove software and hardware
systems against a given specification. \todo{cite this}
Formal software verification is the specific application to software with
the intention to prove its correctness.

\todo{fix this stuff}


\section{Core Concepts}

\subsection{Formal Software Verification Languages}
\comment{
 * Formal Software Verification \\
 * goals of class of languages (could be covered in Whiley)
}

Formal software verification is a field of software engineering with the
purpose to create software that is able to be mathematically verified to be
correct \cite{survey-formal-soft}.
Prof. Sir Tony Hoare issued a grand challenge to create a \textit{verifying compiler}
that guarantees the correctness of the program before running it \cite{Hoare-grand}.
There has been tool sets that expand languages to achieve this, such as
the SPARK tool set for the Ada \cite{spark-ada}\cite{spark-high-integ}.
Though this generally restricts the functionality of the language, for example
using Ada with SPARK disallows the use of pointers which makes general purpose
programming more challenging relative to C \cite{spark-ada}.
There are some languages that are designed and built from the ground up to
be a formal software language and have a \textit{verifying compiler}.
These languages include such as Whiley \cite{whiley-origin} and Dafny \cite{dafny-lang}.

\subsection{Language Semantics}
\comment{
    Explain the basics of reading language semantics \\
    explain how to read a semantics thing, with the $\frac{pre-cond}{post-con}$
}

\subsection{Whiley}
\comment{
Whiley \\
 * goals of the lang \\
 * samples of code \\
 * whiley design paper \\
}

The Whiley language is a general purpose language that has first class
support for formal software verification through its syntax and compiler
tool set \cite{whiley-origin}.
The design goal for Whiley is to make a platform for formal verifications \cite{whiley-origin}.
The tool set includes the compiler and an accompanying theorem prover.
The compiler translates the source code as an intermediate language which
can then be used with the theorem prover or transforms it into a target
language such as JVM byte code or JavaScript \cite{whiley-design}
\cite{wyil}.
Since Whiley is an open source project and the primary contributor, David
J.Pearce, is on staff at Victoria University of Wellington this made Whiley a
prime candidate language to extend.


\subsubsection{Whiley Semantics}
\comment{
Cover the key semantics used in the loop invariants \\
These are semantics that may differ languages so will be covered \\
* while loops \\
* where clause \\
* array copy \\
* array generator
}

In this section there are three key semantics used, while loops with where
clause, copy semantics and array generators.
The Whiley language specification book is available for the full
specification of the language \cite{whiley-spec}.

\paragraph{While loops with where clauses} are the mechanism for the programmer
to inform the compiler and theorem prover with loop invariants.
Figure~\ref{lst:whiley-where} shows an example of using where clauses with a
while loop in Whiley.
Each of the \code{where} clauses are joined together logically conjunctively.
Thus the logical loop invariant for the loop in figure~\ref{lst:whiley-where} would be:

$$(i >= 0 \wedge i <= 10) \wedge i < 10$$

\begin{figure}[ht]
\begin{lstlisting}
int i = 0
while i < 10
    where i >= 0
    where i <= 10:
    i = i + 1
\end{lstlisting}
    \label{lst:whiley-where}
    \caption{Example usage of a \code{where} clause in Whiley}
\end{figure}

\paragraph{Copy semantics} is a design decision of Whiley to
make clones of data instead of referencing the data on assignment \cite{whiley-array}.
Figure~\ref{lst:whiley-array-copy} shows an example of the variable
\code{b} making a copy of \code{a} and that mutating \code{a} later
will have no affect on the value of \code{b}.
This is an important feature of Whiley as it disassociates the data
of the variables after they are copied which allows for simpler verification
and disallows accidental mutations by unknown references \cite{whiley-array}.
Whiley does support explicit references to data but they also include
lifetimes to how long the references can live to remove null reference errors,
similar to Rust's lifetimes \cite{rust-lang}.

\begin{figure}[ht]
\begin{lstlisting}
a = [1, 2, 3]
b = a
assert a == b
assert |a| == |b|

a[0] = 2
assert a != b
assert |a| == |b|
\end{lstlisting}
    \label{lst:whiley-array-copy}
    \caption{Example an array copy in Whiley}
\end{figure}

\paragraph{Array generator} is a feature in Whiley to generate an array
of a specific element duplicated over a specific size.
Figure~\ref{lst:whiley-array-gen} shows an example of using an array generator
in Whiley and its associated results.
This is a short hand for creating arrays with a default value which is useful
to

\todo{finish me}

\begin{figure}[ht]
\begin{lstlisting}
a = [0; 3]

assert |a| == 3
assert |a| == [0, 0, 0]
\end{lstlisting}
    \label{lst:whiley-array-gen}
    \caption{Example an array generator in Whiley}
\end{figure}

\cite{whiley-design}
\cite{whiley-arrays}

\subsection{Loop Invariants}
\comment{
Loop invariants - overview of loop invariants
Theorem prover knows nothing about loop invariants - must give all \\
* must hold at entry \\
* must hold each iteration \\
* must hold on exit \\
* implies post-condition \\
* implied by pre-condition \\
}

Loop invariants are a means to provide a specification for a loop.
A loop invariant is required to prove the properties of a variable
that is altered during the loop.
It is used to as the basis for an inductive proof of a
property that is to be maintained.
The entry of the loop, each iteration of the loop and the exit of the
loop.

At the entry of the loop the loop invariant must hold.
This results in that the pre-condition of the loop must imply
the loop invariant.
This is illustrated in figure~\ref{lst:whiley-inv} where pre-condition
before entering the loop is clearly that \code{i} equals 0 and that holds
true for the invariants, in particular that \code{i >= 0}.

$$\text{Entry} \quad P \implies I$$

With each iteration of the loop the invariant must still hold.
Since the loop continue to iterate the condition of the loop
also holds. Thus each iteration can be described as a conjunction
between the invariant and the condition.

$$\text{Iteration} \quad I \wedge C $$

On exit of the loop the loop invariant must still hold.
Since the loop has ceased the loop condition no longer
holds. Thus an exit of the loop can be described
as the conjunction of loop invariant and the logical
not of the loop condition.
This results in an implication of the post-condition of the
loop.

$$\text{Exit} \quad I \wedge \neg{C} \implies Q$$

Figure~\ref{lst:whiley-inv} illustrates this as the
post condition of the loop can be reasoned to be the following
from the loop condition and the invariant:
$$\neg{ ( i < |items| ) } \wedge i \leq |items| \wedge i \geq 0$$

This can be simplified to down:

$$ i = |items| \wedge i \geq 0$$

This implies the post-condition of the function that $r = |items|$ with
the variable \code{i} renamed to \code{r}.
Each of the stages of a loop are similar to building an inductive proof \cite{invarints-classifiction}.

\subsubsection{Loop Invariants in Whiley}

In the Whiley language the concept of loop invariants are a 1\textsuperscript{st} class
construct.
Whiley gives syntax for providing \code{where} clauses to a \code{while}
statement.
Each of the \code{where} has a boolean expression which is the loop
invariant predicate.
Having multiple \code{where} clauses are equivalent to all the clauses are
conjunctively joined.
Figure~\ref{lst:whiley-inv} gives a full example of using a loop invariant
in Whiley. The \code{where} clauses on lines 6 and 7 are Whiley's syntax for
providing the loop invariant in the form of a predicate.

\begin{figure}[ht]
\begin{lstlisting}
function count(int[] items) -> (int r)
ensures r == |items|:
    int i = 0
    //
    while i < |items|
        where i >= 0
        where i <= |items|:
        i = i + 1
    //
    return i
\end{lstlisting}
    \label{lst:whiley-inv}
    \caption{Example of loop invariant in Whiley}
\end{figure}

\subsection{Classification of Loop Invariants}
\comment{
classification of loop invariants
 * proves property \\
 * hints to theorem prover \\
 * examples \\
}

There are two classifications of loop invariants.
These are bounding loop invariants and essential invariant
\cite{invarints-classifiction}.
With the classification of a loop invariant it helps to make
it clearer what the purpose of the invariant is trying to achieve.

A \textit{bounding loop invariant} is an invariant that is designed
to limit the values of a variable.
Figure~\ref{lst:whiley-inv} has an example of this at line 6, as the invariant
ensures that the variable \code{i} is bounded to be greater than zero
in every iteration of the loop.
Generally these types of invariants are used to ensure the body of the loop
is correct. In the case of figure~\ref{lst:whiley-ex-1} the loop invariant on line
7 do no effect the post-condition but ensures that the array indexing within
the loop is correct.

An \textit{essential invariant} is an invariant that is a step towards the
post-condition.
Figure~\ref{lst:whiley-inv} has an example of this at line 7,
as the invariant shows \code{i} must be less than or
equal to the length of the \code{items} array.
The loop invariant is a weaker form of the postcondition,
since the conjunction of the loop invariant and the negation
of the loop condition imply the postcondition \cite{invarints-classifiction}.
This class of invariant are more tailored to achieve
the goal of the loop than bounding invariants.

\subsection{Categories of Generated Loop Invariants}
\comment{
Methods to generate loop invariants \\
Generally generating loop invariants are impossible! \\
}

It has been proven that it is not possible to generate loop invariants
generally % TODO: cite this
so various methods have been developed to generate some invariants.
These methods can be sorted into one of two categories, static and dynamic.
This section will give an overview of static and dynamic methods of loop
invariants.

\subsubsection{Static Generation of Loop Invariants}

The static generation of loop invariants use the
semantics and analysis of the source code to generate the loop invariants.
This includes the use of language semantics and static analysis to

The key difference between the categories is
that invariants found statical hold true
but is restricted to what can be generated.
\cite{benderfinding}\cite{Leino2005LoopIO}.
The solution presented in this report uses
the principles of static methods to generate loop invariants.

\subsubsection{Dynamic Generation of Loop Invariants}

Dynamic generation of loop invariants involves static inference
with execution of the program to confirm the results.
This allows for a greater range of invariants to be inferred
at the cost of time to verify from testing potential invariants
\cite{infer-dynamic} \cite{infer-postconditions}.
This makes dynamic generation less suited to be apart of
the general compile-test-run development cycle common in general development.

\subsection{Static Analysis}
\comment{overview of static analysis techniques \\
e.g. partial ordered sets, lattices
}

\subsection{Inference Paradox}
\comment{
issue with inference creates correct code
by making specification from code

OK since it is at such a small scale with limited scope
}

\todo{invariant inference paradox}
\cite{infer-postconditions}

\subsection{Loop Design Pattern}

A loop design pattern is similar to an architecture pattern.
In that they can be classified into groups by there intent of
the pattern and how they are implemented.
For example, a strategy pattern is implemented to enable changes of
behaviour dependent on some input by utilising interfaces.
An example of a loop pattern would be a search through
an array for an element or processing all elements of an array
\cite{loop-patterns}.
A loop pattern can be described in the same terms of an architectural pattern
by providing the following outline from figure~\ref{l:design-pattern}
from \textit{A Pattern Language} \cite{pattern-lang}.

\begin{figure}[ht]
\begin{itemize}
    \item{\textbf{Examples} of pattern in application}
    \item{\textbf{Context} in which the pattern is used}
    \item{\textbf{Problem} that the pattern addresses or solves}
    \item{\textbf{Forces} requirements or trade offs that constrain possible
        solutions}
    \item{\textbf{Solution} the pattern}
    \item{\textbf{Resulting Context} having used the pattern, what is needed
        now}
\end{itemize}
    \caption{Elements of a design pattern\cite{pattern-lang}}
    \label{l:design-pattern}
\end{figure}


\cite{pattern-lang}

\section{Related Work}
\comment{Covers the literature review and compare current solution to similar
projects}

\fixme{Should this be in the conclusion section rather than here?}

\subsection{Dafny}
\comment{
explain how dafny is an alternative to Whiley in the Spec Lang space
has some features that are missing in Whiley that would be interesting to use

* declaring variables as increasing or decreasing \\
* defining variables / fields as ghosts, to only be used in specifications \\
dafny - other formal languages \\
}

An alternative to the Whiley language is the Dafny.
Dafny is a research language for a verifying compiler by Microsoft
\cite{dafny-lang}.
The language shares a majority of the same features as

\todo{dafny alternative}
\cite{dafny-started}
\cite{dafny-lang}
\cite{dafny-mech}

\subsection{Methods to Generate Loop Invariants}
\comment{explain how these methods create loop invariants}

\subsubsection{Manual Generation}
\comment{ Manual is always a thing }

\cite{broda-loop-tech}

\subsubsection{Backwards Propagation}

\cite{infer-postconditions}
\cite{infer-dynamic}

\subsubsection{Variable Aging}

\cite{infer-postconditions}
\cite{infer-dynamic}

\subsubsection{Coupling}

\cite{infer-postconditions}
\cite{infer-dynamic}

\subsubsection{Term dropping}

\cite{infer-postconditions}
\cite{infer-dynamic}

\cite{struct-induction}

\chapter{Design}\label{C:design}
% The aim here is to identify the key trade-offs in any design work you have undertaken.
% When solving a complex problem, there are normally many different
% approaches one can take — each with its own advantages and disadvantages.
% It is expected that students will have initially considered a range of different solutions,
% and will then have narrowed these down. The reasons why a particular approach was
% discounted should be documented here.
%
% Remember:
% Appropriate design notation (e.g. UML diagrams) can be very helpful in conveying different aspects of a design.
% It is vital that your design not be carried out in a vacuum.
% Your design should be motivated very clearly by your goals and specifications.
% Make sure that it is clear why you took the decisions that you did.
% Do not give the impression that you settled on a design because it “felt right”
% or that you tinkered around until you found something that worked. 

Since the Whiley compiler has multiple passes over the abstract syntax tree
(AST)
the current design adds a new pass. The compiler passes runs checks of properties
of the program such as the definite assignment of variables and sound usage of
types. This project adds an additional pass that searches for loop patterns and
inserts generated code into the AST.
This has been called the Loop Invariant Generator (LIG) pass.
This is akin to a modern macro system operating directly onto the AST.

The overall design of the Loop Invariant Generator is to collect the required
information before a loop to be used to infer loop invariants.
This is using forward propagation to determine if the
information collected is safe to use. Once the LIG reaches a loop
it tests the loop to see if it fits a loop pattern. This has been
designed using the strategy pattern, where each loop pattern is encoded
into a single strategy and produces a single invariant.
The invariant is known to be valid since they are built up via language
semantics.

This is in contrast with other state of the art loop invariant generators.
The general design of other loop invariant generators are to infer a set
of possible invariants and iterate through them trying to prove which
invariants are valid \cite{infer-dynamic} \cite{infer-postconditions}.
However, these loop invariant generators have the design to be used as a tool
to use with the source code where this project aims to integrate with the
compiler itself. Being apart of the compiler imposes multiple constraints.
This includes a constraint on keeping the impact on compile time to a
reasonable level.
Another constraint to consider is to not fail compilation for otherwise valid
code to ensure backwards compatibility and reduce surprise to the user.
Conversely should an error occur the resulting error message should not be
confusing, such as referring to generated code.


\section{Considerations}

\subsection{Usability}

\section{Limitations}

\section{}


\chapter{Implementation}\label{C:impl}
% The aim here is to explain the technical aspects of the project.
% The challenge is to ensure the text is clear and understandable.
% This is not easy, as ideas and concepts involved are often complex in nature.
% Nevertheless, if an examiners cannot understand how the implementation works,
% he/she cannot award marks for it.
% If this happens, the student is fault for poor communication.
%
% Remember:
% nothing is so complicated that it cannot be clearly explained.
% Classic pitfalls include:
%     * long convoluted sentences,
%     * use of long words,
%     * too much time spent discussing irrelevant details,
%     * poor organisation of sections, subsections and paragraphs,
%     * and too few diagrams or examples.


% TODO: re-do start of this
The first loop pattern is incrementing a variable
each iteration of the loop, most commonly used for indexes into an array to
iterate through it. See figure~\ref{lst:whiley-start} for a simple example.
The second loop pattern is making a copy of an array or creating an array with
the same length as another.
This is used generally when transforming every element in array into a separate
variable. See figure~\ref{lst:whiley-length} for a simple example.

\section{Starting value invariant}
% entry value
% TODO: Update to include lattice

From loop pattern of incrementing a variable each iteration of the loop
a invariant of the starting value can be inferred.
This invariant requires to know which variable is being
mutated in a simple manner each iteration,
the value of the variable at entry of the loop and if the mutation is an
increasing or decreasing sequence.
From this information an invariant be generated that encapsulates that the
variable will be increasing or decreasing from the initial value.
An example of the invariant generated is on line 5 of
figure~\ref{lst:whiley-start}.
The loop invariant inferred is a bounding invariant.

The definition of a simple mutation is restricted to an expression that only
includes addition and subtraction of constant values and the variable in question.
The mutation must be certain with each iteration so the variable must not be
modified inside branching statements such as \code{if} blocks or nested loops.
Since the mutation has to be simple it restricts them to linear monotonic
functions that is either strictly increasing or decreasing the variable with each iteration.
This is to keep the loop pattern simple and deterministic of knowing if the
mutation is increases or decreases the variable with each iteration.
An expression is checked if it is a simple mutation through static analysis of
the AST.

\begin{figure}
    $$f(x) \text{is a linear function}$$

    $$diff = f(f(0)) - f(0)$$

    \[
        diff \begin{cases}
            = 0 \quad f(x) \text{ is stationary}\\
            > 0 \quad f(x) \text{ is increasing}\\
            < 0 \quad f(x) \text{ is decreasing}\\
        \end{cases}
    \]
\label{math:simple-mutation}
\end{figure}

The increase or decreasing nature of the simple mutation it is determined by
executing the expression. An outline of the mathematical process is outlined
in figure~\ref{math:simple-mutation}, the expression is denoted with $f(x)$.
The equations show how the difference between applying the function twice and once on a base value is used to
determine if the function increases or decreases.
In the case that the expression is stationary the variable will not variate
between iterations and left alone.

\begin{figure}[ht]
\begin{lstlisting}
    ...
    int i = 0

    while i < |items|:
        // 'where i >= 0' is inferred
        apply(items[i])
        i = i + 1
    ...
\end{lstlisting}
\caption{Simple example of inferring starting bound of index}
\label{lst:whiley-start}
\end{figure}

With the identification of the variable with a simple mutation and knowing
if it decreases or increases each iteration an invariant can be made.
This is in the form of the variable on the left with either a less than or
equal to ($\leq$) or greater than or equal to ($\geq$) to the initial value on the
right.
In figure~\ref{lst:whiley-start} the variable \code{i} is clearly increasing
with each iteration due to line 7 and the inferred invariant, on line 5, is
obvious from the context.

\section{Equal length arrays invariant}
% array length

From a common pattern of making a copy of arrays or generating another array
with an equal length an invariant can be inferred.
This loop pattern is found when applying a function that changes the type of
the element or the user does not wish to update the original array.
Figure~\ref{lst:whiley-length} shows a simple example of applying a function
\code{apply ()} to each element of an array.
Generally the user would also need to provide an invariant that both arrays
have the same length to prevent possible out-of-range errors or prove a
post-condition.
The loop invariant inferred is a bounding invariant.

With this invariant it is detected that an array is declared with the same
length as another array.
This is achieved by inspecting the AST of the program
using forward propagation to find assignments to arrays.
Due to Whiley's copy semantics it is known that the assignment will result
in a clone of array and are distinct \cite{whiley-origin} \cite{whiley-arrays}.
This can also be achieved by finding a use of the array
generator syntax, see line 4 of figure~\ref{lst:whiley-length}.

The arrays that are shown to be equal in length to another array are
check to ensure the array size does not change.
This is achieved by checking that there is no assignments to either
of the two arrays involved either before entering the loop or
anyway inside the loop.
If there was an assignment it is no longer a simple to infer if
the arrays are equal size and an invariant is not generated.
However, an assignment to an element of the array is passable since it
is known that it won't change the array size just the contents.

\begin{figure}[ht]
\begin{lstlisting}
    ...
    int[] copy = items
    // or
    float[] copy = [0;|items|]
    while i < |items|:
        // 'where |copy| == |items|' is inferred
        copy[i] = apply(items[i])
    ...
\end{lstlisting}
\caption{Simple example of inferring array lengths are equal}
\label{lst:whiley-length}
\end{figure}

From this information an invariant is known and can be generated.
Given the source array and the array of known equal length the
invariant of the lengths are equal.
See figure~\ref{lst:whiley-length} line 5 for the invariant generated
from the example code.

\section{Loop Condition Ageing Invariant}

\subsection{Formal Definition}
\subsection{Example}
\subsection{Limitations}

\section{Array quantified assignment Invariant}

\subsection{Formal Definition}

\subsection{Example}

\subsection{Limitations}

\chapter{Evaluation}\label{C:eval}
% The purpose of the evaluation section is to demonstrate whether you did
% or did not satisfy the project goals or specifications.
% If you can tie the performance of your design to some real specification then
% your evaluation is much stronger. “My code runs in 29 ms” is much weaker than
% “my code runs within the 30 ms window allowable for real-time performance of the. . . ”.
%
% In many cases the evaluation of a project requires significant extra work to design and build test harnesses.
% These should be explained so that the validity and scope of the evaluation can be understood.
% Make liberal use of graphs and other figures.
% They are much more effective at communicating many results than are words.

\section{Goal}

\section{Evaluation tool}\label{S:eval-tool}
% explanation of the evaluation tool and what each piece does

\subsection{Breakdown}
% What this does and how it benefits the evaluation
% * normalises the data set
% * removes the difference between the one-liner master and the laid out
% thinker
% * breaks up conjunctive expressions in the loop invariants

\subsection{Minimizing}
% What this does and how it benefits the evaluation
%  * finds the smallest subset of loop invariants for the code to compile
%  * tests with and without 
%  * tests combonations of 

\subsection{Reporting}
% What this does and how it benefits the evaluation
%   include the eval script here to piece it all together
%  * collects data about each run
%  * explain how each data set is tested
%   + files that do not contain while loops are removed via grep'ing them
%   + are checked if they will compile and verify before any changes (removes original errors from data set)
%   + 


\section{Data Sets}
% Explain that there is not much data to pick from
% Discuss the data sets used and what problems they have

\subsection{Whiley Compiler Tests}

\subsection{Assignments and Labs from SWEN224}

\section{Results}

\subsection{Whiley Compiler Tests}


\begin{center}
\begin{tabular}[c]{r}
    \csvautotabular[]{appendix/summary.csv}{}
\end{tabular}
\end{center}

\subsection{Assignments and Labs from SWEN224}

\section{Discussion}

\chapter{Conclusions}\label{C:con}

\comment{
 Goal of chapter: 

  * reflect on the project, \\
  * how it was accomplished, \\
  * what it accomplished, \\
  * what could be done better next time, \\
  * what was learnt. \\
}

\section{Accomplishments}
\comment{talk about how the goal of the project was accomplished, loop invariants are generated for Whiley}

\section{Reflection}
\comment{How well did things go? Room for improvement? Is this appropriate?}

\section{Related Work}
\comment{Covers the literature review and compare current solution to similar projects}

\fixme{Should this be in the conclusion section rather than here?}

\subsection{Dafny}
\comment{
explain how dafny is an alternative to Whiley in the Spec Lang space
has some features that are missing in Whiley that would be interesting to use

* declaring variables as increasing or decreasing \\
* defining variables / fields as ghosts, to only be used in specifications
dafny - other formal language
}

An alternative to the Whiley language is the Dafny.
Dafny is a research language for a verifying compiler by Microsoft
\cite{dafny-lang}.
The language shares a majority of the same features as

\todo{dafny alternative}
\cite{dafny-started}
\cite{dafny-lang}
\cite{dafny-mech}

\subsection{Pattern-based loop invariant generation}

\cite{pattern-loop-inv}

\subsection{Methods to Generate Loop Invariants}
\comment{explain how these methods create loop invariants}

\subsubsection{Manual Generation}
\comment{Manual is always a thing }

\cite{broda-loop-tech}

\subsubsection{Backwards Propagation}

\cite{infer-postconditions}
\cite{infer-dynamic}

\subsubsection{Variable Aging}

\cite{infer-postconditions}
\cite{infer-dynamic}

\subsubsection{Coupling}

\cite{infer-postconditions}
\cite{infer-dynamic}

\subsubsection{Term dropping}

\cite{infer-postconditions}
\cite{infer-dynamic}

\cite{struct-induction}


\section{Future work}

This section outlines potential future work from the result of this project.
Each section gives a brief outline of an issue raised or faced during this
project and a possible project that could provide a solution to it.

\subsection{Detection of Duplicate Loop Invariant}
\comment{this could be in the form of lexical, AST, logical equivalence}

With the introduction of generated loop invariants the duplication of loop
invariants become an issue.
A future project could be to detect and inform the programmer of these
duplications.
This could be achieved by equating loop invariants with a range of methods.
The most simple method being lexical equivalence, comparing if code are same.
An increased complexity of attempting to detect equivalent syntax with the same
semantics, e.g. \code{i >= 0} is the same as \code{0 <= i}.
With the most complex being logical equivalence, e.g. \code{i > 0 || i == 0} is the same
as \code{i >= 0}.


\subsection{Expanding Loop Invariant Generators}
\comment{
	There is more loop patterns that could be used to generate loop invariants \\
	Not all possible cases are covered by the current implementations \\
	either find these new invariants by hand, or ML \\
}

Only some of the loop patterns in Whiley were identified and exploited for loop
invariants.
A future work could expand the number of loop invariant generators or reduce
the limitations in the current generators.
The current generators currently have some limitations could be lifted with
additional work, for example static evaluation of branches to reduce the
complexity of the code being generated on.
Since the identification of loop patterns is an intensive task an automated
approach could be identified, such as applying machine learning or other static
analysis techniques.

\subsection{Visitor Pattern for the Whiley Compiler}
\comment{
	Current implementations of Generators have cloned structures \\
	Code duplication could be reduced by using Visitor Pattern \\
	* decreases maintenance cost \\
	* makes adding new syntax / AST nodes easier \\
	* technical debt having so many duplicate AST descents \\
}

A technical issue of extending the Whiley compiler is the architecture forces
the programmer to support all statements and expressions.
This causes a large amount of duplication of code for each component of the
compiler to traverse the abstract syntax tree of any of Whiley or the internal
languages of WyIL and WyAL.
This accumulates technical debt for each component as they all need to be
updated to handle new syntax.
Currently the only method to detect if some syntax is not supported is via 
failing a runtime test that the syntax is handled.
A future work of implementing the visitor pattern for Whiley and the internal
languages.
This would improve the code quality by providing strict interfaces and
classes that handle the traversal of abstract syntax trees.
This removes the issue of each component having to re-implementing traversals
and the interfaces allow the user to implement methods to handle each type of
syntax either strictly for each syntax element (via Java interfaces) or
non-strictly (via extending Java classes).

\subsection{Ghost Variables in Whiley}
\comment{
	See Ada's `'Old`  syntax \\
	Declaration of `ghost` prefix to variable declaration so they are only usable \\
	in program verification \\
}

During the project the issue of lack of ghost variables in Whiley blocked some
invariants and limited others.
They were limited by the design constraint of not wanting to generate local
variables due to the runtime cost (see section~\ref{s:design-create-var} for
more detail ).
This cost can be avoided if Whiley had ghost variables, or syntax to refer to
the value of a variable prior to the loop without copying the value.
The future work would be to add syntax to the Whiley language to support this
feature and teach the specification pipeline how to handle the ghosted
variables.
The ghost variables do not only have to be limited to loops and could also be 
extended to include the values of variables before they enter functions or
cross other boundaries.
This has been achieved in other formal specification languages such as Ada and
Dafny.

\subsection{Generated Loop Invariants in WhileyWeb}
\comment{
	Update WhileyWeb to include generated loop invariants \\
	* show in the same manner as counter-examples \\
	* is the tool most likely used in future course work of the course \\
	* a pleasant UX in a web IDE \\
}

WhileyWeb is Whiley's web developer environment that is mostly used when
developing Whiley.
A future project could be to integrate the generated loop invariants with
the web interface such as the new counter-example feature has been added to
Whiley and WhileyWeb.
Generated loop invariants could be shown as warnings with code snippets of what
was generated and why they were generated.
The challenge of this project is to control the loop invariant
generators and display their results in a natural and user friendly way.
This would also provide an opportunity to do user testing on how generated
loop invariants affect student learning of loop invariants that could not be
completed with this project.
\appendix\chapter{Appendix}\label{C:appendix}

\section{Summary of Results}\label{A:summary}

\section{Evaluation Script}\label{A:eval}



%%%%%%%%%%%%%%%%%%%%%%%%%%%%%%%%%%%%%%%%%%%%%%%%%%%%%%%

\backmatter

%%%%%%%%%%%%%%%%%%%%%%%%%%%%%%%%%%%%%%%%%%%%%%%%%%%%%%%


\bibliographystyle{ieeetr}
%\bibliographystyle{acm}
\bibliography{report}


\end{document}
