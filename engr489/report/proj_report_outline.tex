%% $RCSfile: proj_report_outline.tex,v $
%% $Revision: 1.3 $
%% $Date: 2016/06/10 03:41:54 $
%% $Author: kevin $

\documentclass[11pt
              , a4paper
              % , twoside
              , openright
              ]{report}

\usepackage{float} % lets you have non-floating floats
\usepackage{listings}
\usepackage[usenames, dvipsnames]{color}
\usepackage{url} % for typesetting urls
\usepackage{mathtools}
\usepackage{pdfpages} % for including PDF's into the document
\usepackage[parfill]{parskip}
\usepackage{hyperref}
\usepackage[toc,page]{appendix}
\usepackage{csvsimple}
\usepackage{longtable}
\usepackage{booktabs}

% definitions for code
\definecolor{grey}{rgb}{0.95,0.95,0.95}
\lstset{%
    backgroundcolor=\color{grey},
    frame=single,
    numbers=left
}

%
%  We don't want figures to float so we define
%
\newfloat{fig}{thp}{lof}[chapter]
\floatname{fig}{Figure}

%% These are standard LaTeX definitions for the document
%%                            
\title{Generating simple loop invariants for Whiley}
\author{David Barnett}

\usepackage[image,ecs,bschonscomp]{vuwproject}

% You should specifiy your supervisor here with
\supervisor{Lindsay Groves}
% use \supervisors if there is more than one supervisor

% Unless you've used the bschonscomp or mcompsci
%  options above use
\otherdegree{Bachelor of Engineering with Honours in Software Engineering}
% here to specify degree

% Comment this out if you want the date printed.
\date{}

% Personal LateX Commands
\newcommand{\code}[1]{\texttt{#1}}
\newcommand{\sst}{\textsuperscript{st}}
\newcommand{\snd}{\textsuperscript{nd}}
\newcommand{\sth}{\textsuperscript{th}}

%\newcommand{\todo}[1]{
%    {   \break
%        \color{blue} 
%        \fbox{\parbox{\textwidth}{
%        \textbf{TODO}
%        \textcolor{black}{\textit{#1}}
%    }}}
%}
%\newcommand{\fixme}[1]{
%    {   \break
%        \color{red} 
%        \fbox{\parbox{\textwidth}{
%        \textbf{FIXME}
%        \textit{#1}
%    }}}
%}
%\newcommand{\comment}[1]{
%    {  \color{ForestGreen}
%       \fbox{\parbox{\textwidth}{#1}}
%   }
%}

 \newcommand{\fixme}[1]{}
 \newcommand{\todo}[1]{}
 \newcommand{\comment}[1]{}

\begin{document}

% Make the page numbering roman, until after the contents, etc.
\frontmatter

%%%%%%%%%%%%%%%%%%%%%%%%%%%%%%%%%%%%%%%%%%%%%%%%%%%%%%%

%%%%%%%%%%%%%%%%%%%%%%%%%%%%%%%%%%%%%%%%%%%%%%%%%%%%%%%

\begin{abstract}
Whiley is a programming language that can be used for formal software
verification \cite{whiley-origin}.
One aspect of formal software verification is providing loop
invariants.
Some loop invariants are common for verified software and become a burden to
provide the same loop invariant repetitively \cite{whiley-reflection}\cite{spec-usability}.
The primary goal of this project is to reduce the
repetition of loop invariants for the
Whiley programmer by automatically providing some simple loop invariants.
Instead the loop invariants are generated by the Whiley compiler itself.
This project does not attempt to generate loop invariants generally, as it is
known to not be computable, but generates invariants for specific common cases.

\end{abstract}

%%%%%%%%%%%%%%%%%%%%%%%%%%%%%%%%%%%%%%%%%%%%%%%%%%%%%%%

\maketitle

\chapter*{Acknowledgments}\label{C:ack}

Any acknowledgments should go in here, between the title page and the table of contents.
The acknowledgments do not form a proper chapter, and so don't get a number or appear in the table of contents.

\todo{say thanks to Lindsay and DJP, markers?}

\tableofcontents

% we want a list of the figures we defined
\listof{fig}{Figures}

%%%%%%%%%%%%%%%%%%%%%%%%%%%%%%%%%%%%%%%%%%%%%%%%%%%%%%%

\mainmatter

%%%%%%%%%%%%%%%%%%%%%%%%%%%%%%%%%%%%%%%%%%%%%%%%%%%%%%%

% individual chapters included here
\chapter{Introduction}\label{C:intro}
This chapter gives an introduction to the project report.

In Chapter \ref{C:us} we explain how to use this document, and the \texttt{vuwproject} style. In Chapter \ref{C:ex} we say some things about \LaTeX, and in Chapter \ref{C:con} we give our conclusions.
\chapter{Background}\label{C:background}
\comment{
The background should cover any important terminology and/or
concepts used in the remainder of the report, and should demonstrate an
understanding of previous works which are relevant.

Remember: A good related work section does not just provide a list of previous works,
accompanied with short summaries.
Wherever possible it must extract real insight from these works, painting a
picture of how they relate to each other and the project.
}


This project draws from research in the field of formal software verification.
The project is based on the Whiley language which supports formal software
verification.
The concept of loop invariants are a necessity is a key to this project.

Formal methods is an application of mathematics to prove software and hardware
systems against a given specification. \todo{cite this}
Formal software verification is the specific application to software with
the intention to prove its correctness.

\todo{fix this stuff}

\section{Core Concepts}

\subsection{Formal Software Verification Languages}
\comment{
 * Formal Software Verification \\
 * goals of class of languages (could be covered in Whiley)
}

Formal software verification is a field of software engineering with the
purpose to create software that is able to be mathematically verified to be
correct \cite{survey-formal-soft}.
Prof. Sir Tony Hoare issued a grand challenge to create a \textit{verifying compiler}
that guarantees the correctness of the program before running it \cite{Hoare-grand}.
There has been tool sets that expand languages to achieve this, such as
the SPARK tool set for the Ada \cite{spark-ada}\cite{spark-high-integ}.
Though this generally restricts the functionality of the language, for example
using Ada with SPARK disallows the use of pointers which makes general purpose
programming more challenging relative to C \cite{spark-ada}.
There are some languages that are designed and built from the ground up to
be a formal software language and have a \textit{verifying compiler}.
These languages include such as Whiley~\cite{whiley-origin} and Dafny~\cite{dafny-lang}.

\subsection{Whiley}
\comment{
Whiley \\
 * goals of the lang \\
 * samples of code \\
 * whiley design paper \\
}

The Whiley language is a general purpose language that has first class
support for formal software verification through its syntax and compiler
tool set~\cite{whiley-origin}.
The design goal for Whiley is to make a platform for formal verification~\cite{whiley-origin}.
The tool set includes the compiler and an accompanying theorem prover.
The compiler translates the source code to an intermediate language which
can then be used with the theorem prover or transforms it into a target
language such as JVM byte code or JavaScript~\cite{whiley-design}
\cite{wyil}.

The structure of a Whiley program is similar to most imperative languages
such as Java or Python.
In Whiley there are functions and methods with the point of difference being
that functions are pure and must return a result,
where methods can be impure and can optionally return a result
\cite{whiley-spec}\cite{whiley-design}.
Both functions and methods can have pre- and post-conditions using
\code{requires} and \code{ensures} respectfully as shown for the \code{min}
function in figure~\ref{lst:whiley-exp}.
Types can be declared as aliases of structures and can include type invariants,
such as the \code{nat} type in figure~\ref{lst:whiley-exp}.

\begin{figure}[ht]
\begin{lstlisting}
function min(int a, int b) -> (int m)
requires true
ensures m == a || m == b
ensures m <= a && m <= b:
    if a < b:
        return a
    else:
        return b

type nat is (int x) where x >= 0

method main():
    nat i = 0
    while i < 10:
        assert min(i, 0) == 0
        assert min(100, i) == i
        i = i + 1
\end{lstlisting}
\caption{Example of a programs structure in Whiley}
\label{lst:whiley-exp}
\end{figure}

\subsubsection{Whiley Features}
\comment{
Cover the key features used in the loop invariants generators \\
These are features differ languages so will be covered \\
* while loops where clause - covered in While and loop invs \\
* array copy \\
* array generator
}

This section explains two features of Whiley leveraged to generate loop invariants.
These are: copy semantics; and array generators~\cite{whiley-design}.
The Whiley language specification book is available for the full
specification of the language~\cite{whiley-spec}.

\paragraph{Copy semantics} is a design decision of Whiley to
make clones of data instead of referencing the data on assignment \cite{whiley-arrays}.
Figure~\ref{lst:whiley-array-copy} shows an example of the variable
\code{b} making a copy of \code{a} and that mutating \code{a} later
will have no affect on the value of \code{b}.
This is often utilised to make a copy of an array before mutating it
and use it as a reference point in an invariant~\cite{whiley-arrays}.
This is an important usability feature of Whiley as copying values
instead of taking references allows for simpler verification due to lack
of pointers and disallows accidental mutations by unknown references.
\footnote{
Though it is not related to this project,
Whiley does support explicit references to data.
They also include lifetimes to how long the references can live
to remove null reference errors, similar to Rust's lifetimes~\cite{rust-lang}.
}

\begin{figure}[ht]
\begin{lstlisting}
a = [1, 2, 3]
b = a
assert a == b
assert |a| == |b|

a[0] = 2
assert a != b
assert |a| == |b|
\end{lstlisting}
\caption{Example an array copy in Whiley}
\label{lst:whiley-array-copy}
\end{figure}

\paragraph{Array generator} is a feature in Whiley to generate an array
of a specific element duplicated over a specific size~\cite{whiley-spec}
\cite{whiley-arrays}.
Figure~\ref{lst:whiley-array-gen} shows an example of using an array generator
in Whiley and its associated results.
This is a short hand for creating arrays with a default value which is useful
to fill an array of arbitrary length with a default value.
A result of this feature is with static analysis is possible to infer the
length of the array at runtime by equating the array length to the value
from the generator, for example line 7 of figure~\ref{lst:whiley-array-gen}.

\begin{figure}[ht]
\begin{lstlisting}
a = [0; 3]

assert |a| == 3
assert |a| == [0, 0, 0]

b = [0; |a|]
assert |a| == |b|
\end{lstlisting}
    \caption{Example an array generator in Whiley}
    \label{lst:whiley-array-gen}
\end{figure}

\subsection{Loop Invariants}
\comment{
Loop invariants - overview of loop invariants
Theorem prover knows nothing about loop invariants - must give all \\
* must hold at entry \\
* must hold each iteration \\
* must hold on exit \\
* implies post-condition \\
* implied by pre-condition \\
}

Sir Prof.\ Tony Hoare described a general loop with an invariant in
the \textit{Rule of iteration} to be
portion of a program ($S$) that is repeated while a condition ($C$)
holds and an assertion ($I$) that holds on every
iteration~\cite{hoare-logic}~\cite{loop-inv-survey}.
Figure~\ref{lst:general-loop} is an example of this general loop with Whiley
syntax.
The loop invariant is then able to be used to prove properties of the
loop~\cite{loop-inv-survey}.
The loop invariant is a part of the three stages of the loop: entry;
iteration; and exit.

\begin{figure}[ht]
\begin{lstlisting}
while C:
    assert I
    S
\end{lstlisting}
    \caption{A general loop with an invariant $I$}
    \label{lst:general-loop}
\end{figure}

At the entry of the loop the loop invariant must hold.
This results in that the pre-condition of the loop ($P$) must imply
the loop invariant ($I$) as shown below.
This is illustrated in figure~\ref{lst:whiley-inv} where pre-condition
before entering the loop is clearly that \code{i} equals 0 and that holds
true for the invariants, in particular that \code{i >= 0}.

$$\text{Entry} \quad P \implies I$$

With each iteration of the loop the invariant is asserted some where
inside the loop body~\cite{hoare-logic}.
This is generally done at the start of the loop for while
statements~\cite{whiley-origin}.
With the invariant holding at the start of the loop the end of 
the loop must then restore the invariant.
Thus what is known in the loop is that the loop condition ($C$)
and the loop invariant ($I$) holds as shown below.

$$\text{Iteration} \quad C \wedge I $$

On exit of the loop the loop invariant must still hold.
Since the loop has ceased the loop condition no longer
holds. Thus an exit of the loop can be described
as the conjunction of loop invariant and the logical
not of the loop condition.
This results in an implication of the post-condition of the
loop.

$$\text{Exit} \quad I \wedge \neg{C} \implies Q$$

Figure~\ref{lst:whiley-inv} illustrates this as the
post condition of the loop can be reasoned to be the following
from the loop condition and the invariant:
$$\neg{ ( i < |items| ) } \wedge i \leq |items| \wedge i \geq 0$$

This can be simplified to down:

$$ i = |items| \wedge i \geq 0$$

This implies the post-condition of the function that $r = |items|$ with
the variable \code{i} renamed to \code{r}.
Each of the stages of a loop are similar to building an inductive proof \cite{invarints-classifiction}.

\subsubsection{Loop Invariants in Whiley}

In the Whiley language the concept of loop invariants are a 1\textsuperscript{st} class
construct.
Whiley gives syntax for providing \code{where} clauses to a \code{while}
statement.
Each of the \code{where} is a boolean expression which is the loop
invariant predicate.
Having multiple \code{where} clauses is equivalent to all the clauses are
conjunctively joined.
Figure~\ref{lst:whiley-inv} gives a full example of using a loop invariant
in Whiley. The \code{where} clauses on lines 6 and 7 are Whiley's syntax for
providing the loop invariant in the form of a predicate.

\begin{figure}[ht]
\begin{lstlisting}
function count(int[] items) -> (int r)
ensures r == |items|:
    int i = 0
    //
    while i < |items|
        where i >= 0
        where i <= |items|:
        i = i + 1
    //
    return i
\end{lstlisting}
\caption{Example of loop invariant in Whiley}
\label{lst:whiley-inv}
\end{figure}

\subsection{Classification of Loop Invariants}
\comment{
classification of loop invariants
 * proves property \\
 * hints to theorem prover \\
 * examples \\
}

There are two classifications of loop invariants.
These are bounding loop invariants and essential invariant~\cite{invarints-classifiction}~\cite{loop-inv-survey}.
With the classification of a loop invariant it helps to make
it clearer what the purpose of the invariant is trying to achieve.

A \textit{bounding loop invariant} is an invariant that is designed
to limit the values of a variable to assert correctness of the loop
body~\cite{loop-inv-survey}.
Figure~\ref{lst:whiley-inv-class} has an example of a bounding invariant on
line 6. This loop invariant does have any impact on finding an index
of the given item but is to assert that the variable \code{i} will index
the array correctly by being inside the valid range of 0 to the length of
the array (exclusively).
The loop condition provides the upper limit of the range of \code{i} but
the lower bound is needed to be provided as an invariant.
The most common bounding invariant observed is enforcing the starting bound
of a range.

An \textit{essential invariant} is an invariant that is an inductive step towards the
postcondition\cite{invarints-classifiction}~\cite{loop-inv-survey}.
Figure~\ref{lst:whiley-inv-class} has an example of an essential invariant on
line 7.
In this case loop invariant is a weakening of the postcondition to only
check that the item is not contained the array previously processed.
This invariant provides the essential inductive reasoning for when the item is
not contained inside the array.

\begin{figure}[ht]
\begin{lstlisting}
function indexOf(int[] items, int item) -> (int r)
ensures r >= 0 && r < |items| ==> items[r] == item
ensures r == |items| ==> all { i in 0..|items| | items[i] != item }:
    int i = 0
    while i < |items|
        where i >= 0
        where all { j in 0..i | items[j] != item }:
        if items[i] == item:
            break
        i = i + 1
    return i
\end{lstlisting}
\caption{Example of loop with bounding and essential invariants}
\label{lst:whiley-inv-class}
\end{figure}

\subsection{Categories of Generated Loop Invariants}
\comment{
Methods to generate loop invariants \\
Generally generating loop invariants are impossible! \\
}

It has been proven that it is not possible to generate loop invariants
generally \todo{cite this}
so various methods have been developed to generate some invariants.
These methods can be sorted into one of two categories, static and dynamic.
This section will give an overview of static and dynamic methods of loop
invariants.

\subsubsection{Static Generation of Loop Invariants}

The static generation of loop invariants use the
semantics and analysis of the source code to generate the loop invariants
\cite{java-static-symb}.
Some of the techniques used to statically generate loop invariants are:
weakening postconditions~\cite{infer-postconditions};  symbolic symbolic
execution~\cite{java-static-symb}; and structural
induction~\cite{struct-induction}.
These techniques are strict and conservative in generating loop invariants
to ensure what was generated is a valid invariant without being able to
verify it.

% The key difference between the categories is
% that invariants found statical hold true
% but is restricted to what can be generated.
% \cite{benderfinding}\cite{Leino2005LoopIO}.
% The solution presented in this report uses
% the principles of static methods to generate loop invariants.

\todo{EXAMPLES}

\subsubsection{Dynamic Generation of Loop Invariants}

Dynamic invariants are similar to dynamic program analysis
tools, they execute the program to find properties.
In the case of dynamically generating loop invariants a
common techniques to generate a set of candidate invariants
and modify the program to include them for verification.
The techniques used to generate the candidate loop invariants
can be derived from less-strict static analysis techniques and
from observing the execution of the program.
This allows a greater range of invariants to be inferred at
the cost of the time taken to test if the candidate invariants are
valid~\cite{infer-dynamic}\cite{infer-postconditions}.


%\subsection{Loop Design Pattern}
%
%A loop design pattern is similar to an architecture pattern.
%In that they can be classified into groups by there intent of
%the pattern and how they are implemented.
%For example, a strategy pattern is implemented to enable changes of
%behaviour dependent on some input by utilising interfaces.
%An example of a loop pattern would be a search through
%an array for an element or processing all elements of an array
%\cite{loop-patterns}.
%A loop pattern can be described in the same terms of an architectural pattern
%by providing the following outline from figure~\ref{l:design-pattern}
%from \textit{A Pattern Language} \cite{pattern-lang}.
%
%\begin{figure}[ht]
%\begin{itemize}
%    \item{\textbf{Examples} of pattern in application}
%    \item{\textbf{Context} in which the pattern is used}
%    \item{\textbf{Problem} that the pattern addresses or solves}
%    \item{\textbf{Forces} requirements or trade offs that constrain possible
%        solutions}
%    \item{\textbf{Solution} the pattern}
%    \item{\textbf{Resulting Context} having used the pattern, what is needed
%        now}
%\end{itemize}
%    \caption{Elements of a design pattern\cite{pattern-lang}}
%    \label{l:design-pattern}
%\end{figure}


\documentclass[12pt]{article}


\usepackage[margin=1in]{geometry}  % set the margins to 1in on all sides
\usepackage{graphicx}              % to include figures
\usepackage{minted}

\author{David Barnett (ID:300313764)}
\title{NWEN243 Lab 4 - Routing - Design}
\date{}

\begin{document}

\maketitle

\begin{minted}[linenos]{c}
    if (0 == 1) {
        x = &y;
    } else {
        y = &x;
    }
\end{minted}

\end{document}

\chapter{Implementation}\label{C:impl}
% The aim here is to explain the technical aspects of the project.
% The challenge is to ensure the text is clear and understandable.
% This is not easy, as ideas and concepts involved are often complex in nature.
% Nevertheless, if an examiners cannot understand how the implementation works,
% he/she cannot award marks for it.
% If this happens, the student is fault for poor communication.
%
% Remember:
% nothing is so complicated that it cannot be clearly explained.
% Classic pitfalls include:
%     * long convoluted sentences,
%     * use of long words,
%     * too much time spent discussing irrelevant details,
%     * poor organisation of sections, subsections and paragraphs,
%     * and too few diagrams or examples.


% TODO: re-do start of this
The first loop pattern is incrementing a variable
each iteration of the loop, most commonly used for indexes into an array to
iterate through it. See figure~\ref{lst:whiley-start} for a simple example.
The second loop pattern is making a copy of an array or creating an array with
the same length as another.
This is used generally when transforming every element in array into a separate
variable. See figure~\ref{lst:whiley-length} for a simple example.

\section{Starting value invariant}
% entry value
% TODO: Update to include lattice

From loop pattern of incrementing a variable each iteration of the loop
a invariant of the starting value can be inferred.
This invariant requires to know which variable is being
mutated in a simple manner each iteration,
the value of the variable at entry of the loop and if the mutation is an
increasing or decreasing sequence.
From this information an invariant be generated that encapsulates that the
variable will be increasing or decreasing from the initial value.
An example of the invariant generated is on line 5 of
figure~\ref{lst:whiley-start}.
The loop invariant inferred is a bounding invariant.

The definition of a simple mutation is restricted to an expression that only
includes addition and subtraction of constant values and the variable in question.
The mutation must be certain with each iteration so the variable must not be
modified inside branching statements such as \code{if} blocks or nested loops.
Since the mutation has to be simple it restricts them to linear monotonic
functions that is either strictly increasing or decreasing the variable with each iteration.
This is to keep the loop pattern simple and deterministic of knowing if the
mutation is increases or decreases the variable with each iteration.
An expression is checked if it is a simple mutation through static analysis of
the AST.

\begin{figure}
    $$f(x) \text{is a linear function}$$

    $$diff = f(f(0)) - f(0)$$

    \[
        diff \begin{cases}
            = 0 \quad f(x) \text{ is stationary}\\
            > 0 \quad f(x) \text{ is increasing}\\
            < 0 \quad f(x) \text{ is decreasing}\\
        \end{cases}
    \]
\label{math:simple-mutation}
\end{figure}

The increase or decreasing nature of the simple mutation it is determined by
executing the expression. An outline of the mathematical process is outlined
in figure~\ref{math:simple-mutation}, the expression is denoted with $f(x)$.
The equations show how the difference between applying the function twice and once on a base value is used to
determine if the function increases or decreases.
In the case that the expression is stationary the variable will not variate
between iterations and left alone.

\begin{figure}[ht]
\begin{lstlisting}
    ...
    int i = 0

    while i < |items|:
        // 'where i >= 0' is inferred
        apply(items[i])
        i = i + 1
    ...
\end{lstlisting}
\caption{Simple example of inferring starting bound of index}
\label{lst:whiley-start}
\end{figure}

With the identification of the variable with a simple mutation and knowing
if it decreases or increases each iteration an invariant can be made.
This is in the form of the variable on the left with either a less than or
equal to ($\leq$) or greater than or equal to ($\geq$) to the initial value on the
right.
In figure~\ref{lst:whiley-start} the variable \code{i} is clearly increasing
with each iteration due to line 7 and the inferred invariant, on line 5, is
obvious from the context.

\section{Equal length arrays invariant}
% array length

From a common pattern of making a copy of arrays or generating another array
with an equal length an invariant can be inferred.
This loop pattern is found when applying a function that changes the type of
the element or the user does not wish to update the original array.
Figure~\ref{lst:whiley-length} shows a simple example of applying a function
\code{apply ()} to each element of an array.
Generally the user would also need to provide an invariant that both arrays
have the same length to prevent possible out-of-range errors or prove a
post-condition.
The loop invariant inferred is a bounding invariant.

With this invariant it is detected that an array is declared with the same
length as another array.
This is achieved by inspecting the AST of the program
using forward propagation to find assignments to arrays.
Due to Whiley's copy semantics it is known that the assignment will result
in a clone of array and are distinct \cite{whiley-origin} \cite{whiley-arrays}.
This can also be achieved by finding a use of the array
generator syntax, see line 4 of figure~\ref{lst:whiley-length}.

The arrays that are shown to be equal in length to another array are
check to ensure the array size does not change.
This is achieved by checking that there is no assignments to either
of the two arrays involved either before entering the loop or
anyway inside the loop.
If there was an assignment it is no longer a simple to infer if
the arrays are equal size and an invariant is not generated.
However, an assignment to an element of the array is passable since it
is known that it won't change the array size just the contents.

\begin{figure}[ht]
\begin{lstlisting}
    ...
    int[] copy = items
    // or
    float[] copy = [0;|items|]
    while i < |items|:
        // 'where |copy| == |items|' is inferred
        copy[i] = apply(items[i])
    ...
\end{lstlisting}
\caption{Simple example of inferring array lengths are equal}
\label{lst:whiley-length}
\end{figure}

From this information an invariant is known and can be generated.
Given the source array and the array of known equal length the
invariant of the lengths are equal.
See figure~\ref{lst:whiley-length} line 5 for the invariant generated
from the example code.

\section{Loop Condition Ageing Invariant}

\chapter{Evaluation}\label{C:eval}
% The purpose of the evaluation section is to demonstrate whether you did
% or did not satisfy the project goals or specifications.
% If you can tie the performance of your design to some real specification then
% your evaluation is much stronger. “My code runs in 29 ms” is much weaker than
% “my code runs within the 30 ms window allowable for real-time performance of the. . . ”.
%
% In many cases the evaluation of a project requires significant extra work to design and build test harnesses.
% These should be explained so that the validity and scope of the evaluation can be understood.
% Make liberal use of graphs and other figures.
% They are much more effective at communicating many results than are words.

\section{Goal}

\section{Evaluation tool}\ref{S:eval-tool}
% explanation of the evaluation tool and what each piece does

\subsection{Breakdown}
% What this does and how it benefits the evaluation
% * normalises the data set
% * removes the difference between the one-liner master and the laid out
% thinker
% * breaks up conjunctive expressions in the loop invariants

\subsection{Minimizing}
% What this does and how it benefits the evaluation
%  * finds the smallest subset of loop invariants for the code to compile
%  * tests with and without 
%  * tests combonations of 

\subsection{Reporting}
% What this does and how it benefits the evaluation
%   include the eval script here to piece it all together
%  * collects data about each run
%  * explain how each data set is tested
%   + files that do not contain while loops are removed via grep'ing them
%   + are checked if they will compile & verify before any changes (removes original errors from data set)
%   + 


\section{Data Sets}
% Explain that there is not much data to pick from
% Discuss the data sets used and what problems they have

\subsection{Whiley Compiler Tests}

\subsection{Assignments and Labs from SWEN224}

\section{Results}

\subsection{Whiley Compiler Tests}

\subsection{Assignments and Labs from SWEN224}

\section{Discussion}

\chapter{Conclusions}\label{C:con}

\comment{
 Goal of chapter: 

  * reflect on the project, \\
  * how it was accomplished, \\
  * what it accomplished, \\
  * what could be done better next time, \\
  * what was learnt. \\
}

\section{Accomplishments}
\comment{talk about how the goal of the project was accomplished, loop invariants are generated for Whiley}

This project has accomplished its primary goal.
The burden of providing repetitive simple loop invariants has been reduced by generating simple loop invariants.
This was achieved by studying simple loop invariants to find a general loop pattern where they arise.
From this pattern a loop invariant generator to discover when these patterns apply
then adding the invariant to the loop.
The success of this approach is evident from the 70\% overall reduction of the number of loop invariants 
that the programmer is required to provide in the Whiley compiler test suite.
From this it is evident that the burden on the programmer to write
these repetitive simple loop invariants have been reduced.

The secondary goal of allowing more programmers and student utilise formally verifiable
languages was not evaluated.
Thus it is undetermined that the introduction of generated loop invariants would achieved this goal.
This is because of a change in direction of the project to focus on generating simple loop
invariants over the study of the implications of generated invariants on programmers (see Section~\ref{s:requirements}).

\section{Related Work}
\comment{Covers the literature review and compare current solution to similar projects}

The problem of loop invariants have been a target for generation for decades.
Over this time other formally variable languages

\todo{THIS SECTION}

\fixme{Should this be in the conclusion section rather than here?}

\subsection{Dafny}
\comment{
	explain how dafny is an alternative to Whiley in the Spec Lang space
	has some features that are missing in Whiley that would be interesting to use
	
	* declaring variables as increasing or decreasing \\
	* defining variables / fields as ghosts, to only be used in specifications
	dafny - other formal language
}

Dafny is language a similar language to Whiley as they both are design with the
goal to be  being formally verifiable.
Dafny is a research language for a verifying compiler built by Microsoft
Research~\cite{dafny-lang}.
Dafny shares a majority of features with Whiley and makes advances in others.

One feature Dafny is to provide a termination metric.
This tells Dafny that the given expression will decrease with each iteration. 
In turn this proves that with each iteration a step towards its termination is achieved \cite{dafny-started}\cite{dafny-lang}.
By default this is inferred from analysis of the loop.
This feature is similar in nature to a key mechanism behind the starting bound and upper bound
generators, being able to determine if a variable increases or decreases
(see Section~\ref{s:sequence-dir}). 

Another feature of Dafny is variables can be annotated to be ghost variables
and the previous value of a variable can be referenced.
In Dafny a ghost variable is one that can only be used for verifications.
This has no impact on the runtime of the program \cite{dafny-started}\cite{dafny-lang}.
The previous value of a variable can use accessed by using a built-in
function, \code{old}, that returns the previous value in specifications.
These features would simplify the equal length and iterative assignment
generators since they could generate zero-cost clones of arrays to use
with their invariants. 


\subsection{Methods to Generate Loop Invariants}
\comment{explain how these methods create loop invariants}

There has been research into automatic inference of loop invariants over
the last three decades.
Over this time a large range of methods has been devised to infer certain
classes of invariants \cite{infer-dynamic}\cite{infer-postconditions}\cite{struct-induction}.
This section will include two tool that utilise these methods,
\text{gin-pink} \cite{infer-postconditions}, and \text{gin-dyn} \cite{infer-dynamic}.

These include logically weakening the post-conditions by back-propagating them
to a loop invariant \cite{infer-postconditions}\cite{infer-dynamic} and
matching a loop to a common pattern\cite{pattern-loop-inv}, similar to what is achieved in this
report.

\todo{this should really be completed at some point}

\subsubsection{Gin-Pink}

In \textit{Inferring Loop Invariants using Postconditions} 
the \textit{gin-pink}'s proposed \cite{infer-postconditions}.
This is a dynamic loop invariant generator similar to the generator
present in this report.
\textit{gin-pink} generates loop invariants by considering the program's code
and the supplied post-conditions.
To achieve this a number of candidate invariants are generated from the 
post-conditions of procedures.
The candidate invariants are generating by applying heuristics to the
post-conditions to form weaker conditions to be used.
Each candidate invariant is tested to see if it holds.
This contrasts with this report as they are both aiming to generate loop
invariants but focus on different foundations.
Where \textit{gin-pink} is aiming for more general invariants 
this report is focused on known repetitive invariants.

\subsubsection{Gin-Dyn}

In \textit{Automating Full Functional Verification of Programs with Loops} the
\textit{gin-dyn} tool is proposed \cite{infer-dynamic}.
Like \textit{gin-pink} this tool generates a range of candidate invariants,
however \textit{gin-dyn} takes it a step further by also using generated tests
as a biases to create loop invariants.


% \cite{infer-dynamic}

% \subsubsection{Variable Aging}

% \cite{infer-postconditions}
% \cite{infer-dynamic}

% \subsection{Pattern-based loop invariant generation}

% \cite{pattern-loop-inv}

% \subsubsection{Coupling}

% \cite{infer-postconditions}
% \cite{infer-dynamic}

% \subsubsection{Term dropping}

% \cite{infer-postconditions}
% \cite{infer-dynamic}

% \cite{struct-induction}

\section{Future work}

This section outlines potential future work from the result of this project.
Each section gives a brief outline of an issue raised or faced during this
project and a possible project that could provide a solution to it.

\subsection{Detection of Duplicate Loop Invariant}
\comment{this could be in the form of lexical, AST, logical equivalence}

With the introduction of generated loop invariants the duplication of loop
invariants become an issue.
A future project could be to detect and inform the programmer of these
duplications.
This could be achieved by equating loop invariants with a range of methods.
The most simple method being lexical equivalence, comparing if code are same.
An increased complexity of attempting to detect equivalent syntax with the same
semantics, e.g. \code{i >= 0} is the same as \code{0 <= i}.
With the most complex being logical equivalence, e.g. \code{i > 0 || i == 0} is the same
as \code{i >= 0}.


\subsection{Expanding Loop Invariant Generators}
\comment{
	There is more loop patterns that could be used to generate loop invariants \\
	Not all possible cases are covered by the current implementations \\
	either find these new invariants by hand, or ML \\
}

Only some of the loop patterns in Whiley were identified and exploited for loop
invariants.
A future work could expand the number of loop invariant generators or reduce
the limitations in the current generators.
The current generators currently have some limitations could be lifted with
additional work, for example static evaluation of branches to reduce the
complexity of the code being generated on.
Since the identification of loop patterns is an intensive task an automated
approach could be identified, such as applying machine learning or other static
analysis techniques.

\subsection{Visitor Pattern for the Whiley Compiler}
\comment{
	Current implementations of Generators have cloned structures \\
	Code duplication could be reduced by using Visitor Pattern \\
	* decreases maintenance cost \\
	* makes adding new syntax / AST nodes easier \\
	* technical debt having so many duplicate AST descents \\
}

A technical issue of extending the Whiley compiler is the architecture forces
the programmer to support all statements and expressions.
This causes a large amount of duplication of code for each component of the
compiler to traverse the abstract syntax tree of any of Whiley or the internal
languages of WyIL and WyAL.
This accumulates technical debt for each component as they all need to be
updated to handle new syntax.
Currently the only method to detect if some syntax is not supported is via 
failing a runtime test that the syntax is handled.
A future work of implementing the visitor pattern for Whiley and the internal
languages.
This would improve the code quality by providing strict interfaces and
classes that handle the traversal of abstract syntax trees.
This removes the issue of each component having to re-implementing traversals
and the interfaces allow the user to implement methods to handle each type of
syntax either strictly for each syntax element (via Java interfaces) or
non-strictly (via extending Java classes).

\subsection{Ghost Variables in Whiley}
\comment{
	See Ada's `'Old`  syntax \\
	Declaration of `ghost` prefix to variable declaration so they are only usable \\
	in program verification \\
}

During the project the issue of lack of ghost variables in Whiley blocked some
invariants and limited others.
They were limited by the design constraint of not wanting to generate local
variables due to the runtime cost (see section~\ref{s:design-create-var} for
more detail ).
This cost can be avoided if Whiley had ghost variables, or syntax to refer to
the value of a variable prior to the loop without copying the value.
The future work would be to add syntax to the Whiley language to support this
feature and teach the specification pipeline how to handle the ghosted
variables.
The ghost variables do not only have to be limited to loops and could also be 
extended to include the values of variables before they enter functions or
cross other boundaries.
This has been achieved in other formal specification languages such as Ada and
Dafny \cite{dafny-lang}.

\subsection{Generated Loop Invariants in WhileyWeb}
\comment{
	Update WhileyWeb to include generated loop invariants \\
	* show in the same manner as counter-examples \\
	* is the tool most likely used in future course work of the course \\
	* a pleasant UX in a web IDE \\
}

WhileyWeb is Whiley's web developer environment that is mostly used when
developing Whiley.
A future project could be to integrate the generated loop invariants with
the web interface such as the new counter-example feature has been added to
Whiley and WhileyWeb.
Generated loop invariants could be shown as warnings with code snippets of what
was generated and why they were generated.
The challenge of this project is to control the loop invariant
generators and display their results in a natural and user friendly way.
This would also provide an opportunity to do user testing on how generated
loop invariants affect student learning of loop invariants that could not be
completed with this project.

\subsection{Solving verification problems in Whiley}

An issue that arose during this project is a lack of substantially sized code base
of Whiley code.
To mitigate this a future project to implement a large range of 
verification problems in Whiley would be beneficial.
The range of problems could be from simple singular functions such an element
contained in a list and scaling up to implementations of data structures and related
functions.
This project would also provide an opportunity to identify problem areas
in Whiley's programming experience that then could be investigated or improved.
\begin{appendices}

\chapter{Project Details}\label{A:proj-details}

\section{Original Project Outline}\label{A:proj-outline}

Whiley is a programming language designed to support formal verification, and now using in teacher SWEN224 at VUW.
One of the biggest obstacles to practical use of Whiley,
especially in a teaching context, is the need for  the programmer to provide detailed loop invariants,
many of which are stating trivial or obvious properties.
The aim of this project is to develop techniques to generate simple loop invariants for Whiley,
so as to ease the burden on the programmer,
allowing them to focus on more interesting invariants that relate directly to
the properties the are trying to establish.

It is well that generating loop invariants is very hard, indeed, it is uncomputable in general.
The emphasis in this project is not on generality but on practicality.
The first step will be to look at some of the approaches that have been described in the literature,
and then implement one of them, or some combination, in Whiley, then look at ways in which this can be improved.
Ideally, then implementation will be available in time to be evaluated with SWEN224 students in Trimester 2.

\chapter{Code examples}\label{A:code-examples}

\todo{add more examples}

This appendix contains multiple examples of invariants as code samples.
Some of these samples are fully verified when available where some may
not due to issues present in Whiley. 
These can be run using Whiley's web IDE ( \url{http://whileylabs.com} ) .

\section{Starting bound}

\begin{figure}[ht]
\begin{lstlisting}
function sum(int[] items) -> (int r)
requires all { i in 0..|items| | items[i] >= 0 }:
  int i = 0
  int sum = 0
  //
  while i < |items|
    where sum >= 0
    where i >= 0: // starting bound
    //
    sum = sum + items[i]
    i = i + 1
  //
  return sum
\end{lstlisting}
\caption{Example of a starting bound invariant.}
\end{figure}

\begin{figure}[ht]
\begin{lstlisting}
function count(int[] items, int x) -> (int c):
  int i = 0
  int c = 0
  while i < |items|
    where i >= 0
    where c >= 0:
    //
    if items[i] == x:
      c = c + 1
    //
    i = i + 1
  //
  return c
\end{lstlisting}
\caption{Example of a starting bound invariant with mutation within a conditional block.}
\end{figure}

\section{Equal array length}

\begin{figure}[ht]
\begin{lstlisting}
type nat is (int x) where x >= 0

function toNat(int[] items) -> (nat[] r)
ensures |items| == |r|
ensures all { i in 0..|items| | (items[i] >= 0 ==> items[i] == r[i])
                             && (items[i] < 0 ==> r[i] == 0 ) }:
  nat[] result = [0;|items|]
  int i = 0
  while i < |items|
  where |result| == |items| // equal array length
  where i >= 0
  where all { j in i..|result| | result[j] == 0 }
  where all { j in 0..i | (items[j] >= 0 ==> items[j] == result[j]) 
                     && (items[j] < 0 ==> result[j] == 0 ) }:
    if items[i] >= 0:
      result[i] = items[i]
    i = i + 1
  //
  return result
\end{lstlisting}
\caption{Example equal array length with generated arrays.}
\end{figure}

\section{Upper bound}

\section{Array iterative assignment}

\chapter{Evaluation}\label{A:eval}

\section{Failed to Evaluate}\label{A:eval-failed}

\begin{center}
\begin{tabular}{r}
    \csvautotabular[]{appendix/blacklist.csv}{}
\end{tabular}
\end{center}

\section{Code for evaluation walk through}\label{A:eval-walkthrough}

\subsection{Original code}

\subsection{After broken down phase}

\subsection{After minimized phase}

\subsubsection{Without loop invariant generation}

\subsubsection{With loop invariant generation}

\subsection{Output of reporting phase}

\todo{code blocks of a walk through of evaluation}

\end{appendices}



%%%%%%%%%%%%%%%%%%%%%%%%%%%%%%%%%%%%%%%%%%%%%%%%%%%%%%%

\backmatter

%%%%%%%%%%%%%%%%%%%%%%%%%%%%%%%%%%%%%%%%%%%%%%%%%%%%%%%


\bibliographystyle{ieeetr}
%\bibliographystyle{acm}
\bibliography{report}


\end{document}
