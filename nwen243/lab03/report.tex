\documentclass[11pt]{article}
\title{\textbf{NWEN243 Lab 3 Report}}
\author{David Barnett ID:300313764}

\usepackage{graphicx}
\usepackage{array}
\usepackage{epstopdf}
\usepackage{float}
\floatstyle{boxed}
\restylefloat{figure}
\DeclareGraphicsExtensions{.eps}

\begin{document}

\maketitle

\section{Abstract}\label{abstract}

This experiment aims to investigate a the efficiency of a range of
routing protocols in a simulated network. While changing the different network
maps the methods of the protocols and maximum hops were also changed. Through
the experiment it was shown that the \emph{Flooding 3} routing algorithm was
most efficient. Changing the maximum hops showed no significant difference.

\section{Introduction}\label{introduction}

In this experiment affect of different routing protocols with a range of
variety of topologies. This was composed by with firstly comparing all
the different routing protocols with each of the topology. Secondly
comparing the efficiency of using the 3rd routing protocol
against two topologies with varying maximum hops.

\section{Method}\label{method}

\paragraph{}
The method used to simulate the different protocols over a range of topologies
running a cnet, a network simulator program by University of Western
Australia, simulation with the given protocol and map. Repeated the simulation
for all protocols on all maps for a simulated time of 10 minutes on 56Kbps
network.

\paragraph{}
The method to obtain the varying hop was done by increasing protocol 3's
max hop count the running the same simulation over the New Zealand +
Australian and World maps.

\paragraph{}
The three protocols used in this experiments used different strategies to
route the packets through the network. The protocol dubbed \emph{Flooding 1}
use a method of which once it receives a packet it then sends it across all
links. \emph{Flooding 2} use a similar method to \emph{Flooding 1} but
did not send the packet back along the wire it was sent from. While
\emph{Flooding 3} has the same tactic as \emph{Flooding 2} to start with
but as packets are exchanged the node builds a table of links over time.
The node then uses the table to appropriate to send the packet down.


\section{Results}\label{results}

\paragraph{}
The results are in terms of delivery efficiency as a percentage of number of
packets sent that are received by the destination.

\subsection{Protocols \& Topography}\label{}

\includegraphics[width=0.5\textwidth]{res/NORTH-ISLAND-MAP-plot.png}
\includegraphics[width=0.5\textwidth]{res/NEWZEALAND-MAP-plot.png}
\includegraphics[width=0.5\textwidth]{res/NZ-AUS-MAP-plot.png}
\includegraphics[width=0.5\textwidth]{res/WORLD-MAP-plot.png}

Note: The results from the simulation of the WORLD.MAP is not complete as the
simulation stopped mid way in each run for the different protocols due to congestion
errors at different intervals for each routing protocol.

\subsection{Varying Hops}\label{hops}

\includegraphics[width=0.9\linewidth]{res/NZ-AUS-MAP-hops-plot.png}

\includegraphics[width=0.9\linewidth]{res/WORLD-MAP-hops-plot.png}

\section{Discussion}\label{discussion}

\paragraph{}
The results show a variety of properties of each of the tested routing
protocols. In the case of the North Island map there was not much difference
between the three protocols. Most likely due the small size of the map, only 2 nodes,
would prevent the problems that the larger maps would of have.


\paragraph{}
The results from the New Zealand map show that the \emph{Flooding 1} and
\emph{Flooding 2} are constant with their efficiency and both considerably
lower than \emph{Flooding 3}. The limiting factor of

\paragraph{}
The log-like curve featured in all of the large maps that \emph{Flooding 3}
displays is attributed to its major design difference, learning the network as
it is used. This enables the node to sends out packets more intelligently
than the other routing protocols and reduce congestion in the network allowing
for more successful communications on the network.

\paragraph{}
The New Zealand and Australia map show the gap between \emph{Flooding 2}
\emph{Flooding 3} grow larger as the node count and complexity of the node
graph grows. In these results and the New Zealand map results show that
\emph{Flooding 2} preforms better than \emph{Flooding 1}. This would be
contributed to the slight difference in design where \emph{Flooding 2} does not
send back the packet from where it came from. In a map that is more akin to a
linked list, like the North Island and (but less so) the New Zealand map,
allowing the packet to propagate to the destination while causing less
network congestion than \emph{Flooding 1}.


\paragraph{}
The World map shows the limits of the routing protocols. As the world map is by
far the most complex of all the given maps an equally complex routing system
would be need to route the packs without the failure that all three of the
protocols has shown. All three of the protocols failed to complete their
simulations due to the network falling over completely. \emph{Flooding 1}
failed the quickest due to its rapid production of packets filling the network.
While \emph{Flooding 2} did a bit better \emph{Flooding 3} managed to keep the
network longer than other protocols. It However did not seem like it made pass
the setup phase as shown by the logarithmic nature of the curve. As it is
setting up it behaves in a similar manner to \emph{Flooding 2} so a crash is
possible.

\paragraph{}
Re-running the World and New Zealand + Australia maps with \emph{Flooding 3}
plus having the doubled hop limit, maximum nodes the packet would be routed through
before being dropped, showed no significant difference in the both maps. The
variance between the 4 and 8 hops were below 0.2\%. This was most likely due to
the affect of the congestion due to the protocol out weighed or neutralized the
benefits of being able to send a packet over a greater distance.

\section{Conclusion}\label{conclusion}

The experiment showed the limitation of the a simple routing algorithms. It
also found that small changes like not sending the packet back from where it
came can cause large improvements in delivery efficiency and even greater if
collecting data from the network to build a table to help make routing
decisions. The experiment show as well that that increasing the maximum hops
did not have an affect on the efficiency of \emph{Flooding 3}.

\end{document}
