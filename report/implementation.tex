\chapter{Implementation}\label{C:impl}
\comment{
The aim here is to explain the technical aspects of the project.
The challenge is to ensure the text is clear and understandable.
This is not easy, as ideas and concepts involved are often complex in nature.
Nevertheless, if an examiners cannot understand how the implementation works,
he/she cannot award marks for it.
If this happens, the student is fault for poor communication. \\

Remember: \\
nothing is so complicated that it cannot be clearly explained.
Classic pitfalls include: \\
    * long convoluted sentences, \\
    * use of long words, \\
    * too much time spent discussing irrelevant details, \\
    * poor organisation of sections, subsections and paragraphs, \\
    * and too few diagrams or examples.
}


\todo{Looks best to move the next sections into impl chapter}

\comment{
	Biggest design decision  was at what level are the invariants going to be inferred
	then where are they going to be generated
}
\section{Approaches to Loop Invariant Generation}

The principle design decision of this project was at
what point in the Whiley Compiler pipeline
to generate the loop invariants.
There are three candidate stages in the Whiley Compiler
that the input program will go through in the process of
being verified, the list is not necessarily in order of occurrence:

\begin{itemize}
	\item{Whiley Intermediate Language}
	\item{Whiley Assertion Language}
	\item{Whiley's Abstract Syntax Tree}
\end{itemize}

Each of these stages have different representations of the Whiley program and
associated information about the program itself.
In the resulting solution the design approach taken was to build upon
Whiley's abstract syntax tree.


\begin{figure}
\noindent\begin{minipage}{.45\textwidth}
\begin{lstlisting}[caption={Whiley Code},frame=tlrb,numbers=none]{Name}
i = 0
while i < |items|:
  i = i + 1
\end{lstlisting}
\end{minipage}\hfill
\begin{minipage}{.45\textwidth}
\todo{PUT A UML DIAGRAM OF THE AST}
\end{minipage}
\caption{Example of a Whiley snippet and UML of its AST}
\label{lst:design-whiley}
\end{figure}

\subsection{Whiley Intermediate Language (WyIL)}
\comment{
	possible method to create tool on WyIL
	
	What could be achieve with this:
	
	* is designed to be a binary version of the code
	
	* WyIL goal is to make intermediate language between ALL targets of Whiley
	
	* AST converted into loosely connected values via integers
	
	* less information readily available
}

The Whiley Intermediate Language (WyIL) is a binary representation of a Whiley program
that can be easily transformed into other targets such as languages or
instruction sets \cite{wyil}.
It is a register based language with no nested expressions
that also includes blocks to group together statements such as the body of a loop.
This allows the language to keep a majority of its structure with the
exception of local variables and intermediate values becoming indistinguishable
without analysis.

The structure of internal API to query and modify WyIL code impedes a
generator's ability to match it against a loop pattern.
This is from the internal structure of a statement in WyIL where each
statement is loosely defined with op codes and blocks.
By generating the loop invariants as additional byte code the ability to
inspect the result and debug them are hampered relative to operating on the
AST.  This is because the ability to decompile from WyIL to Whiley could
be impeded which prevents the user from being able to inspect the generated
code.
WyIL can also be decompiled to a representation similar to assembly, see
figure~\ref{lst:design-wyil} for an example of Whiley code compared to its
intermediate language equivalent.
This example shows the amount of information available about the variables
in the program and their usage in the loop.
This approach was not selected due to the disconnect between loop patterns
in the programmer source code to their representation in WyIL.


\begin{figure}
\noindent\begin{minipage}{.45\textwidth}
\begin{lstlisting}[caption={Whiley Code},frame=tlrb,numbers=none]{Name}
i = 0
while i < |items|
where i >= 0:
  i = i + 1
\end{lstlisting}
\end{minipage}\hfill
\begin{minipage}{.45\textwidth}
\begin{lstlisting}[caption={WyIL Code},frame=tlrb,numbers=none]{Name}
const %1 = 0
loop modifies %1
invariant :
const %2 = 0
ifge %1, %2
goto lab_0
fail
.lab_0
return
ifge %1, %0
goto lab_1
add %1 = %1, %0
.lab_1
return %1
\end{lstlisting}
\end{minipage}
\caption{Example of simple Whiley snippet as WyIL}
\label{lst:design-wyil}
\end{figure}

\subsection{Whiley Assertion Language (WyAL)}
\comment{
	WyAL, what it is:
	
	* changes would be more dramatic
	
	* Does not have the same level of control
	
	* hosted inside the theorem prover
}

The Whiley Assertion language (WyAL) is derived from WyIL to create
assertions for the Whiley automated theorem prover \cite{wyal}.
WyAL follows a similar syntax to regular Whiley except that the program
is a collection of boolean expressions with additional constructs such as
if branches and universal quantifiers.
No program logic or structure is held in WyAL but are kept in the source
WyIL program.

Figure~\ref{lst:design-wyal} shows an example of WyAL code generated from
the given Whiley code.
This figure shows the assertions for a simple loop invariant as well as the
assertion message which are directly displayed to the user on failure.
The main benefit of generating loop invariants at the WyAL level is the
impact of the new loop invariants are contained to only the verification
process and prevents a generator author from breaking a design consideration,
such as not creating variables, as the Whiley code is immutable by this point.
However, since the lack of program logic in WyAL requires the use of a WyIL program
as backing the same issues with WyIL are present with working on WyAL.
This includes the difficultly to correlate between the high-level structure
in a Whiley program to the resulting WyIL code.

\begin{figure}
\noindent\begin{minipage}{.45\textwidth}
\begin{lstlisting}[caption={Whiley Code},frame=tlrb,numbers=none]{Name}
i = 0
while i < |items|
where i >= 0:
  i = i + 1
\end{lstlisting}
\end{minipage}\hfill
\begin{minipage}{.45\textwidth}
\begin{lstlisting}[caption={WyAL Code},frame=tlrb,numbers=none,breaklines=true]{Name}
function count(int[] ls) -> (int n)

define count_loopinv_6(int i) is:
  i >= 0
      
assert "loop invariant does not hold on entry":
    forall(int i$0):
      if:
        i$0 == 0
      then:
        count_loopinv_6(i$0)

assert "loop invariant not restored":
  forall(int i$1, int i, int i$2):
    if:
      i == 0
      count_loopinv_6(i$1)
      i$1 > 0
      i$2 == (i$1 + 1)
    then:
      count_loopinv_6(i$2)
\end{lstlisting}
\end{minipage}
\caption{Example of simple Whiley snippet as WyAL}
\label{lst:design-wyal}
\end{figure}

\subsection{Whiley Abstract Syntax Tree}
\comment{
	Whiley files \\
	* Closest to programmer code \\
	* most information in the most related form \\
}

The Whiley abstract syntax tree (AST) is the closest representation to
the source code.
This also includes the most information available, which includes detailed
types, variable names and expression trees \cite{whiley-origin}.
The AST is broken into three types of: top level declarations of types,
methods and functions ; statements blocks for individual statements
such as assignments or structures such as if statements; and, expressions
that include the usage of variables, function calls and constants.
There is some overlap between statements and expressions in the AST,
such as function calls are both statements and expressions given that they
return a value.
This gives the invariant generator to be able to identify loop patterns in the
structure.

However, the amount of information comes as at a cost of being able to parse
it and discover relevant information.
The structure of internal API of the Whiley AST forces a recursive decent
approach, much like the other internal languages in the compiler.
However, the AST is the richest in terms of number of types of nodes
in the AST needed to be supported to parse properly with
statement trees, expression trees and top-level declarations of methods and
types.
The cost of supporting all of these different nodes in the AST is
repeated code handling traversal and needing to handle a large range
of cases to parse a statement.
These costs are acceptable since it allows for generators to be
the most expressive and enable other design constraints of
debugging and inspection.

\section{Architecture of Loop Invariant Generators}
\comment{
	ALL ONE BLOB! vs strategy + a common set of information \\
	Explain how each generator is distinct code \\
	UML of strategy pattern \\
	the utility package
}

Another critical design decision was the architecture of how loop invariant
generators operate on the AST.
The architecture needs to fit with other similar static analysis checks
implemented, such as the definite assignment and type system, which all
traverse the AST once.
For this solution two architectural options are available to solve this issue:

\begin{itemize}
	\item{Monolithic, traversing the AST once and apply all generators at once,}
	\item{Multiple sweeps, traversing the AST once but repeat loops multiple times for each generator}
\end{itemize}

The architecture chosen for the resulting solution is  traversing the AST once
but repeating on loops so that each generator could inspect the loop.

\subsection{Monolithic Single Sweep}
\comment{
	What would a monolithic look like \\
	single pass \\
	how would it impact metrics such as maintainability \\
	how would it impact performance \\
}

A monolithic single sweep of the AST will only visit each
node in the AST once to generate the loop invariants.
This requires all loop invariant generators to be operating
during the pass or have all of their required information collected
in the single pass.
This minimizes the operational cost of traversing the loop body
multiple times at the cost of maintaining a monolithic loop invariant generator.
However, with the interweaving of multiple generators the maintenance of the
monolithic structure becomes issue prone.
This option was not implemented as the maintenance cost out weighed any
runtime performance benefits gained.


\todo{include UML of what the pipeline would look like with this arch}

\subsection{Multiple Sweeps with the Strategy Pattern}
\comment{
	What would a monolithic look like \\
	* multi-pass, no entire file multiple-times but same section \\
	* how would it impact metrics such as maintainability \\
	* how would it impact performance
}

Multiple sweeps of loops in the AST with a different strategy for each sweep
will only visit most nodes once but loop bodies.
This gives loop generators complete control over how they traverse the loop
body's AST at the cost of traversing the loop multiple times.
The additional benefit of each generator being separate is the maintenance cost
of adding a new generator or modifying an existing generator is greatly reduced
relative to a monolithic approach as each generator is not intertwined with
each other.
This option was selected as the cost of maintenance was significantly less
than a monolithic solution.
Another factor is that the architecture could still progress towards
a partial monolithic-strategy based architecture if the performance hit
of traversing loops becomes a bottleneck.
This can be achieved by finding patterns of what information generators are
seeking and provide an interface where that information is computed once for
all generators that require it and keep the traversal interface for unique generators.

\todo{include UML of what the pipeline would look like with this arch}

\todo{re-do start of this}
The first loop pattern is incrementing a variable
each iteration of the loop, most commonly used for indexes into an array to
iterate through it. See figure~\ref{lst:whiley-start} for a simple example.
The second loop pattern is making a copy of an array or creating an array with
the same length as another.
This is used generally when transforming every element in array into a separate
variable. See figure~\ref{lst:whiley-length} for a simple example.

\section{Implemented Loop Invariant Generators}

This section will cover how each loop invariant generator was implemented and their use cases.
Each generator only cover a subset of the cases described 
for their corresponding loop invariant in section~\ref{s:target-loop-inv}.

\subsection{Starting bound}

The implementation of the starting bound invariant covers only linear monotonic counters.
This generator utilises and composes multiple mechanisms to determine 
if a variable in the
loop is a counting variable with a known starting bound to apply the invariant for.
These mechanisms are: the loop context (see section \ref{s:loop-context}) ;
determining a simple mutation (see section \ref{s:simple-mut}) ; and,
determining if a mutation is increase or decreasing (see section \ref{s:sequence-dir}).
From this a variable is known to be increasing or decreasing, this is known as its direction.

The generator supports multiple mutations of the same variable in the loop.
To determine the final direction of the variable each mutation is merged though
using the lattice described in figure~\ref{eq:start-lattice}.
This shows that a variable could have four directions, unknown ($\top$),
increasing ($+$), decreasing ($-$) and indeterminable ($\bot$).
Merges occur on an assignment or when branches, 
such as \code{if}'s or nested loops, join back.
With the lattice a given variable's direction can be determined and also
decided if it is valid (increasing or decreasing) or invalid (unknown or indeterminable).

\begin{figure}
$$\text{values} = \{ \top, +, -, \bot \}$$
\[
\text{Lattice} \begin{cases*}
\quad X \cup \top = X \\
\quad X \cup X = X \\
\quad X \cup Y = \bot
\end{cases*}
\]
\caption{Starting bound lattice to determine mutation direction}
\label{eq:start-lattice}
\end{figure}

For variables that have a valid direction and a valid entry in the loop context an invariant
can be generated.
There are three components to the generated invariant: variable being mutated, an inequality based
on the variable's direction, and the known value of the variable before entering the loop.
The inequalities used in the expression are appropriate to the direction. 
If the variable was increasing then the variable will be greater than or equal to the known value
and vis versa for a decreasing variable.

To illustrate this generator the figures~\ref{lst:whiley-start} and \ref{lst:whiley-start-if}
show simplified examples.
Figure~\ref{lst:whiley-start} shows the most simplistic and common use case for this
generator. Where figure~\ref{lst:whiley-start-if} shows a more complex case 
with an \code{if} branch that changes the magnitude of the mutation but not the direction. 

\begin{figure}[ht]
\begin{lstlisting}
...
int i = 0

while i < |items|:
  // 'where i >= 0' is inferred
  apply(items[i])
  i = i + 1
...
\end{lstlisting}
\caption{Simple example of inferring starting bound of index}
\label{lst:whiley-start}
\end{figure}

\begin{figure}[ht]
\begin{lstlisting}
int i = x

while i < |items|:
  // 'where i <= x' is inferred
  apply(items[i])
  if skip(item[i]):
    i = i - 2
  else:
    i = i - 1
\end{lstlisting}
\caption{Starting bound invariant with branch}
\label{lst:whiley-start-if}
\end{figure}

\subsubsection{Limitations}

There are many limitations to the starting bound generator.
The main root of these limitations is only handling the case
of single variable linear monotonic functions. 
This prevents the invariant being generated for other
methods of counting or using complex logic that determines
the mutation each iteration.

One example is having multiple nested loops to count or index
the inner-element then combining them together to a total variable.
In this example each counter for the nested loops will have a generated
starting bound invariant but the total will not.
This is due to the mutation of the total failing the test of a simple
mutation where the only variable allowed is itself.

Another example of the limitations present is shown in figure~\ref{lst:whiley-start-limit}.
This shows that the starting bound invariant will not be generated because
the direction is indeterminable.
Even though on inspection it is clear that \code{i} will not decrement.
However, the components of the generator does not understand this
since they lack the ability to test branches for tautologies
to prune their traversals. 

\begin{figure}[ht]
\begin{lstlisting}
int i = 0

while i < |items|:
// 'where i >= 0' cannot be  inferred
apply(items[i])
  if false:
    i = i - 1
  i = i + 1
\end{lstlisting}
\caption{Example of a limitation of the Starting bound generator}
\label{lst:whiley-start-limit}
\end{figure}

\subsection{Equal array length}

\fixme{remove techniques to design}

The equal array length invariant generator is implementation handles copied and
generated arrays.
This generator unities the associated array mechanism to determine if the
length of two arrays are equal (see Section~\ref{s:copied-array}).
From this association the generator then discovers what arrays should
be asserted equal in the loop invariant by checking if they are used in the
loop body.
This is to filter out known unessential invariants since they are known not to
be used.

The figure~\ref{lst:whiley-length} shows an example of the equal array length
invariant being generated.
Lines 1 and 3 show that the second array's association can be formed by either
copying or generating an array with by using the length of another array.
From either of these lines \code{copy} and \code{items} are linked until
either of these arrays are re-assigned or one falls out of scope.
Line 6 show that both of the associated arrays are used, one in an expression
and the other having an element assigned.
Both of these uses are examples of what the generator considers as usage in
the loop body.
Given these findings the generator then can generate the invariant that
the two associated arrays equal length as shown on line 5.

\begin{figure}
\begin{lstlisting}
    int[] copy = items
    // or
    float[] copy = [0.0;|items|]
    while i < |items|:
        // 'where |copy| == |items|' is inferred
        copy[i] = apply(items[i])
\end{lstlisting}
\caption{Simple example of inferring array lengths are equal}
\label{lst:whiley-length}
\end{figure}

\subsubsection{Limitations}

The limitations present in the implementation of the generator is
that other methods of creating arrays are not supported.
This includes the return of a function and an array constant.
In the case of a function the post-condition would need to
give a specification about the length of the array to create
such an invariant.
These are illustrated in figure~\ref{lst:whiley-length-limit}.

\begin{figure}
\begin{lstlisting}
function zeroes(int len) -> (int[] r)
requires len >= 0
ensures |r| == len:
    return [0;len]

function f():
    int i = 0
    int[] a = zeroes(5)
    int[] b = [1,2,3,4,5]
    while i < |a|
    // where |a| == |b| could be inferred
        // omitted
\end{lstlisting}
\caption{Limitations of equal array length invariant generator}
\label{lst:whiley-length-limit}
\end{figure}

\subsection{Upper bound}


\subsubsection{Limitations}

\subsection{Array iterative assignment}


\subsubsection{Limitations}
