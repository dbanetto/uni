\chapter{Design}\label{C:design}
\comment{
The aim here is to identify the key trade-offs in any design work you have undertaken.
When solving a complex problem, there are normally many different
approaches one can take — each with its own advantages and disadvantages.
It is expected that students will have initially considered a range of different solutions,
and will then have narrowed these down. The reasons why a particular approach was
discounted should be documented here.


Remember:
Appropriate design notation (e.g. UML diagrams) can be very helpful in conveying different aspects of a design.
It is vital that your design not be carried out in a vacuum.
Your design should be motivated very clearly by your goals and specifications.
Make sure that it is clear why you took the decisions that you did.
Do not give the impression that you settled on a design because it “felt right”
or that you tinkered around until you found something that worked.

Should cover: \\
 * techniques of loop invariant generation, \\
 * problem analysis \\
 * express the requirements of the solution \\
}

% intro to problem, what is the design space for this issue
%

\fixme{Update to include techniques are here, architectures in impl}

This chapter will discuss the requirements and considerations taken to design the implementation
of this project.
These considerations are derived by the project requirements
and guided by improving or retaining the same level of
usability of the Whiley compiler for a Whiley developer and a maintainer of the
compiler fixing or updating the modified compiler.
It will also discuss the internal design decisions of
where to place the loop invariants generators in the
current architecture of the Whiley compiler.
This includes where the generated loop invariants are found and
how they are inserted into the program.
The final design decision discussed is how the multiple generators are managed.

\section{Requirements}\label{s:requirements}
\comment{
    should cover the functional and non-functional requirements
}

The requirements for this project are derived from the original project
outline (see Appendix~\ref{A:proj-outline}).
The base requirements of each loop invariant generator is to be grounded
in practicality of the invariant not the generality of its use cases.
This is to say that it is not worth expanding a loop invariant
generator to cover cases that are unlikely to actually occur for generality
sake.

The other major requirement is that the Whiley compiler will be improved
with the addition of loop invariant generators.
This implicitly includes maintaining the current level of quality or
improving the experience with this extension.

Some parts of the original project outline fell out of scope of this project.
This included the students of SWEN224 would be used to evaluate this solution.
The decision to drop this requirement was in favour of focusing on implementing
many loop invariant generator rather than making a split effort of implementing
less generators so that proper user testing can be completed.
It was also found to be impracticable as the course layout of SWEN224 saw the
Whiley section taught in the last half of the semester with loop invariant in
particular taught in the final weeks making properly timed user studies
impracticable.

\todo{expand on the problem}

\todo{make requirements eat the considerations section}

Below is a list of design considerations that were identified to keep to ensure
that these requirements are met.
Each of these considerations either focus the development of the loop
generators or to force quality control on how they interact with the rest of the
compiler.

\begin{itemize}
    \item{Backward compatibility with non-loop invariant generated code.}
    \item{Not to hamper the ability to inspect and debug generated code.}
    \item{Avoid impacting runtime with generating code.}
    \item{Ensure the user can control the feature.}
    \item{Not all of the generated loop invariants will be useful.}
\end{itemize}

\subsection{Backwards Compatibility}
\comment{
pre-existing code should not fail because of the generated invariants
code that compiles without the flag should still compile with the flag
restricts loop invariants to be precise  and strict to avoid "wrong"
}

To maintain the quality of compiler code that previously would verify without 
generated loop invariants should still verify with them.
This is important for any existing codebases in Whiley to have choice of
moving to use the generated loop invariants without needing to resolving new
errors.
If and when this occurs it is considered a bug and evidence that the implementation 
of a generated loop invariant is unsound and should be reviewed.

\subsection{Inspection and Debugging}
\comment{
Allow debugging / reviewing  of generated code
achieved by emitting compiler messages about what, where and why they are
generated
}

The ability to be able to inspect or debug the generated code is an important feature to
ensure quality of extension to Whiley.
This is in the form of compiler message describing what, where and why
the loop invariants were generated or by writing the modified program back
to disk as source code.
This allows a user to be able to inspect and review the generated loop invariants for
debugging of their code or in the process of debugging a generator.
The compromise made by this feature is that it requires coupling to other
subsystems to report back these changes.
For example to emit a compiler message appropriately it would need
to couple with the logging sub-system to enforce common log modes such as
\code{DEBUG} and \code{VERBOSE}.

\subsection{Impacting Runtime}\label{s:design-create-var}
\comment{
Whiley does not have the notion of ghost variables \\
* possible name collisions \\
* This is to be avoided \\

Does force the programmer to make ghost variables or make copies
that make sense if they were making the invariants themselves vs generated
}

To ensure the quality of generated code there should be no impact on the
runtime of the program.
The most problematic situation is attempting to refer to previous values of
a variable during or before the loop.
Whiley does not have a notion of ghost variables, variables only used for
specifications, this forces the use of additional local variables instead~\cite{whiley-origin}.
Figure~\ref{lst:vars} shows a common code example when the additional local
variables are used, in this case \code{items} variable is used to compare
against the previous value.
This forces a loop generator to either create their own ghost variables or
attempt to identify another local variable in the program that will act as the
same.
To generate the local variable the type of the copied variable is needed to be
known to mimic and a new statement directly before the loop is needed to be
added.
For example figure~\ref{lst:vars} would replace the \code{modified} variable
with \code{items} and the generated code will have an assignment just before
the while loop.
The issue of creating local variables is that it could add a runtime cost by
adding an additional copy of a potentially large data structure
and holding it in memory without being used.
The alternative to this is to restrict the loop invariant generators to check
for other variables that can work as pseudo-ghost variables.
This pattern is common in Whiley code as the lack of ghost variables require
the style of code in figure~\ref{lst:vars} to verify.
This resolves the issue of hidden local variables from the programmer
and gives them control over how they are handled.
The trade-off of not having a hidden runtime cost with restricted generators
is the most beneficial for the end-user as existing code will also benefit
and for more complex invariants the ghost variables are still required.

\begin{figure}[ht]
\begin{lstlisting}
function add_one(int[] items) -> int[]:
    i = 0
    modified = items
    //
    while i < |items|
        where all { k in 0..i | modified[i] == items[i] + 1 }
        where |modified| == |items|:
        modified = items[i] + 1
        i = i + 1
    //
    return modified
\end{lstlisting}
\caption{Example of Whiley program with \code{modified}.}
\label{lst:vars}
\end{figure}

\subsection{Usability}
\comment{
COMPILER FLAGS \\
How easy it is to control the functionality  \\
Error messages \\
No goal on time impact \\
Choice to not repeat compilation and verification
}

The command line interface and the WhileyWeb IDE are the only user interfaces
of the Whiley compiler.
The modification of the WhileyWeb IDE was out of scope of this project as it
would only be beneficial if user studies were conducted.
Thus the only user interface that the usability needs to be maintained or
improved is the command line interface.
This includes how users can control the loop invariants generators and what
compiler feedback is required.
This is handled by exposing control of the loop invariants generators via
a compiler flag.
This is inline with how other optional features are handled in Whiley and
other standard compilers such as the GNU C compiler.

\subsection{Essential Invariants}\label{s:essential-inv}
\comment{
It is known that all invariants generated will not be useful to proving the post-condition
(due to P-C has no input generated inv) \\

It is OK to have the minimal set of invariants
to test if it is essential requires repetitive verifications, infringes on
impact on compile time.
}

To provide some base quality of loop invariant generators that generated invariants 
can be redundant.
This means that it is acceptable that a generated invariant is not
checked that it is redundant for the verification of the loop or the function
as whole.
If the generated loop invariant were checked if they were redundant
they would need to determine what is essential either by greatly increasing
the complexity of the generators by inferring if there is another invariant
that is a stronger or weaker condition than itself.
Another method would be to attempting to find the
minimal set of invariants by testing combinations of invariants.
The trade-off of this decision is that at the cost of having some redundant
invariants that the complexity of code and time generating loop invariants is
greatly reduced.

\section{Targeted loop invariants}\label{s:target-loop-inv}

This project targeted four simple loop invariants to generate.
These invariants have been observed to be repetitive in Whiley code and 
can be inferred from the code before the loop and  in the loop body itself.
This section will describe each of these invariants and their conditions
with examples annotated to highlight the target invariant\footnote{There are additional examples of each invariant included in appendix~\ref{A:code-examples}}.
The targeted loop invariants are:

\begin{itemize}
	\item Starting bound invariant
	\item Equal array lengths
	\item Upper bound
	\item Array iterative assignment
\end{itemize}

\fixme{not sure if the loop pattern stuff actually works out now}
\fixme{would loop invariant patterns be more applicable?}

\todo{refer to appendix of code examples for additional code}

\subsection{Starting bound}\label{s:design-starting-bound}
\comment{
	finds entry value \\
	creates invariants from composing things \\
	has the loop pattern of \\
	int $i$ = $X$ \\
	while $C$: \\
	   $P_1$ \\
	   $i = i (+|-) N$ \\
	   $P_2$
}

The starting bound invariant is a common invariant used with counting
variables.
It is used to assert that the counting variable will increase or decrease
in terms of a starting value.
The most common use of this invariant is when the counting variable is 
used to index an array.
For example, given a variable \code{i} that is used to iterate 
through an array from start to end, 
\code{i} can be said to always be equal to or greater than \code{0}.
This would become \code{i >= 0} as a Whiley loop invariant.
Other uses include counting down attempts or counting the number of iterations
that meet a condition.

The resulting loop invariant is built up of three components.
The first component is the counting variable ;
the second component is the known start point of the counting variable before the loop ; 
and, the third is an inequality that is determined by if the counting variable will increase or decrease
from its starting point during the loop.
Figure~\ref{lst:starting-bound-gen} shows a generalised form of the invariant
in both the increasing (left) and decreasing (right) forms.

\begin{figure}[ht]
\noindent\begin{minipage}{.45\textwidth}
\begin{lstlisting}
int i = C
...
while P
  where i >= C:
  ... 
  i = i + D
  ...
\end{lstlisting}

\end{minipage}\hfill
\begin{minipage}{.45\textwidth}
\begin{lstlisting}
int i = C
...
while P
  where i <= C:
  ... 
  i = i - D
  ...
\end{lstlisting}	
\end{minipage}\hfill
\caption{The generalized form of the starting bound invariant}
\label{lst:starting-bound-gen}
\end{figure}

To identify the invariant each component of it needs to be identified in the program.
The starting point of the invariant is identified by observing a variable being
assigned to a known value before the loop.
From this observation the starting point of the variable is known.
In the general form in figure~\ref{lst:starting-bound-gen} shows that a variable \code{i} 
is assigned a value of \code{C} some time before the loop.
This is generally being set to 0 when the count increases or the length of
an array when the count decreases.
Figure~\ref{lst:starting-bound-double} and \ref{lst:starting-bound-contains} illustrates this
in the context of working Whiley functions.
In figure~\ref{lst:starting-bound-double} it can be observed
that \code{i} and \code{r} are both set to \code{0} before the loop with no modifications.
Figure~\ref{lst:starting-bound-contains} is similar except that 
the starting value for \code{i} is \code{|items|}
instead of a constant value. The value of \code{|items|} is known since the 
size of \code{items} is not mutated before or during the loop.

To identify if a variable is used as for counting the value has to be
mutated during the loop with relation it self.
This behaviour is generally observed with all counting since the 
new value builds upon its previous value.
In the general form in figure~\ref{lst:starting-bound-gen} this is
shown on line 6 where \code{i} is mutated by changing the value by plus or minus
\code{D}.
\code{D} in this case is the total change in one iteration.
This allows for \code{D} to be within conditional blocks or nested-loops.
The value of \code{D} must be determinable to either monotonically increase or decrease \code{i} each iteration.


If the mutation can be determined to be either increasing or decreasing 
the appropriate inequality can be used with the counting variable and starting value.
This can be seen in invariants of figures~\ref{lst:starting-bound-double} and \ref{lst:starting-bound-contains}.
In figure~\ref{lst:starting-bound-double} it can be observed that \code{i} and \code{r} are both
increasing and have starting value of \code{0}.
From this the invariants on lines 7 and 8 can be inferred.
In figure~\ref{lst:starting-bound-contains} shows that \code{i} has a starting value of
\code{|items|} and is decreasing with each iteration thus the invariant on line 6 can be inferred.


\begin{figure}[ht]
\begin{lstlisting}
function double(int n) -> (int r)
requires n >= 0
ensures n * 2 == r:
	int i = 0
	int r = 0
	while i < n
		where r >= 0  // starting bound
		where i >= 0: // starting bound
		r = r + 2
		i = 1 + 1
	return r
\end{lstlisting}
\caption{Example of a starting bound invariant for doubling a number.}
\label{lst:starting-bound-double}
\end{figure}


\begin{figure}[ht]
\begin{lstlisting}
function contains(int[] items, int item) -> (bool r)
ensures  r ==> some { i in 0..|items| | items[i] == item }
ensures !r ==> all  { i in 0..|items| | items[i] != item }:
  int i = |items|
  while i > 0
  where i <= |items| // starting bound
  where all { j in i..|items| | items[j] != item }:
    i = i - 1
    if items[i] == item:
      return true
  //
  return false
\end{lstlisting}
\caption{Example of a starting bound invariant for a contains function.}
\label{lst:starting-bound-contains}
\end{figure}

\subsection{Equal array length}
\comment{
	* finds associated arrays \\
	* checks to make sure source is not modified \\
	* asserts their length is the same \\
	
	pattern: \\
	
	$x = ( a | [X;|a|] )$ \\
	while C: \\
	$P$ - not includes $x = X$, $a = Y$\\
}

The equal array length invariant is a common invariant when accessing two
arrays with the same length.
This invariant correlates the length of two arrays by asserting they are equal.
The most common usage of this is when the values of one array are being mutated
and stored into another array.
Other common uses are when each element in an array is being applied
a function then stored in another array of equal length.

The resulting loop invariant has a single components it needs to identify.
That there is an association of length between arrays and it is maintained
throughout the loop.
From these components the loop invariant can be inferred.
Figure~\ref{lst:equal-array-length-gen} is a generalized form
of this invariant.


\begin{figure}[ht]
\noindent
\begin{minipage}{.45\textwidth}
\begin{lstlisting}
copy = items
...
while C
  where |copy| == |items|:
  ...
  items[i] = f(copy[i])
  // or
  copy[i] = f(item[i])
  ...
\end{lstlisting}
\end{minipage}\hfill
\begin{minipage}{.45\textwidth}
\begin{lstlisting}
copy = [0; |items|]
...
while C
  where |copy| == |items|:
  ...
  copy[i] = f(items[i])
  // or
  items[i] = f(copy[i])
  ...
\end{lstlisting}
\end{minipage}
\caption{Generic example of equal array lengths}
\label{lst:equal-array-length-gen}
\end{figure}

The association of length between the arrays can be identified by inspecting 
how the arrays are constructed.
Arrays could be directly associated with an array copy or generating
a new array with the length set to another array's length.
They can also be indirectly related by two arrays being generated
to have same length via the same value (e.g. variable or constant).
To ensure that this association is maintained the arrays must not 
be assigned a new value before or during the loop after the association
is formed.
Figure~\ref{lst:equal-array-length-gen} shows these two methods to
identify these associations, with copy on the left and generated 
arrays on the right.
Figure~\ref{lst:array-length-eq} shows an example of 
Whiley code using this invariant.
The \code{mod} and \code{items} arrays are associated on line
8 when \code{mod} is assigned a clone of \code{items}. 

\begin{figure}[ht]
\begin{lstlisting}
function add(int[] items, int delta) -> (int[] r)
ensures |r| == |items|
ensures all { i in 0..|r| | items[i] + delta == r[i] }:
    int i = 0
    int[] mod = items
    while i < |items|
    where |mod| == |items| // equal array length
    where i >= 0
    where all { j in 0..i | items[j] + delta == mod[j] }:
        mod[i] = items[i] + delta
        i = i + 1
    //
    return mod
\end{lstlisting}
\caption{Example of equal array length invariant.}
\label{lst:array-length-eq}
\end{figure}

\subsection{Upper bound}\label{s:design-upper-bound}
\comment{
	Finds simple mutations \\
	Figures out actually how much the value changes every iteration
	(is slight change on finding sequence direction, just stricter) \\
	checks to ensure that the variable is apart of the termination condition of the
	loop \\
	Allow mutations of +1 or -1 each iteration\\
	
	pattern: \\
	
	$i = N$ \\
	while $C$ \\
	$P_1$ \\
	$i = i (+|-) (1|-1)$ \\
	$P_2$ \\
}

The upper bound invariant limits the range of a counting variable.
This is common when the counter has a known range it has to be within,
such as indexing an array.
The common use case is when finding the index of an element,
this requires the index to be bound between the start and end of the
array with the inclusion of the error result of -1 or the length of the array.
Much like how the starting bound invariant bounds
the counting variable based on knowledge before the loop,
the upper bound invariant gives a bound based on during the loop.

The resulting loop invariant is built of three components.
The first is identifying the counting variable; 
the second is identifying the upper bound by inspecting the
loop conditions; and, the third is determining the 
magnitude of which the counting variable changes each iteration.
Figure~\ref{lst:upper-bound-gen} shows a generalised form
of this invariant with its two variants of a positive 
upper bound (left) and a negative upper bound (right).

\begin{figure}[ht]
\noindent
\begin{minipage}{.45\textwidth}
\begin{lstlisting}
int i
...
while i > C && P
  where i - D > C:
  ...
  i = i + D
  ...
\end{lstlisting}
\end{minipage}\hfill
\begin{minipage}{.45\textwidth}
\begin{lstlisting}
int i
...
while i < C && P
  where i + D < C:
  ...
  i = i - D
  ...
\end{lstlisting}
\end{minipage}
\caption{Generic example of upper bound}
\label{lst:upper-bound-gen}
\end{figure}

The method to identify the counting variable component
of the invariant is the same as described in the 
starting bound invariant (see Section~\ref{s:design-starting-bound}).
This is shown in the generic example of figure~\ref{lst:upper-bound-gen}
that a variable \code{i} exists.

To find the upper bound the loop condition is analyzed.
This is to identify a loop condition that includes an
inequality that includes the counting variable.
To achieve this the loop condition can be split
by conjunctive statements so each can be checked on their own.
This is shown in the generic example in figure~\ref{lst:upper-bound-gen}
by the while condition is a conjunctive between and condition code{P}
and an inequality involving the counting variable \code{i}.
From this it can be inferred that the upper bound of the counting 
variable is \code{C}.
Figures~\ref{lst:upper-bound} and \ref{lst:upper-bound-indexof}
both provide full examples of the loop condition being mutated
into the upper bound invariant.

The final component of the invariant determines the range of 
which \code{i} will exceed the upper bound on termination of the
loop.
This is similar to the starting bound invariant's method to 
determine the direction of a mutation (see
Section~\ref{s:design-starting-bound}).
However, in this case instead of tracking if the change is positive or
negative the value of the mutation is tracked.
This allows sequential mutations to concatenate and when merging
branches differing values would cause it to be indeterminable.

This invariant can be simplified in particular cases.
The conditions for this are that the magnitude
of the change of the count variable is either positive or negative one
and the inequality in the loop condition is strict.
With these conditions the invariant can be simplified to converting the
strict inequalities to non-strict inequalities.
This is due to counting variables always being integer values.
This is shown in figures~\ref{lst:upper-bound} and
\ref{lst:upper-bound-indexof} where both upper bound
invariant as in this simplified form.

\begin{figure}[ht]
\begin{lstlisting}
function len(int[] items) -> (int r)
ensures r == |items|:
  int i = 0
  while i < |items|
  where i <= |items| // upper bound
  where i >= 0:
    i = i + 1
  return i
\end{lstlisting}
\caption{Example of an upper bound invariant for the length of a list.}
\label{lst:upper-bound}
\end{figure}

\begin{figure}[ht]
\begin{lstlisting}
function indexOf(int[] items, int item) -> (int r)
ensures r == |items| ==> all { i in 0..|items| | items[i] != item }
ensures r >= 0  && r < |items| ==> items[r] == item
ensures r >= 0 && r <= |items|:
  //
  int i = 0
  while i < |items|
  where i >= 0
  where i <= |items| // upper bound
  where all { k in 0 .. i | items[k] != item }:
    //    
    if items[i] == item:
      return i
    i = i + 1
  //
  return i
\end{lstlisting}
\caption{Example of an upper bound invariant for indexOf.}
\label{lst:upper-bound-indexof}
\end{figure}

\subsection{Array iterative assignment}
\comment{
	Finds arrays that are being set \\
	Ensures the assignments are not self-dependent \\
	Identifies the index variable and is simple mutations \\
	Ensure index start of mutation on array, like starting bound \\
	figures out sequence direction of index, ascending or descending the array \\
	use index as the upper or lower bound \\
	determines range of the mutation \\
	
	pattern: \\
	
	$i = N$ \\
	$a = [ .. ]$ \\
	while $C$: \\
	$a[i] = f(i)$ \\
	$i = i (+|-) (1|-1)$ \\
	
	creates $all { \_i in C .. i | a[\_i] == f(\_i) }$
}

%\begin{itemize}
%	\item Find an array that has an element assigned every iteration
%	\item Identify the index variable
%	\item Finds the range of assignment, starting bound + current position
%	\item Combines range with array element assignment with variables replaced
%\end{itemize}

The array iterative assignment invariant describes how each element
of an array is assigned.
The common use case of this invariant is when a new array is having
its elements assigned, either by applying a function over an element
of another array or based on the index variable.
This is similar to the \code{map} function common in functional languages.
Figure~\ref{lst:array-iter-gen} is a generalized example of the invariant 
and shows that it is unique, relative to the other targeted invariants, by
using a universal quantifier.

This loop invariant is built up of two components.
The first component is the lower and upper bounds of the
universal quantifier, one of the bounds will be the index
variable and the other is the starting value of the index.
The second component is the assignment expression used to
determine the value of the element.
Figure~\ref{lst:array-iter-gen} is a generalized example of the invariant 
and shows the two possible variants of it.
The left side shows an ascending assignment, going from starting value to the index, and
the right shows the descending assignment, index to the starting value.


\begin{figure}[ht]
\noindent
\begin{minipage}{.45\textwidth}
\begin{lstlisting}
int i = 0
...
while P
  where all { j in 0..i |
      items[j] = f(j) }:
  ...
  items[i] = f(i)
  i = i + 1
  ...
\end{lstlisting}
\end{minipage}\hfill
\begin{minipage}{.45\textwidth}
\begin{lstlisting}
int i = |items|
...
while P
  where all { j in i..|items| |
      items[j] = f(j) }:
  ...
  items[i] = f(i)
  i = i - 1
  ...
\end{lstlisting}
\end{minipage}
\caption{Generic example of array iterative assignment invariant}
\label{lst:array-iter-gen}
\end{figure}

The bounds of the quantifier is discovered by identifying the
starting bound of an index variable.
This is achieved in the same way as described for the starting bound
invariant (see Section~\ref{s:design-starting-bound}).
To ensure that every element is assigned during the loop, the magnitude of
the change to the index variable is also checked to be either one or negative.
This is enforced by using the same method in the upper bound invariant to
determine the magnitude of the change (see Section~\ref{s:design-upper-bound}).
The ordering of the index variable and its starting value is determined
by the which direction it changes during the loop.
For example, if the index is known to be increasing the bounds of the 
quantifier will be starting value to index variable.
This is demonstrated in figure~\ref{lst:array-iter-assign} where
the index variable, \code{i}, is increasing each iteration and
the starting value of quantifier is 0 with the upper bound of \code{i}.

The assignment expression for each element is simply selecting the
assignment to an element of the array.
The assignment expression cannot have any dependencies on the previous
value of the array being assigned or to variables only available in the loop body.
This is due to limitation of being able to refer to previous values of
variables or element and variables within the loop.
If these are not present the assignment expression can be
safely used.
This is shown in figure~\ref{lst:array-iter-assign} with the 
assignment on line 11.

From these components they are merged to form the invariant.
The identified bounds are used as is while the assignment expression
requires some modification.
To create the quantifier a name for the quantifier's index is required.
This can be any name so long that it does not collide with any existing variable.
The assignment expression is modified from being an assignment to a equivalence
expression and that all uses of the index variable is replaced with the
quantifier's index variable.
From this the invariant is formed.
Figure~\ref{lst:array-iter-assign} shows a full example of this invariant.

\begin{figure}[ht]
\begin{lstlisting}
function add(int[] items, int delta) -> (int[] r)
ensures |r| == |items|
ensures all { i in 0..|r| | items[i] + delta == r[i] }:
  int i = 0
  int[] mod = items
  while i < |items|
  where i >= 0
  where |items| == |mod|
  // iterative assignment
  where all { j in 0..i | mod[j] == items[j] + delta }:
    mod[i] = items[i] + delta
    i = i + 1
  return mod
\end{lstlisting}
\caption{Example of array iterative assignment invariant.}
\label{lst:array-iter-assign}
\end{figure}
