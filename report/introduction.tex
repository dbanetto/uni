\chapter{Introduction}\label{C:intro}
\comment{
The purpose here is to introduce the problem being solved,  to motivate \\
why it is a problem one should care about, and to outline the solution developed \\
during the project. \\


Remember:  the introduction is the first part of the report an examiner will read. \\

If he/she finishes reading it without a proper understanding of the \\
problem being solved or what has been done, then they will almost certainly  \\
struggle with understanding the remainder.  You should attempt to make the project goals \\
and associated specifications as clear and as quantifiable as possible. These goals and \\
specifications should inform everything else that follows, so it is important to establish \\
them in the examiners mind. \\

Should cover \\
* motivation of problem \\
* context of the problem \\
* explain why it is interesting \\
* hint at how the this project solves it \\
* overview   of how it unfolds
}

% outline
\todo{Get a better opening line}
\todo{replace intro paragraph, focus on the teaching aspect of this project}


A common burden of writing formally verified code is providing loop invariants.
This is exacerbated by simple and repetitive loop invariants are present throughout the program.
Whiley shared this problem and this project is to alleviate some of this burden by
generating some simple loop invariants.

\section{Motivation}

% Why
The primary motivation behind this project is to improve the experience writing
verified programs in Whiley.
Currently the process of writing loop invariants can be tedious with 
repetition of common invariants.
To relieve some of these burden this project is to generate loop invariants.
Though it is known that generating loop invariants generally is uncomputable,
the focus is to generate practical loop invariants. \todo{There should be a cite for this}
The most practical loop invariants to generate are loop invariants that
are simple in nature and common throughout verified code.
For example, the index variable is within the range of the array being
iterated through~\cite{loop-patterns}.

The secondary motivation for this project is to enable more programmers and
students to utilise formally verified languages.
The hypothesis of David J. Pearce, \textit{Verification is challenging when it requires a creative step},
describes his observations with the difficultly of student's with loop invariants~\cite{spec-usability}.
The hypothesis covers the disconnect between loop invariants can have from the
post-condition with invariants about implementation details, such as an index
of an array.
In practise from tutoring students in Whiley
it has been observed that another issue is from students being able to identify trivial
properties and assumes the verifier can too.
This project seeks to relieve some of these problems meeting some of these
expectations by generating some loop invariants.
An example of a common misconception is that the verifier knows that a copied
array has the same size as its source.

\section{Solution}

% brief overview of solution
This project provides to this problem is an extension to
the Whiley Compiler to generate simple loop invariants.
This includes the identification of simple loop invariants and being able
to generate a loop invariant from the source code of a Whiley program.
Since the Whiley Compiler is an open source project this project will be
hosted in a fork of compiler.

% 'simple'
The definition of a simple loop invariants for this project is an invariant that is
obvious at inspection that an invariant both exists and would hold.
An example would be that while iterating through an array the index
will not go out of bounds of the array, such as figure~\ref{lst:whiley-ex-1}.
The purpose of having a definition of an simple invariant is to prevent
possible confusion when what looks to be unproven code is verified and to point
the scope of the project away from being too general, such as supporting
a rare scenarios such as indexing an array with a polynomial formula.
This leaves the more interesting invariants for the user to define.

\begin{figure}[ht]
\begin{lstlisting}
function indexOf (int[] items, int item) -> (int r)
ensures r == |items| || items[r] == item:

    int i = 0

    while i < |items|
      where i >= 0 && i <= |items|:

        if items[i] == item:
            break
        i = i + 1

    return i
\end{lstlisting}
\caption{Verifiable Whiley code to find index of an item in an array}
\label{lst:whiley-ex-1}
\end{figure}
%
%% loop patterns
%These simple loop invariants are a product of common design patterns in loops.
%For this report the idea of loop patterns are introduced and are akin to
%software architecture design patterns on a more mechanical level.
%From studying a known loop pattern properties can be derived from it.
%Such as figure~\ref{lst:whiley-ex-1} that has a common loop pattern of
%iterating an array via an index variable.
%From this pattern you know that that during the loop that the index will be
%within bounds of the array.
%This can be exploited by statically detecting the loop patterns and assisting
%the Whiley Compiler by generating loop invariants.

\section{Overview}

This report is broken up into 7 chapters with each chapter detailing a separate part of the project.
Chapter~\ref{C:background} explains the background research, concepts and related work to this report.
Chapter~\ref{C:design} discusses the design limitation, goals and choices that went into designing the
solution for this project.
Chapter~\ref{C:impl} provides the details of the mechanisms behind the implemented 
loop invariant generators including examples of application and their limitations.
Chapter~\ref{C:eval} details of the experiment carried out with this project to determine if it is
successful, including the tool and data sets used.
Chapter~\ref{C:con} concludes the report with the conclusions drawn from the project,
related work and future work.  