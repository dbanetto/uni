\chapter{Design}\label{C:design}
% The aim here is to identify the key trade-offs in any design work you have undertaken.
% When solving a complex problem, there are normally many different
% approaches one can take — each with its own advantages and disadvantages.
% It is expected that students will have initially considered a range of different solutions,
% and will then have narrowed these down. The reasons why a particular approach was
% discounted should be documented here.
%
% Remember:
% Appropriate design notation (e.g. UML diagrams) can be very helpful in conveying different aspects of a design.
% It is vital that your design not be carried out in a vacuum.
% Your design should be motivated very clearly by your goals and specifications.
% Make sure that it is clear why you took the decisions that you did.
% Do not give the impression that you settled on a design because it “felt right”
% or that you tinkered around until you found something that worked. 

% intro to problem, what is the design space for this issue
%

This chapter will discuss the consideration taken to design the implementation
of this project.
These considerations are guided by improving or retaining the same level of 
usability of the Whiley Compiler for a Whiley developer and a maintainer of the
compiler fixing or updating the modified compiler. 
It will also discuss the internal design decisions of
where to place the loop invariants generators in the
current architecture of the Whiley Compiler.
This includes where the generated loop invariants are found and
how they are inserted into the program.
The final design decision discussed is how the multiple generators are managed.


\section{Considerations}
% These are designs for the project and the changes and impacts on the Whiley
% Language and compiler

Throughout the design of the overall architecture and each individual loop invariant
generator some design considerations were kept in mind.
\begin{itemize}
    \item{Backward creditability with non-loop invariant generated code.}
    \item{Not to hamper the ability to inspect and debug generated code.}
    \item{Avoid creating new local variables when generating code.}
    \item{Ensure the user can control the feature.}
    \item{Not all of the generated loop invariants will be useful.}
    \item{The final artifact needs to be maintainable.}
\end{itemize}

\subsection{Backwards Compatibility}
% pre-existing code should not fail because of the generated invariants
% code that compiles without the flag should still compile with the flag
% restricts loop invariants to be precise  and strict to avoid "wrong"

A design consideration for this project was that existing code that compiled
and verify without the loop invariant generator should still compile and verify
with it.
This consideration limits the generated loop invariants from creating
invariants that would not be supported by the program during verification.
Figure~\ref{lst:backwards-compat} shows an example of a potential loop
invariant generator that would cause previously working code to fail.
%% TODO: explain the example
When this occurs it is considered a bug and evidence that the implementation or
the design of the generated loop invariant is unsound and should be reviewed.

\begin{figure}[ht]
\begin{lstlisting}
// TODO: Make an example of where a generated invariant would make the code
// fail
\end{lstlisting}
    \label{lst:backwards-compat}
    \caption{Example of a Whiley program with generated loop invariants that will
    make it fail to verify.}
\end{figure}

However, this consideration is not reflexive.
This implies that it does not ensure that a program that verifies with
generated loop invariants must still verify without them.
This design issue is considered as it is important that a user could
switch to using the generated loop invariants without suddenly having
their program failing to verify.


\subsection{Inspection and Debugging}
% Allow debugging / reviewing  of generated code
% achieved by emitting compiler messages about what, where and why they are
% generated

The ability to be able to inspect the generated code is an important feature to
design into the solution.
This could come in the form of compiler message describing what, where and why
the loop invariants were generated or by writing the modified program back
to disk as source code.
This allows a user to be able to inspect and review the generated loop invariants for
debugging of their code or in the process of debugging a generator.
The compromise made by this feature is that it requires coupling to other
subsystems to report back these changes.
For example to emit a compiler message appropriately it would need
to couple with the logging sub-system to enforce common log modes such as
\code{DEBUG} and \code{VERBOSE}.

\subsection{Creating Variables}
% Whiley does not have the notion of ghost variables
%  * possible name collisions
%  * This is to be avoided
%
%  Does force the programmer to make ghost variables or make copies
%  that make sense if they were making the invariants themselves vs generated

Whiley does not have a notion of ghost variables that are only used for
specifications \cite{whiley-origin}.
This forces a loop generator to either create their own ghost variables or
attempt to identify another local variable in the program that will act as the
same.
Generating new local variables 

\subsection{Usability}
% COMPILER FLAGS
% How easy it is to control the functionality 
% Error messages
% No goal on time impact
% choice to not repeat compilation and verification

\subsection{Essential Invariants}
% It is known that all invariants genereated will not be useful to proving the post-condition
% (due to P-C has no input generated inv)
% It is OK to have the minimal set of invariants
% to test if it is essential requires repetitive verifications, infringes on
% impact on compile time

During the design of this solution it is known that not all generated
invariants will be essential invariants.
This implies that not all of the generated loop invariants will be useful
in proving the post-condition of the loop's function.
If only the essential invariants where generated the solution would need
to determine what is essential either by increasing the complexity of the
generators by inferring the invariants from the post-condition or by 
attempting to minimize the generated loop invariants by testing combinations 
of invariants.
The trade-off of this decision is that at the cost of having some non-essential
invariants that the complexity of code and time generating loop invariants is
greatly reduced.

\subsection{Maintainability}
% Whiley is an on-going project
% how could loop inv generated best be handled over time
% two cases: adding a new one, changing an older one

% Biggest design decision  was at what level are the invariants going to be inferred
% then where are they going to be generated

\section{Approaches to Loop Invariant Generation}

The principle design decision of this project was at
what point in the Whiley Compiler pipeline
to generate the loop invariants.
There are three candidate stages in the Whiley Compiler
that the input program will go through in the process of
being verified, the list is not necessarily in order of occurrence:

\begin{itemize}
    \item{Whiley Intermediate Language}
    \item{Whiley Assertion Language}
    \item{Whiley's Abstract Syntax Tree}
\end{itemize}

Each of these stages have different representations of the Whiley program and
associated information about the program itself.

\subsection{Whiley Intermediate Language (WyIL)}
% possible method to create tool on WyIL
% What could be achieve with this
%  * is designed to be a binary version of the code
%  * WyIL goal is to make intermediate language between ALL targets of Whiley
%  * AST converted into loosely connected values via integers
%  * less information readily available 

\cite{wyil}

\subsection{Whiley Assertion Language (WyAL)}
%% WyAL
% what it is
% * changes would be more dramatic
% * Does not have the same level of control
% * hosted inside the theorem prover

\subsection{Whiley Abstract Syntax Tree}
% Whiley files
% * Closest to programmer code
% * most information in the most related form
% * 

\section{Architecture of Loop Invariant Generators}
% ALL ONE BLOB! vs strategy + a common set of information
% Explain how each generator is distinct code
% UML of strategy pattern
% the utility package

\subsection{Monolithic}
% What would a monolithic look like
% how would it impact metrics such as maintainability
% how would it impact performance

\subsection{Strategy Pattern}
% What would a monolithic look like
% how would it impact metrics such as maintainability
% how would it impact performance
