\documentclass[11pt, a4paper, twoside, openright]{report}

\usepackage{float} % lets you have non-floating floats

\usepackage{listings}
\usepackage{color}
\usepackage{url} % for typesetting urls
\usepackage[parfill]{parskip}

%  We don't want figures to float so we define
%
\newfloat{fig}{thp}{lof}[chapter]
\floatname{fig}{Figure}

\definecolor{grey}{rgb}{0.95,0.95,0.95}

\lstset{ %
    backgroundcolor=\color{grey},
    frame=single,
    numbers=left
}

%% These are standard LaTeX definitions for the document
%%
\title{Generating simple loop invariants for Whiley}
\author{David Barnett}

%% This file can be used for creating a wide range of reports
%%  across various Schools
%%
%% Set up some things, mostly for the front page, for your specific document
%
% Current options are:
% [ecs|msor|sms]          Which school you are in.
%                         (msor option retained for reproducing old data)
% [bschonscomp|mcompsci]  Which degree you are doing
%                          You can also specify any other degree by name
%                          (see below)
% [font|image]            Use a font or an image for the VUW logo
%                          The font option will only work on ECS systems
%
\usepackage[font,ecs]{vuwproject} 

\supervisors{Lindsay Groves}
\otherdegree{Bachelor of Engineering with Honours in Software Engineering}

% Comment this out if you want the date printed.
\date{}
\newcommand{\code}[1]{\texttt{#1}}

\begin{document}

% Make the page numbering roman, until after the contents, etc.
\frontmatter

%%%%%%%%%%%%%%%%%%%%%%%%%%%%%%%%%%%%%%%%%%%%%%%%%%%%%%%

\begin{abstract}
    % TODO: abstract
\end{abstract}

%%%%%%%%%%%%%%%%%%%%%%%%%%%%%%%%%%%%%%%%%%%%%%%%%%%%%%%

\maketitle

%%%%%%%%%%%%%%%%%%%%%%%%%%%%%%%%%%%%%%%%%%%%%%%%%%%%%%%

\mainmatter

%%%%%%%%%%%%%%%%%%%%%%%%%%%%%%%%%%%%%%%%%%%%%%%%%%%%%%%

\section*{1 Introduction}
% This should briefly outline the project and if necessary 
% reevaluate the original plan in light of what has been learned in the interim.  In
% particular, any significant deviations in the problem being addressed, or the solution
% being developed should be clearly highlighted and justified.

\section*{2. Background}
% This should discuss any existing solutions to the given problem,
% and may reference academic papers,  books and other sources as appropriate.   Care
% should be taken to identify key differences between these solutions,  and that being
% developed in the project.

\section*{3. Work Done}
% This should discuss what progress has been made on designing, implementing
%  and evaluating the artifact. Care must be taken to ensure that any discussion
% of technical points are clearly explained, with diagrams being used where appropriate.
% In many cases, the evaluation proper will not yet have begun. However, it is important
% to demonstrate that sufficient thought has been given to the evaluation.

\section*{4. Future Plan}
% This should highlight the main components which remain to be done,
% and provide a proposed time-line in which this will happen. In putting together a time
% line, students must take into account upcoming examinations, coursework deadlines
% and other disruptions.

\section*{5. Request for Feedback}
% This  should  highlight  any  difficulties  currently  faced,  and
% make specific requests for guidance from the examination committee.  For example,
% a  student  may  be  unsure  how  best  to  evaluate  their  artifact,  and  would  appreciate
% suggestions for alternative methods.

%%%%%%%%%%%%%%%%%%%%%%%%%%%%%%%%%%%%%%%%%%%%%%%%%%%%%%%
\backmatter
%%%%%%%%%%%%%%%%%%%%%%%%%%%%%%%%%%%%%%%%%%%%%%%%%%%%%%%

%\bibliographystyle{ieeetr}
\bibliographystyle{acm}
\bibliography{report}
\end{document}

