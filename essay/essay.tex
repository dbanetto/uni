\documentclass{CRPITStyle} 
\usepackage[authoryear]{natbib}
\renewcommand{\cite}{\citep}
\pagestyle{empty}
\thispagestyle{empty}
\hyphenation{roddick}

\begin{document}

\title{When to use NoSQL}
\author{David Barnett}
\affiliation{School of Engineering and Computer Science \\
Victoria University of Wellington, \\
PO Box 600, Wellington, 6140 \\
Email:~{\tt barentdavi@myvuw.ac.nz}}

\maketitle

\begin{abstract}
    TODO Abstract
\end{abstract}

\vspace{.1in}

\noindent{\em Keywords:} NoSQL, Use cases, Document Store, Column Base, Graph database

\section{Introduction}
% Cover the aim of the essay

\paragraph{} With growing use of NoSQL databases in production environments it has become more
important to understand when and why to use them over a traditional
relational database.
NoSQL can be broken down to many different types which differ on how
data is represented~\cite{type_nosql}.
For example the traditional relational databases are based upon relational
model and relational algebra~\cite{relational_db}. 

\paragraph{} This report will cover some common types of NoSQL databases.
These are: Document store; Column base; and, Graph databases.
For each of these types this report will give an overview on how each type 
operates and a description of its use case with a comparison with a 
traditional SQL database.

\cite{whats_new}
\cite{sql_nosql_gap}
\cite{scalable_sql}

\section{What is NoSQL}
% brief overview

\paragraph{} NoSQL is a class of databases that are not based on the
traditional relational data model~\cite{nosql_db}.
This has been motivated by some limitations of relational databases
when handling large amounts of data that is produced and consumed
quickly across a large geographic area~\cite{nosql_db}.

Where relational databases are based on a structured
data model that is strictly enforced~\cite{relational_db}

% compare with relational model

% give definition of a database? (managed store of data)
% relational is based on set theory and relational algebra

%% BASE
% What it is

% How does it compare to ACID

% What does this mean

%% CAP theory
% what is it

% how it relates to NoSQL

% trade-off between consistency, availability and partition tolerant

% === Document Store ===

\section{Document Store}
% How does it work
%  - key to a structured document
%  - document can be inspected by DBMS
% Relate it back to BASE


\subsection{Use case}
% When are people using this
%  - JSON store
%  - XML store
% Example of use case
%  - Searching

\subsection{When to use it}
% Why are people choosing this
% Compare to relational counterpart

% === Column base ===

\section{Column base}
% How does it work
% Relate it back to BASE

\subsection{Use case}
% When are people using this
% Example of use case

\subsection{When to use it}
% Why are people choosing this
% Compare to relational counterpart

% === graph ===

\section{Graph}
% How does it work
% Relate it back to BASE

\subsection{Use case}
% When are people using this
% Example of use case

\subsection{When to use it}
% Why are people choosing this
% Compare to relational counterpart

\section{Conclusion}


\bibliographystyle{agsm}
\bibliography{essay}

\end{document}
