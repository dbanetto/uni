\chapter{Implementation}\label{C:impl}
\comment{
The aim here is to explain the technical aspects of the project.
The challenge is to ensure the text is clear and understandable.
This is not easy, as ideas and concepts involved are often complex in nature.
Nevertheless, if an examiners cannot understand how the implementation works,
he/she cannot award marks for it.
If this happens, the student is fault for poor communication. \\

Remember: \\
nothing is so complicated that it cannot be clearly explained.
Classic pitfalls include: \\
    * long convoluted sentences, \\
    * use of long words, \\
    * too much time spent discussing irrelevant details, \\
    * poor organisation of sections, subsections and paragraphs, \\
    * and too few diagrams or examples.
}

This chapter will cover the implementation of the targeted loop invariants
and critical choices made during the implementation phase of the project.
The areas covered are: the architectural decisions; how the generators
were structured; key mechanisms that are common across the generators; 
and, how the generators work and their current limitations.

\todo{Looks best to move the next sections into impl chapter}

\comment{
	Biggest design decision  was at what level are the invariants going to be inferred
	then where are they going to be generated
}
\section{Approaches to Loop Invariant Generation}

A critical architecture decision of this project was to where 
to add the loop invariant generators into Whiley's compiler pipeline.
There are three candidate stages in the Whiley Compiler
that the input program will go through in the process of
being verified, the list is not necessarily in order of occurrence:

\begin{itemize}
	\item{Whiley Intermediate Language}
	\item{Whiley Assertion Language}
	\item{Whiley's Abstract Syntax Tree}
\end{itemize}

Each of these stages have different representations of the Whiley program and
associated information about the program itself.
In the resulting solution the design approach taken was to build upon
Whiley's abstract syntax tree.

\subsection{Whiley Intermediate Language (WyIL)}
\comment{
	possible method to create tool on WyIL
	
	What could be achieve with this:
	
	* is designed to be a binary version of the code
	
	* WyIL goal is to make intermediate language between ALL targets of Whiley
	
	* AST converted into loosely connected values via integers
	
	* less information readily available
}

The Whiley Intermediate Language (WyIL) is a binary representation of a Whiley program
that can be easily transformed into other targets such as languages or
instruction sets \cite{wyil}.
It is a register based language with no nested expressions
that also includes blocks to group together statements such as the body of a loop.
This allows the language to keep a majority of its structure with the
exception of local variables and intermediate values becoming indistinguishable
without analysis.

The structure of internal API to query and modify WyIL code impedes a
generator's ability to identify components of invariants.
This is from the internal structure of a statement in WyIL where each
statement is loosely defined with op codes and blocks.
By generating the loop invariants as additional byte code the ability to
inspect the result and debug them are hampered relative to operating on the
AST.  This is because the ability to decompile from WyIL to Whiley could
be impeded which prevents the user from being able to inspect the generated
code.
WyIL can also be decompiled to a representation similar to assembly, see
figure~\ref{lst:design-wyil} for an example of Whiley code compared to its
intermediate language equivalent.
This example shows the amount of information available about the variables
in the program and their usage in the loop.
This approach was not selected due to the disconnect between the loops
seen by in the programmer source code to their representation in WyIL which
would make it difficult to understand why an invariant was generated.


\begin{figure}
\noindent\begin{minipage}{.45\textwidth}
\begin{lstlisting}[caption={Whiley Code},frame=tlrb,numbers=none]{Name}
i = 0
while i < |items|
where i >= 0:
  i = i + 1
\end{lstlisting}
\end{minipage}\hfill
\begin{minipage}{.45\textwidth}
\begin{lstlisting}[caption={WyIL Code},frame=tlrb,numbers=none]{Name}
const %1 = 0
loop modifies %1
invariant :
const %2 = 0
ifge %1, %2
goto lab_0
fail
.lab_0
return
ifge %1, %0
goto lab_1
add %1 = %1, %0
.lab_1
return %1
\end{lstlisting}
\end{minipage}
\caption{Example of simple Whiley snippet as WyIL}
\label{lst:design-wyil}
\end{figure}

\subsection{Whiley Assertion Language (WyAL)}
\comment{
	WyAL, what it is:
	
	* changes would be more dramatic
	
	* Does not have the same level of control
	
	* hosted inside the theorem prover
}

The Whiley Assertion language (WyAL) is derived from WyIL to create
assertions for the Whiley automated theorem prover \cite{wyal}.
WyAL follows a similar syntax to regular Whiley except that the program
is a collection of boolean expressions with additional constructs such as
if branches and universal quantifiers.
No program logic or structure is held in WyAL but are kept in the source
WyIL program.

Figure~\ref{lst:design-wyal} shows an example of WyAL code generated from
the given Whiley code.
This figure shows the assertions for a simple loop invariant as well as the
assertion message which are directly displayed to the user on failure.
The main benefit of generating loop invariants at the WyAL level is the
impact of the new loop invariants are contained to only the verification
process and prevents a generator author from breaking a design consideration,
such as not creating variables, as the Whiley code is immutable by this point.
However, since the lack of program logic in WyAL requires the use of a WyIL program
as backing the same issues with WyIL are present with working on WyAL.
This includes the difficultly to correlate between the high-level structure
in a Whiley program to the resulting WyIL code.

\begin{figure}
\noindent\begin{minipage}{.45\textwidth}
\begin{lstlisting}[caption={Whiley Code},frame=tlrb,numbers=none]{Name}
i = 0
while i < |items|
where i >= 0:
  i = i + 1
\end{lstlisting}
\end{minipage}\hfill
\begin{minipage}{.45\textwidth}
\begin{lstlisting}[caption={WyAL Code},frame=tlrb,numbers=none,breaklines=true]{Name}
// some omitted
      
assert "loop invariant does not hold on entry":
    forall(int i$0):
      if:
        i$0 == 0
      then:
        count_loopinv_6(i$0)

// some omitted
\end{lstlisting}
\end{minipage}
\caption{Example of simple Whiley snippet as WyAL}
\label{lst:design-wyal}
\end{figure}

\subsection{Whiley Abstract Syntax Tree}
\comment{
	Whiley files \\
	* Closest to programmer code \\
	* most information in the most related form \\
}

The Whiley abstract syntax tree (AST) is the closest representation to
the source code.
This also includes the most information available, which includes detailed
types, variable names and expression trees \cite{whiley-origin}.
The AST is broken into three types of: top level declarations of types,
methods and functions ; statements blocks for individual statements
such as assignments or structures such as if statements; and, expressions
that include the usage of variables, function calls and constants.
There is some overlap between statements and expressions in the AST,
such as function calls are both statements and expressions given that they
return a value.
This gives the invariant generator to be able to identify loop patterns in the
structure.

However, the amount of information comes as at a cost of being able to parse
it and discover relevant information.
The structure of internal API of the Whiley AST forces a recursive decent
approach, much like the other internal languages in the compiler.
However, the AST is the richest in terms of number of types of nodes
in the AST needed to be supported to parse properly with
statement trees, expression trees and top-level declarations of methods and
types.
The cost of supporting all of these different nodes in the AST is
repeated code handling traversal and needing to handle a large range
of cases to parse a statement.
These costs are acceptable since it allows for generators to be
the most expressive and enable other design constraints of
debugging and inspection.

\section{Architecture of Loop Invariant Generators}
\comment{
	ALL ONE BLOB! vs strategy + a common set of information \\
	Explain how each generator is distinct code \\
	UML of strategy pattern \\
	the utility package
}

Another critical design decision was the architecture of how loop invariant
generators operate on the AST.
The architecture needs to fit with other similar static analysis checks
implemented, such as the definite assignment and type system, which all
traverse the AST once.
For this solution two architectural options are available to solve this issue:

\begin{itemize}
	\item{Monolithic, traversing the AST once and apply all generators at once,}
	\item{Multiple sweeps, traversing the AST once but repeat loops multiple times for each generator}
\end{itemize}

The architecture chosen for the resulting solution is  traversing the AST once
but repeating on loops so that each generator could inspect the loop.

\subsection{Monolithic Single Sweep}
\comment{
	What would a monolithic look like \\
	single pass \\
	how would it impact metrics such as maintainability \\
	how would it impact performance \\
}

A monolithic single sweep of the AST will only visit each
node in the AST once to generate the loop invariants.
This requires all loop invariant generators to be operating
during the pass or have all of their required information collected
in the single pass.
This minimizes the operational cost of traversing the loop body
multiple times at the cost of maintaining a monolithic loop invariant generator.
However, with the interweaving of multiple generators the maintenance of the
monolithic structure becomes issue prone.
This option was not implemented as the maintenance cost out weighed any
runtime performance benefits gained.


\todo{include UML of what the pipeline would look like with this arch}

\subsection{Multiple Sweeps with the Strategy Pattern}
\comment{
	What would a monolithic look like \\
	* multi-pass, no entire file multiple-times but same section \\
	* how would it impact metrics such as maintainability \\
	* how would it impact performance
}

Multiple sweeps of loops in the AST with a different strategy for each sweep
will only visit most nodes once but loop bodies.
This gives loop generators complete control over how they traverse the loop
body's AST at the cost of traversing the loop multiple times.
The additional benefit of each generator being separate is the maintenance cost
of adding a new generator or modifying an existing generator is greatly reduced
relative to a monolithic approach as each generator is not intertwined with
each other.
This option was selected as the cost of maintenance was significantly less
than a monolithic solution.
Another factor is that the architecture could still progress towards
a partial monolithic-strategy based architecture if the performance hit
of traversing loops becomes a bottleneck.
This can be achieved by finding patterns of what information generators are
seeking and provide an interface where that information is computed once for
all generators that require it and keep the traversal interface for unique generators.

\section{Techniques used to infer properties of a loop}
\comment{
	Cover the individual techniques that were used in each generator \\
	- discovering loop context \\
	- determining if a variable has a simple mutation \\
	- determining if a sequence is increasing or decreasing  \\
	- decomposing loop conditions \\
	- identify usage of variables in the loop \\
	- associating a copied array back to its source \\
	- checking for self dependencies \\
}

Each loop invariant is the composition multiple techniques to generate their
invariants. This section will provide discuss the techniques, what they
accomplish and where their limitations lay.
The techniques covered are:

\begin{itemize}
	\item{Loop Context}
	\item{Determining a simple mutation}
	% \item{Decomposing Loop conditions and Invariants}
	\item{Associate copied arrays to source}
\end{itemize}

\subsection{Loop Context}\label{s:loop-context}
\comment{
	Write about the pre-loop stuff \\
	* how it collects data \\
	* what data it collects \\
	* why it collects \\
	* how it decides what to collect \\
	* how it differ between variables and parameters
}

\fixme{Break up and expand on this}

The loop context is a summary about the prelude of the loop
that generators will operate on.
The main focus of the loop context is to provide details about
local variables and parameters before the loop, such as their
values or types.
This is used by every loop invariant generator to identify local variables
that could be used for the invariant.

The loop context collects what it can determine about a variable.
When analysing a simple block of code it updates the variable last known to
the last known value. This is achieved by capturing the expression
assigned to the variable.
At a branching points, such as \code{if}'s or loops, the context of each 
block is collected independently then merged back to the parent at the end of the block.
If the block was a conditional branch the intersection of all paths are taken
before merging with the parent.
During the intersection if a variable is modified different in both branches it
is then considered indeterminable as it is unknown which branch will be taken.
Just as local variables fall out of scope in Whiley, they also do for loop contexts.

The loop context checks assignments by deciding if the value is determinable.
This is achieved by analysing the variable's assignment expression.
Currently the expression is considered determinable if the expression contains constant
values or other determinable variables used with any binary or unary operators.
This is due to the results will always be consistent.
However, if the expression contains a method call it is considered indeterminable.
This is due to method's in Whiley have no guarantee of consistency.


Figure~\ref{lst:loop-context} shows an illustration of information captured as
the context is collected though the statements before the while loop.
In particular the variable \code{limit} shows how branches are handled
and that it becomes indeterminable at the loop due to being mutated in the \code{if}
branch.

\begin{figure}[ht]
\begin{lstlisting}
function f(int count): // C = { count: <parameter }
  int[] items = [0; count] // C + { items -> [0; count] }

  int limit = 0 // C +  { limit -> 0 } 

  if g(items): // branch: C -> C`
    int f = 1 // C' + { f -> 1 }
    limit = h(items) // C' + { limit -> h(items) }
  // C` intersection C => C + { limit -> ?? }

  int i = 0 // C + { i -> 0 }

  // Final Loop context
  // C is { count -> <parameter>, items -> [0;count],
  //   limit -> ??, i -> 0 }
  while i <= limit:
    items[i] = i
    i = i + 1
\end{lstlisting}
\caption{Example of the information gathered by the loop context}
\label{lst:loop-context}
\end{figure}

\subsection{Determining a simple mutation}\label{s:simple-mut}

\fixme{Break up and expand on this}

A simple mutation is an expression that only includes addition or subtraction
of a known variable or constant.
This is to test that the expression is a monotonic expression \cite{tanton2005encyclopedia}.
Operations such as multiplication, division and function calls are disallowed
because the monotonic property of these operations are dependent on their
runtime values.
For example a multiplication between a constant and a variable,
the result of the expression may change from being an increasing expression
to decreasing after some unknown point, such as a polynomial equation.

The current design only allows a single variable to be in a simple mutation
expression.
That variable must also be the variable that is being mutated as well.
For example in figure~\ref{lst:simple-mutations} the assignment of \code{i} on
line 3 is simple since it refers to itself variable.
However, the assignment of \code{f} on line 6 is not simple as
it refers to a variable that is not \code{f}.
This limitation is due to a only needing to 
support single variable mutations for the loop invariants
that depended on this mechanism.

\begin{figure}[ht]
	\begin{lstlisting}
	int i = 0
	int f = 0
	i = i + 1 // simple
	i = i + 1 - 10 + 1 + 1000 // simple
	f = g(f)  // not simple
	f = i + 1 // not simple
	f = f * f // not simple
	\end{lstlisting}
	\caption{Example simple mutations}
	\label{lst:simple-mutations}
\end{figure}

\subsection{Determining if a value is increasing or decreasing}\label{s:sequence-dir}

\fixme{Break up and expand on this}

From a simple mutation inside a loop it can be determined if the value is
increasing or decreasing with each iteration.
To achieve this the symbolic executable property of a simple mutation is
exploited to simulate two iterations worth of mutations.
The known variable in the mutation is set to an initial value of zero for
the first iteration and uses the result for the second.
Since the simple mutation is a linear monotonic function it is known that
the difference between each iteration will be constant.
From the result of difference between the first and second iteration the
value can be observed to increase for all iterations if it is positive or
decrease if negative. If there is no difference between the iterations then
it is considered stationary and is considered an error state.
This process is outlined in figure~\ref{math:simple-mutation}.

\begin{figure}
	$$\text{Given that } f(x) \text{ is a linear monotonic function}$$
	
	$$diff = f(f(0)) - f(0)$$
	
	\[
	diff \begin{cases}
	= 0 \quad f(x) \text{ is stationary}\\
	> 0 \quad f(x) \text{ is increasing}\\
	< 0 \quad f(x) \text{ is decreasing}\\
	\end{cases}
	\]
	\caption{Outline of determining if a function increases or decreases with
		each application}
	\label{math:simple-mutation}
\end{figure}

%\subsection{Decomposing Loop conditions and Invariants}\label{s:cond-decompose}
%\comment{
%    Whiley semantics means all where clauses are conjunctively joined, \\
%    A loop condition that is conjunctively joined can be broken into
%    segments
%}

\subsection{Associate length of copied arrays to source}\label{s:copied-array}
\comment{ was a array initialised with a copy from an array type or use a
	generator }

The length of a new array can determine by associating the length to a source
array.
This relies on how the new array is created, for example if the array is
generated using an array generator or from copying another array it can be.
However, if the array is generated by a function it is more difficult to
determine and was out-of-scope.
When using an array generator the length of the array can be determined by
evaluating the count of the generator.
Figure~\ref{lst:simple-mutations} has an example of this for the initialisation
of \code{a}, where the length of the array is associated with the value 10.
In the case of an array copy the array size can be determined by taking the
size of the source array as they will be equal.
This is shown in figure~\ref{lst:simple-mutations} when initialising \code{b}
and \fixme{should associations be transitive? or transitive to a list of values}

This technique is used to associate ghost arrays used to assist in writing
loop invariants.

\begin{figure}[ht]
	\begin{lstlisting}
	int[] a = [0; 10] { |a| -> 10 }
	int[] b = a { |b| -> 10, |a| -> |b| }
	int[] c = f() { c -> ?? }
	
	a = f() { |a| -> ?? }
	
	// final association: { |b| -> 10, |a| -> ??, |c| -> ?? }
	\end{lstlisting}
	\caption{Examples of associated array lengths}
	\label{lst:assoicated-arrays}
\end{figure}

\section{Implemented Loop Invariant Generators}

This section will cover how each loop invariant generator was implemented and their use cases.
Each generator only cover a subset of the cases described 
for their corresponding loop invariant in section~\ref{s:target-loop-inv}.

\subsection{Starting bound}

The implementation of the starting bound invariant covers only linear monotonic counters.
This generator utilises and composes multiple mechanisms to determine 
if a variable in the
loop is a counting variable with a known starting bound to apply the invariant for.
These mechanisms are: the loop context (see section \ref{s:loop-context}) ;
determining a simple mutation (see section \ref{s:simple-mut}) ; and,
determining if a mutation is increase or decreasing (see section \ref{s:sequence-dir}).
From this a variable is known to be increasing or decreasing, this is known as its direction.

The generator supports multiple mutations of the same variable in the loop.
To determine the final direction of the variable each mutation is merged though
using the lattice described in figure~\ref{eq:start-lattice}.
This shows that a variable could have four directions, unknown ($\top$),
increasing ($+$), decreasing ($-$) and indeterminable ($\bot$).
Merges occur on an assignment or when branches, 
such as \code{if}'s or nested loops, join back.
With the lattice a given variable's direction can be determined and also
decided if it is valid (increasing or decreasing) or invalid (unknown or indeterminable).

\begin{figure}
$$\text{values} = \{ \top, +, -, \bot \}$$
\[
\text{Lattice} \begin{cases*}
\quad X \cup \top = X \\
\quad X \cup X = X \\
\quad X \cup Y = \bot
\end{cases*}
\]
\caption{Starting bound lattice to determine mutation direction}
\label{eq:start-lattice}
\end{figure}

For variables that have a valid direction and a valid entry in the loop context an invariant
can be generated.
There are three components to the generated invariant: variable being mutated, an inequality based
on the variable's direction, and the known value of the variable before entering the loop.
The inequalities used in the expression are appropriate to the direction. 
If the variable was increasing then the variable will be greater than or equal to the known value
and vis versa for a decreasing variable.

To illustrate this generator the figures~\ref{lst:whiley-start} and \ref{lst:whiley-start-if}
show simplified examples.
Figure~\ref{lst:whiley-start} shows the most simplistic and common use case for this
generator. Where figure~\ref{lst:whiley-start-if} shows a more complex case 
with an \code{if} branch that changes the magnitude of the mutation but not the direction. 

\begin{figure}[ht]
\begin{lstlisting}
...
int i = 0

while i < |items|:
  // 'where i >= 0' is inferred
  apply(items[i])
  i = i + 1
...
\end{lstlisting}
\caption{Simple example of inferring starting bound of index}
\label{lst:whiley-start}
\end{figure}

\begin{figure}[ht]
\begin{lstlisting}
int i = x

while i < |items|:
  // 'where i <= x' is inferred
  apply(items[i])
  if skip(item[i]):
    i = i - 2
  else:
    i = i - 1
\end{lstlisting}
\caption{Starting bound invariant with branch}
\label{lst:whiley-start-if}
\end{figure}

\subsubsection{Limitations}

There are many limitations to the starting bound generator.
The main root of these limitations is only handling the case
of single variable linear monotonic functions. 
This prevents the invariant being generated for other
methods of counting or using complex logic that determines
the mutation each iteration.

One example is having multiple nested loops to count or index
the inner-element then combining them together to a total variable.
In this example each counter for the nested loops will have a generated
starting bound invariant but the total will not.
This is due to the mutation of the total failing the test of a simple
mutation where the only variable allowed is itself.

Another example of the limitations present is shown in figure~\ref{lst:whiley-start-limit}.
This shows that the starting bound invariant will not be generated because
the direction is indeterminable.
Even though on inspection it is clear that \code{i} will not decrement.
However, the components of the generator does not understand this
since they lack the ability to test branches for tautologies
to prune their traversals. 

\begin{figure}[ht]
\begin{lstlisting}
int i = 0

while i < |items|:
// 'where i >= 0' cannot be  inferred
apply(items[i])
  if false:
    i = i - 1
  i = i + 1
\end{lstlisting}
\caption{Example of a limitation of the Starting bound generator}
\label{lst:whiley-start-limit}
\end{figure}

\subsection{Equal array length}

\fixme{remove techniques to design}

The equal array length invariant generator is implementation handles copied and
generated arrays.
This generator unities the associated array mechanism to determine if the
length of two arrays are equal (see Section~\ref{s:copied-array}).
From this association the generator then discovers what arrays should
be asserted equal in the loop invariant by checking if they are used in the
loop body.
This is to filter out known unessential invariants since they are known not to
be used.

The figure~\ref{lst:whiley-length} shows an example of the equal array length
invariant being generated.
Lines 1 and 3 show that the second array's association can be formed by either
copying or generating an array with by using the length of another array.
From either of these lines \code{copy} and \code{items} are linked until
either of these arrays are re-assigned or one falls out of scope.
Line 6 show that both of the associated arrays are used, one in an expression
and the other having an element assigned.
Both of these uses are examples of what the generator considers as usage in
the loop body.
Given these findings the generator then can generate the invariant that
the two associated arrays equal length as shown on line 5.

\begin{figure}
\begin{lstlisting}
    int[] copy = items
    // or
    float[] copy = [0.0;|items|]
    while i < |items|:
        // 'where |copy| == |items|' is inferred
        copy[i] = apply(items[i])
\end{lstlisting}
\caption{Simple example of inferring array lengths are equal}
\label{lst:whiley-length}
\end{figure}

\subsubsection{Limitations}

The limitations present in the implementation of the generator is
that other methods of creating arrays are not supported.
This includes the return of a function and an array constant.
In the case of a function the post-condition would need to
give a specification about the length of the array to create
such an invariant.
These are illustrated in figure~\ref{lst:whiley-length-limit}.

\begin{figure}
\begin{lstlisting}
function zeroes(int len) -> (int[] r)
requires len >= 0
ensures |r| == len:
    return [0;len]

function f():
    int i = 0
    int[] a = zeroes(5)
    int[] b = [1,2,3,4,5]
    while i < |a|
    // where |a| == |b| could be inferred
        // omitted
\end{lstlisting}
\caption{Limitations of equal array length invariant generator}
\label{lst:whiley-length-limit}
\end{figure}

\subsection{Upper bound}\label{s:upper-bound-impl}

The upper bound generator is implemented to only handle a change of one each iteration.
This generator is made from composing and modifying mechanisms from other generators.
The core mechanisms used are finding a simple mutation (see Section~\ref{s:simple-mut}) ,
a modification to determining if the mutation is increasing or decreasing (see Section~\ref{s:sequence-dir})
and breaking down the loop condition.
From these the loop invariant can be generated.

For the upper bound invariant the generator needs to know how much the variable changes
every iteration. 
To achieve this the mechanism to determine if the expression is increasing or decreasing is
modified to expose the change of value between each iteration rather than if it is a positive or
negative change.
The lattice in figure~\ref{eq:start-lattice} is modified such that the values used in the lattice
are now integers instead of only positive or negative.
The case of handling assignments were also modified such that the value is summed instead of merged.
This allows for the total change for a block of code to be computed.
The current implementation only allows a total change of plus or minus one.
This is due to the generator's instability when generating other values.

After the value that a variable changes every iteration is identified the loop
condition is used to identify an upper bound.
To achieve this the loop condition is broken down into each conjunctive parts to handle
separately.
Each of these parts are checked for a simple inequality that contains a variable
with a known mutation compared to a bound.
Once a candidate condition is found it's weakened by applying the change of the variable
to the bound.
An optimisation is added that when the changed value is plus or minus one with the condition containing
a strict inequality of the corresponding type it is weakened to a regular inequality.

\begin{figure}
\begin{lstlisting}
function find(int[] items, int item) -> (int r)
ensures all { i in 0..r | items[i] != item }
ensures r != |items| ==> items[i] == item:
	int i = 0
	while i < |items|
		where all { j in 0..i | items[j] != item }
		// `where i >= 0` inferred starting bound
		// `where i <= |items|` inferred upper bound
		if items[i] == item
			break
		i = i + 1
	//
	return i
\end{lstlisting}
\caption{Example of a generated upper bound invariant}
\label{lst:whiley-upper-bound}
\end{figure}

Figure~\ref{lst:whiley-upper-bound} illustrates this a case when this invariant is generated.
The value that \code{i} changes each iteration is clear on inspection that it is one and
the condition of the loop both contains \code{i} and a simple inequality.
In this case the optimised method is used by weakening the strict inequality of less than to
less than or equal to.

\subsubsection{Limitations}

The main limitation applied to this generator is only being able to 
handle the case of a changed value of plus or minus one.
This is the most common case for this generator but it could handle
some more simple cases.

Another limitation is in how the loop condition is broken down.
Currently the simple task of splitting on conjunctions.
This has limits the possibility to identify an upper bound.
To relieve some this limitation the conditions could
be rearranged or simplified to better identify an upper bound. 

\subsection{Array iterative assignment}

The array iterative assignment generator has many steps to build
a universal quantifier.
The first step is to identify an array that has elements being assigned each iteration.
For that assignment the indexing expression used is checked that it linearly
iterates through the array.
From there the element assignment is checked to be not depend on itself.
With this the invariant can be generated.
This generator is mostly a composition of the starting and upper bound generators with
the addition of checking assignments.

To identify an assignment that occurs every iteration the generator
only considers top-level statements in the loop body.
This disallows variables that are used in other blocks such as
\code{if}'s or nested loops.
With this restriction it is known that every iteration of the loop 
the next element is assigned.

To ensure that each element is assigned in ascending or descending order the index variable's mutations
are checked.
This is to make sure that the element assignments proceed one by one and in a consistent order.
To achieve this the index variable is check that it's mutated by plus or minus one each
iteration in the same manner as the Upper Bound (see Section~\ref{s:upper-bound-impl}).
This lets the generator knows that the assignments are ascending or descending order.

With the ordering and assignment the loop invariant can be built.
There are four components of the universal quantifier generated.
The first is the variable name of counter in the quantifier expression,
this is derived from the index variable name.
For example in figure~\ref{lst:array-iter-assign} the index
variable \code{i} is transformed into \code{\_i}.
This done so there is traceability between the generated invariant to
its the original parts.

The range of the quantifier is split into two parts, the lower and upper
bound. One of the bounds will be the starting value of the index variable
and the other the index variable itself.
The ordering of this is dependent on if the assignment is ascending or 
descending through the array.

The final component is the expression of the universal quantifier.
This is a modification of the original assignment in the loop body.
First the assignment is transformed into a boolean equal.
The second modification is replacing all occurrences of index variable
with the quantifier's counting variable.
From this the invariant is complete.  

\begin{figure}
\begin{lstlisting}
int[] a = [0 ; 100]
int i = 0
while i < |a|
	// `where all { _i in 0..i | a[_i] == f(_i) }`
	a[i] = f(i)
	i = i + 1
\end{lstlisting}
\caption{Example array iterative assignment generator}
\label{lst:array-iter-assign}
\end{figure}

\subsubsection{Limitations}

A limitation of this generator is that the assignment expression cannot
reference itself.
This is shown in figure~\ref{lst:limit-array-iter-assign} where the assignment
of \code{a[i]} depends on the previous value of \code{a[i]}.
The limitation of is from the generator not being able to compare the previous
value of the element.
A possible solution of this would be to introduce a ghost variable to hold the
previous value, however this functionality is undesirable (see section~\ref{s:design-create-var}),
but could utilised if introduced as a language feature.


\begin{figure}
\begin{lstlisting}
int[] a = g(100)
int i = 0
while i < |a|
	a[i] = a[i] + f(a)
	i = i + 1
\end{lstlisting}
\caption{Limitation of array iterative assignment generator}
\label{lst:limit-array-iter-assign}
\end{figure}
