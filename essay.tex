\documentclass{style/CRPITStyle}
\usepackage{epsfig}   % Packages to use if you wish
\usepackage{lscape}   %
\usepackage[authoryear]{natbib}
\renewcommand{\cite}{\citep}
\pagestyle{empty}
\thispagestyle{empty}
\hyphenation{roddick}

\begin{document}

\title{Software Development Life Cycles: History and Future}
\author{David Barnett}
\affiliation{School of Engineering and Computer Science \\
Victoria University of Wellington, \\
PO Box 600, Wellington, 6140 \\
Email:~{\tt barentdavi@myvuw.ac.nz}}

\maketitle

\begin{abstract}
    % TODO
\end{abstract}
\vspace{.1in}

\noindent {\em Keywords:} Software Process Models, Sequential, Agile Methods

\vspace{.1in}

% \section{Introduction}
% TODO
% thesis statement

\section{Importance of using process models}
% Q: describe the importance of using process models to guide software development
Fail to plan, is a plan to fail. 
A process model is a model of how to setup a project with a set of tasks to perform,
such as requirements acquisition or construction, of a similar nature that together form 
the scaffolding of a plan of how to approach to build a project.
As software has increased in complexity and scale of projects increase it
became more apparent that software should be managed and planned differently.
\cite{nato:1969}

\vspace{.1in}

A software project may face a variety of problems in many aspects of the
development. A single  

There are a variety of different process models giving different solutions to
the variety of different problems that a project could face. This can ranges from
having a ever changing set of requirements to dead certain requirements
enshrined in law so different process model reflects their suitability to these constraints
with their processes such as scope management of the project or lack there of.

% --- 5 Model Section ---
% Q: provide a systematic review of process models that have been proposed so far,
% Q: compare and contract 5 selected distinct process models for software development
% Q: identify central themes that have been considered important for evolving process models
\section{Waterfall} % sequential
% theme: sequential
% strengths: requirements up front, easy
% compare:
% contrast:

The Waterfall model \cite{Waterfall:1970}

\section{V-Model} % sequential, testing
% theme:
% compare:
% contrast:

\section{Spiral Model} % sequential / Iterative, testing, risk
% theme:
% compare:
% contrast:

\section{Phased Development Model} % iterative, testing, risk
% theme:
% compare:
% contrast:

\section{Extreme Programming} % iterative, testing, risk
% theme:
% compare:
% contrast:

\section{Possible evolutions of process models}
% Q: try to forecast how current process models may evolve in the future


\bibliographystyle{agsm}
\bibliography{essay}

\end{document}

% ------- Example Area ------
% section example
% \section{Heading Level 1}
% \subsection{Heading Level 2}
% \subsubsection{Heading Level 3}

% figure example
% \begin{figure}[htb]
% \fbox{\parbox[b]{.99\linewidth}{
% \vskip 0.5cm
% \centerline{Figure Content}
% \vskip 0.5cm}}
% \caption{\protect\label{xyz}  Caption}
% \end{figure}

% ...as proved by \citet{Snodgrass87} and \citet{FPS96} and referred to in other works \cite{BenZvi82,Bentley86,AIS93} the process..., etc.

% list example
% \begin{enumerate}
% \item Not having the correct margins -- they are 0.8 inch all round.
% \item Not using A4 paper format when creating the pdf file.
% \end{enumerate}
% vim:set spell et sw=4 ts=4 tw=80:
