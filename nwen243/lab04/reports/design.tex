\documentclass[12pt]{article}


\usepackage[margin=1in]{geometry}  % set the margins to 1in on all sides
\usepackage{graphicx}              % to include figures
\usepackage{minted}

\author{David Barnett (ID:300313764)}
\title{NWEN243 Lab 4 -- Routing -- Design}
\date{}

\begin{document}

\maketitle

The design of the routing algorithm was based off the Distance Vector routing
algorithm. Distance Vector routing is aimed to make distributed routing with
each node have independent routing table.
The algorithm has three main parts: advertising the node's best routes to all
of nodes to its neighbours; updating  the routing table from the routing
tables shared by the node's neighbours; and routing packets to their
destination for the lowest cost.

\section*{Routing table}

The routing table is a table with rows representing each node in the whole
network and each column being a physical link from the current node to a
neighbour.

With deciding the best path of the packets the cost of transmission of the
packets are taken into account. The metric used in this version was cost
per frame plus the cost given by the other nodes.

\begin{minted}{c}
    int cost = table[i] + linkinfo[arrived_on].costperframe;
\end{minted}

With table[i] being the table of best routes from a neighbour.

\subsection*{Building routing table}

The routing table is built over time from information shared from neighbours.
On startup the routing table is empty with all values set to infinity
representing that there is no known route to the node expect for the node
itself which is initialised with the value of 0.

\paragraph{}

Over time the node will receive \emph{NL\_ROUTING} packets from its neighbours
and call \emph{update\_routing\_table} to unpack the packet to update the routing
table. The table is adjusted using the metric then compared against the local
table and updating for the minimum cost.

\begin{minted}[linenos]{c}
        if (rtable[node][arrived_on] != cost && cost < INF) {
            int old = rtable[node][arrived_on];
            rtable[node][arrived_on] = cost;
            change = 1;

            if (old >= INF) {
                CNET_enable_application(p->src);
            }
        }
\end{minted}

When a node changed from infinity (\emph{INF} in the code) to non-infinity the
node has a valid route from the current node to the source of the packet. After
the table is updated if there was any changes the node will then advertise the
nodes it can reach through its links for their given cost to all of its
neighbours.

\subsection*{Sharing routing table}

After the node updates its routing table or is initialised the node shares the
best of its routing table to all of its neighbours.

\begin{minted}[linenos]{c}
    for(i = 0; i < rtable_size; i++) {
        table[i] = INF;
        for(link = 0; link <= nodeinfo.nlinks; link++) {
            if (rtable[i][link] < table[i]) {
                table[i] = rtable[i][link];
            }
        }
    }
\end{minted}

The above code collapses the 2D routing table into an array of the best routes
this node can provide. The node sends this array off to be shared around.

\section*{Routing packets}

As the node forms more and more routes to nodes more application level packets
are sent and need to be managed. This is when the routing table is actually
used to route packets to their destination. To do so the node must determine
which link is best to use to reach a given node.

\begin{minted}[linenos]{c}
    for (i = 0; i < nodeinfo.nlinks+1; i++) {
        if (rtable[p->dest][i] < min) {
            min = rtable[p->dest][i];
            link = i;
        }
    }
\end{minted}

The above code does just that, choosing the best link for the job. Afterwards
the node just has to fire that packet down the link and that node will deal
with the rest.

\end{document}
