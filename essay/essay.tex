\documentclass{CRPITStyle}
\usepackage[authoryear]{natbib}
\renewcommand{\cite}{\citep}
\pagestyle{empty}
\thispagestyle{empty}
\hyphenation{roddick}

\begin{document}

\title{When to use NoSQL}
\author{David Barnett}
\affiliation{School of Engineering and Computer Science \\
Victoria University of Wellington, \\
PO Box 600, Wellington, 6140 \\
Email:~{\tt barentdavi@myvuw.ac.nz}}

\maketitle

\begin{abstract}
    With a large range of new database options in the current environment it
    is hard to know what are NoSQL databases and when to use them.
    In this report the data model and operations of three types of NoSQL
    databases (Document store, Columnar, Graph) are explained.
    This also includes some discussion of the operation of each database
    and a use case to highlight when a NoSQL database could be utilised.
\end{abstract}

\vspace{.1in}

\noindent{\em Keywords:} NoSQL, Use cases, Document Store, Columnar, Graph database

\section{Introduction}
% Cover the aim of the essay

\paragraph{}
With growing use of NoSQL databases in production environments it has become more
important to understand when and why to use them over a traditional
relational database.
NoSQL can be broken down to many different types which differ on how
data is represented~\cite{type_nosql}.
For example the traditional relational databases are based upon relational
model and relational algebra~\cite{relational_db}.

\paragraph{}
This report will cover some common types of NoSQL databases.
These are: Document store; Columnar; and, Graph databases.
For each of these types this report will give an overview on how each type
operates and a description of its use case relative to a relational database
offering.

\section{What is NoSQL}
% brief overview

\paragraph{}
NoSQL is a class of databases that are not based on
a relational data model~\cite{nosql_db,nosql_survey}.
Instead of using relations the data is modeled using a large
variety of models are used, such as graphs, documents or objects.
These allow for application specific optimisations that
a relational database could not provide.
The need for these optimisations are motivated by the growing
volume of data used in large distributed systems~\cite{nosql_db,nosql_survey}.

\subsection{CAP Theorem}

\paragraph{}
The CAP theorem states that no distributed system can be
simultaneously always consistent, available and
partitions tolerant but at most two of these properties~\cite{base,nosql_db,compare_nosql,nosql_survey}.
This is due to the conflict between these properties, such as a
strictly consistent database handling an update would
need to lock the modified data as the update is being distributed
so all clients have the same consistent data.

% trade-off between consistency, availability and partition tolerant

\paragraph{}
In the case of databases they can be designed to provide a
range of guarantees and to how the data is stored and retrieved.
For relational databases they are designed to hold
the ACID properties~\cite{relational_db,base}.
Figure~\ref{l:acid} provides a brief outline of the ACID properties.
These properties ensure that the data committed will be stored in a
consistent manner that allows some concurrency.
This is highly desired in applications that require guaranteed consistency
for critical data such as bank accounts.
In terms of the CAP theorem the consistency aspect is the most important
guarantee.

\begin{figure}
\begin{itemize}
    \item \textbf{A}tomic - every transaction will either
        succeed and be applied or fail and commit no modifications.
    \item \textbf{C}onsistent - the state of all relations are
        always in a consistent state.
    \item \textbf{I}solation - transactions are performed as if sequentially.
    \item \textbf{D}urability - after a successful transaction the results
        are persistent over crashes.
\end{itemize}
    \caption{Outline of the ACID properties~\cite{relational_db,base}}\label{l:acid}
\end{figure}
%
% give definition of a database? (managed store of data)
% relational is based on set theory and relational algebra

\subsection{BASE}
% What it is

\paragraph{}
An alternative to ACID is BASE where the objective is replaced from consistency
to availability~\cite{nosql_survey}.
This is desirable in large distributed systems that do not need to be
fully consistent but require to always be available.
For example an application will not mind that it has some stale or
inconsistent data since it has been built to handle it.
This is generally present in NoSQL databases when used in multiple
locations, such as a distributed document store~\cite{base}.
Figure~\ref{l:base} has a brief outline of these properties.

\begin{figure}
\begin{itemize}
    \item \textbf{B}asically \textbf{A}vailable - When a part of the system
        fails only some or none of the data maybe be unavailable rather than
        all of it.
    \item \textbf{S}oft state - The data present on a node may change over time
        without a direct update as eventual consistency is applied.
    \item \textbf{E}ventual consistency - The shared data on a node will become consistent with
        other nodes over time.
\end{itemize}
\caption{Overview of BASE properties~\cite{base}}\label{l:base}
\end{figure}


% === Document Store ===

\section{Document Store}
% How does it work
%  - key to a structured document
%  - document can be inspected by DBMS
% Relate it back to BASE

\paragraph{}
Document store databases are designed to store their data in the form of
structured documents~\cite{base,nosql_survey}.
There are a wide range of structured documents used with document store
databases, such as XML, JSON and more~\cite{base,compare_nosql}.
Two implementations of a document store database are: \textbf{MongoDB} which
structures its data as JSON~\cite{base}; and, \textbf{BaseX} which is based on XML
documents~\cite{basex}.
Though the format of the documents are well defined the
way how data is laid out within this format is not.
This makes documents stores a semi-structured database.

\subsection{Data model}

\paragraph{}
The data model for a document store can be simpler relative a relational
model~\cite{nosql_survey}.
A single document is an instance of an entity, such as someone's contact information.
The document is addressable by a unique key, similar to a key in a key-value
store~\cite{nosql_survey,sql_nosql}
Where in a relational model someone's contact information would be
normalized to be spread over many tables as there are many optional
sections.
In this case the document representation is simpler as it has all of the
information in one place.
This lead to the pattern of storing all the data for a single entity into
one document rather than spread it over many relations that then need to
joined.
Documents can be grouped by various means,
these include: metadata; hierarchy; tags; and, collections~\cite{sql_nosql}.
With the groupings queries can be refined to certain groups to optimise
queries.

\subsection{Operations}

\paragraph{}
The usage of a document store database is as simple as the data it hold.
This is due to simplicity of the API required to retrieve and store data into
the database.
As evident by implementations of document stores that model all operations
with a RESTful interface, such as CouchDB~\cite{couchdb}.

\paragraph{}
A benefit of the semi-structure of a document is the ability to query it
without a schema defining its layout.
This is achieved by databases leveraging the known properties of the
document format to allow for optimised traversals.
For example in a document store using XML the query language XPath is a
powerful tool to be able to query elements of the document
efficiently~\cite{xpath}.

\paragraph{}
% examples of querying
Figures~\ref{lst:mongo} and~\ref{lst:basex} illustrate queries on a document
store about books.
These figures show how the each database uses the format to search
the documents.
For example, the MongoDB query in figure~\ref{lst:mongo} that would retrieve all
stored documents about books with the title
\textit{Database}~\cite{nosql_db}.
This query is simply looking for a union between the given object and a
document that shares the same structure.

Another example the BaseX in figure~\ref{lst:basex} with its
ability to use XPath to query a document~\cite{basex}.
This query is functionally equivalent to the MongoDB query before, expect it uses
an XML specific query language to achieve it.
These queries return documents that share the structure and values used in the
query.

Since the documents are not enforced to have this structure not all documents
checked during the search would match, even simply on structure.
For example if audio books were also included in the document store the user
could query for play time or page count and only retrieve a book that these
attributes apply to.

\begin{figure}
\begin{verbatim}
books.find({ title: "Database" })
\end{verbatim}
\caption{Example query for MongoDB}
\label{lst:mongo}
\end{figure}

\begin{figure}
\begin{verbatim}
/book/title[text() = 'Database']
\end{verbatim}
\caption{Example query for BaseX using XPath}
\label{lst:basex}
\end{figure}

\subsection{Use case}
% When are people using this
%  - JSON store
%  - XML store
% Example of use case
%  - Searching

\paragraph{}
The use case to use a document store database is when
the data is unstructured and can have large range of attributes.
For example a database about all the parts and serial numbers
of a car.
Each part will have common attributes, such as its serial numbers,
manufactures, etc. 
However, they also have varying attributes such
as number of ignitions for a spark plug or total revolutions for
a tyre.

\paragraph{}
For a document store database could each part in its own document.
Each of the parts would need its name and serial number, this would
be enforced by the application as document stores do not enforce these
constraints~\cite{nosql_db,basex,nosql_survey}.
From there any additional attribute would be optional and could
be added to any part with any type.
This has the downside of possibly having a shared attribute that
is inconsistently encoded which will have to be handled by the application
code.

\paragraph{}
In comparison a relational database that provides the same functionally
would be more complicated than the document store.
The relational approach would be to have a \textit{parts} table
and a large number of optional tables that can be joined to the parts
table to get all of the part's attributes.
This has the downside of using large joins that will slow down
queries~\cite{scalable_sql}.

% should include what people use it for and why they choose it

% \subsection{When to use it}

% === Columnar ===

\section{Columnar}
% How does it work
% Relate it back to BASE

\paragraph{}
Columnar or column-orientated databases are a type of NoSQL databases that
stores data in tables by column rather than by row~\cite{nosql_survey}.
The goal of a columnar database is to store large quantities of data efficiently
while maintaining efficient operations on columns~\cite{nosql_survey,nosql_eval}.
Implementations of a columnar database are Google's
\textbf{BigTable}, which is used for large scale~\cite{bigtable}, and
\textbf{Cassandra},~\cite{cassandra}.

\subsection{Data model}

% data model
\paragraph{}
The data model of columnar databases is a multi-dimensional
map~\cite{bigtable,nosql_eval}.
Each instance of an entity is a row in a table.
With each row it is given a row name, this is used to uniquely identify the
given row.
Each row has a set of columns that describe the entity
The columns maybe grouped together into a column family which then are
stored together.
The members of a column family are generally all the same data type as each
other~\cite{bigtable}.
This is to allow optimised retrieval of commonly paired columns effectively.
To query data from a row of the database the row name and column name are
required to uniquely access the data~\cite{bigtable}.

\paragraph{}
Columnar databases are similar to relational databases but has some key
characteristics that differentiate them.
These range from how the data is stored on disk, what parts of the
data is indexed, how foreign relations are handled and more~\cite{nosql_survey}.

\paragraph{}
The column-orientated method of storing data in columnar databases is a
key characteristic.
To achieve this each column is serialized to disk
separately~\cite{bigtable,nosql_survey}.
This allows for faster reads of a single column, such as aggregates.
Figures~\ref{lst:row} and~\ref{lst:column} show the difference in serialisation
between a row-orientated database and a column-orientated database where each
line is a separate file.
A side effect of having similar data stored together is a greater potential to
compress the data more efficient~\cite{nosql_survey}.
This enables a columnar database to be sparse with its storage of data~\cite{bigtable}.

\paragraph{}
The common usage of indexes in databases are to store the indexed columns
separately for quick access~\cite{relational_db}.
However, in columnar databases all of the columns are already stored
separately and thus an index~\cite{nosql_survey}.

\paragraph{}
A characteristic of columnar databases is not supporting associations between
rows.
This implies that common features from relational models, such as a foreign relation,
are not supported~\cite{relational_db,nosql_survey}.
Thus when making changes to a column only the effected column would be needed
to be checked for consistency errors.
This allows for many concurrent reads and writes across the database as each
column is independent in both consistency and in storage~\cite{nosql_survey}.


\begin{figure}
\begin{verbatim}
    SWEN304:Database Egineering:T2:2017
    COMP304:Programming languages:T2:2017
\end{verbatim}
\caption{Example of a table stored by rows}
\label{lst:row}
\end{figure}

\begin{figure}
\begin{verbatim}
    SWEN304:COMP304
    Database Egineering:Programming languages
    T2:T2
    2017:2017
\end{verbatim}
\caption{Example of a table stored by column}
\label{lst:column}
\end{figure}

\subsection{Operations}

\paragraph{}
Using a columnar database is similar to a relational database~\cite{usingcolumn}.
Some implementations of columnar databases even use similar query languages to
SQL that is common in relational databases.
For example, Cassandra offers a query language that is very close to
SQL called CQL~\cite{cassandra}.
This allows knowledge from relation databases to be easily transferred to
columnar databases.

\paragraph{}
Accessing data in a columnar database is closer to a key-value
store than a relational database.
This is because to access a column of a database the composite of
the row name and the column name is used to identify the
data~\cite{nosql_survey,usingcolumn}.

\subsection{Use case}
% When are people using this
%  - Data warehousing
% Example of use case

\paragraph{}
The use case for columnar database for long term storage of large volumes of
data, or commonly referred to data warehousing~\cite{usingcolumn,bigtable}.
This is due to its ability to efficiently store the data on disk
due to the compressibility of data when stored as columns as opposed
to rows.
Another factor is that even with large volumes of data columns can
be quickly accessed as they are effectively indexed when stored as
columns.
This allows for fast aggregations over a column.

\paragraph{}
Compared to a relational database the feature set of a columnar database.
This is the trade-off between optimising for disk space and having
complex features such as foreign key relationships.
The speed to read data is comparative to an indexed attribute
for a relational database as each column in a columnar database is
indexed.

\paragraph{}
The time to use a columnar database is when long term storage is important.
This is because of the space savings of compressed columns but still able
to use the data as indexes.

% === graph ===

\section{Graph}
% How does it work
% Relate it back to BASE

\paragraph{}
Graph databases are a class of NoSQL databases that are based on graph
theory~\cite{nosql_survey,sql_nosql}.
The goal of a graph database is to capture the connections between
nodes and being able to traverse them,
such as hierarchies~\cite{type_nosql}.
This is achieved by using the graph data model which is based upon 
sets of nodes, edges and properties~\cite{nosql_survey,sql_nosql}.
Some implementations of graph databases are \textbf{Neo4j}~\cite{neo4j}
and \textbf{GraphDB}~\cite{graphdb}.

\subsection{Data model}

\paragraph{}
The data model for a graph database is built on graph
theory~\cite{nosql_survey,sql_nosql,graphdb}.
Thus it uses the building blocks of graphs (nodes, edges and
properties~\cite{graphdb}).
Each nodes in the graph represent an entity in the model.
The properties of the graph are represent the attributes of 
an instance of an entity.
These are key-value pairs on the node with no schema to defined what
key-value pairs that should be provided~\cite{neo4j}.
This would be comparative to a single row in a relational data
model~\cite{relational_db}.
The edges between two nodes represent the relationship between
them.
These edges can be directed or bi-direction, depending on the database
implementation~\cite{graphdb,neo4j}.
The edges may also provide some properties
An example of an edge would be an associate relation between two user accounts
to store such a relationship between the users.

\paragraph{}
Due to the data model graph databases have many properties and
characteristic that unique.
These are all derived from the use of graph theory.
Some of these properties include:
constant time traversals over any size graph;
deep traversals across many edges;
and, shortest path calculation.

%Graph traversals are executed with constant speed independent of total size of the graph.
%There are no set operations involved that decrease performance as seen with join operations in RDBMS
\paragraph{}
Graph databases are well optimised for traversals between nodes.
This occurs in constant time no matter the size of the graph~\cite{sql_nosql}.
This is due to all edges between the node points directly
to the other node.
As a result of this it is very cheap to traverse through many
edges to find a deep relationships of a given node in a single
query.
In comparison if this was attempted in a relational database
the act of traversing multiple layers would be difficult and
require many expensive join to traverse deep
relationships~\cite{sql_nosql_gap,relational_db,neo4j}.

\paragraph{}
A property of the graph model is that joining many nodes is cheap.
This is in contrast of relational databases where joining relations
together is a costly~\cite{relational_db}.
This difference is from how relations are stored in each of these
databases.
For a graph database a relation between two entities can be made
by adding an edge between them.
To retrieve this data it only takes a simple traversal to the
node~\cite{neo4j}.
However, in a relational database the relationship must be included
in the schema as a foreign key.
To retrieve this data the foreign relation must then be searched for
which tuple each given row is referring to~\cite{relational_db}.

%These are used for shortest path calculations.
\paragraph{}
Since a graph database is based on graph theory all of the
traversal algorithms can be applied to the database~\cite{sql_nosql}.
This allows for a large variety of complex queries that is
not practical in a relational database.
For example the shortest path between two nodes or finding
clusters of nodes.
From applying these algorithms a new perspective on the data can
be found that no other database type could offer.

\subsection{Operations}

\paragraph{}
The operation of a graph database is not as simple as
other databases.
Where other databases will have loosely coupled entities
or none at all, a graph has direct ties between nodes.
This can lead to some additional features in the operation
of the database and some additional complexities.

\paragraph{}
The general usage of a graph database is simple.
This includes creating a new node, adding a property to one
and forming a relationship between two nodes.
This is achieved by a simple query languages such
as Cypher from Neoj4~\cite{neo4j}.
The language is modeled on SQL so a relational database
developer can read the queries with existing knowledge.
However, the introduction of new syntax for traversals
in the language makes this slightly more difficult.

\paragraph{}
One benefit of the tightly coupled nodes is cheap
adjacency lists~\cite{type_nosql}.
These are critical part of making an efficient graph
algorithm as they are required in every operation.

\paragraph{}
Due to the tightly coupled nodes some graph databases
implement the ACID properties~\cite{type_nosql,neo4j}.
An example of this is Neo4j which has support for
concurrent transactions~\cite{neo4j}.

\paragraph{}
A difficulty in using a graph database is that it is complex
to distribute it over many servers~\cite{sql_nosql}.
This is due to the complexity of balancing the
graph across many servers due to how the graph is
cut determines how effective the balancing is.
For example if the graph is split such that all of the
actively queried nodes are one server then the graph is
not distributed well since the load is not balanced across
all servers. If the nodes are naïvely spread across the nodes
there maybe many edges that require traversals across servers
which slows down the overall query time.
To resolve this a minimal cut of the graph is taken, minimising edges
between servers but still spreading the node count~\cite{neo4j}.

\subsection{Use case}
% When are people using this
% Example of use case

\paragraph{}
The use case for a graph database is in a relationship
heavy application.
This is due to its advantage of efficiently handling
common graph operations such as adjacency lists~\cite{neo4j,nosql_survey}.
An example of this would be a social network that keeps track
of relationships between users.
The main advantages of using a graph database in this context is
the complexity of operations such as suggesting friends-of-friends is
much less than a relational database~\cite{nosql_performance,neo4j}.

% finding colleges or friends
% Using adjacency lists to recommend things
% Literal graph work 

\section{Conclusion}

% briefly cover:
%  * what each database is
%  * when to use each database
% end with: there are many good situations to use NoSQL database etc.

\paragraph{}
Each NoSQL database has a use case that it has been optimised for.
These are evident from looking at the data model and the operations of
each type of database.
In this report three types of NoSQL databases have been discussed.
The first was Document store which is a semi-structured database that
utilises the format of the documents to enable fast queries of the stored
documents.
This is best suited to be used when the data is largely self contained and can
be serialized into a uniform data format such as JSON or
XML~\cite{type_nosql,basex}.
The second was Columnar which is a column-orientated store that focuses
on the storage of large volumes of data then relationships between
rows.
This is best suited for long term storage of data since it is optimised to
be efficient with its disk space usage~\cite{bigtable}.
The third was Graph database which used graph theory to model the data.
This is best suited for relationship heavy applications that would
be utilising many graph-based algorithms to extract insight from the data.


\bibliographystyle{agsm}
\bibliography{essay}

\end{document}
