\chapter{Design}\label{C:design}
% The aim here is to identify the key trade-offs in any design work you have undertaken.
% When solving a complex problem, there are normally many different
% approaches one can take — each with its own advantages and disadvantages.
% It is expected that students will have initially considered a range of different solutions,
% and will then have narrowed these down. The reasons why a particular approach was
% discounted should be documented here.
%
% Remember:
% Appropriate design notation (e.g. UML diagrams) can be very helpful in conveying different aspects of a design.
% It is vital that your design not be carried out in a vacuum.
% Your design should be motivated very clearly by your goals and specifications.
% Make sure that it is clear why you took the decisions that you did.
% Do not give the impression that you settled on a design because it “felt right”
% or that you tinkered around until you found something that worked. 

% intro to problem, what is the design space for this issue
%

\section{Considerations}
% These are designs for the project and the changes and impacts on the Whiley
% Language & compiler

\subsection{Backwards Compatibility}
% pre-existing code should not fail because of the generated invariants
% code that compiles without the flag should still compile with the flag
% restricts loop invariants to be precise  and strict to avoid "wrong"

\subsection{Inspection and Debugging}
% Allow debugging / reviewing  of generated code
% achieved by emitting compiler messages about what, where and why they are
% generated

\subsection{Creating Variables}
% Whiley does not have the notion of ghost variables
%  * possible name collisions
%  * This is to be avoided
%
%  Does force the programmer to make ghost variables or make copies
%  that make sense if they were making the invariants themselves vs generated

\subsection{Usability}
% COMPILER FLAGS
% How easy it is to control the functionality 
% Error messages

\subsection{Impact on Compile time}
% No goal on time impact
% choice to not repeat compilation and verification

\subsection{Essential Invariants}
% It is known that all invariants genereated will not be useful to proving the post-condition
% (due to P-C has no input generated inv)
% It is OK to have the minimal set of invariants
% to test if it is essential requires repetitive verifications, infringes on
% impact on compile time


% Biggest design decision  was at what level are the invariants going to be inferred
% then where are they going to be generated
\section{Approaches to Loop Invariant Generation}

The principle design decision of this project was at what point in the Whiley Compiler pipeline
to generate the loop invariants at.
There are three main stages in the Whiley Compiler that the input program will 

\subsection{Working on the Whiley Intermediate Language (WyIL)}
% possible method to create tool on WyIL
% What could be achieve with this
%  * is designed to be a binary version of the code
%  * WyIL goal is to make intermediate language between ALL targets of Whiley
%  * AST converted into loosely connected values via integers
%  * less information readily available 

\cite{wyil}

\subsection{Working on the Whiley Specification Language (WyAL)}
%% WyAL
% what it is
% * changes would be more dramatic
% * Does not have the same level of control
% * hosted inside the theorem prover

\subsection{Working on the Whiley Abstract Syntax Tree}
% Whiley files
% * Closest to programmer code
% * most information in the most related form
% * 

\section{Structure of Loop Invariant Generators}
% ALL ONE BLOB! vs strategy + a common set of information
% Explain how each generator is distinct code
% UML of strategy pattern
% the utility package
