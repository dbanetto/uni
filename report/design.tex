\chapter{Design}\label{C:design}
% The aim here is to identify the key trade-offs in any design work you have undertaken.
% When solving a complex problem, there are normally many different
% approaches one can take — each with its own advantages and disadvantages.
% It is expected that students will have initially considered a range of different solutions,
% and will then have narrowed these down. The reasons why a particular approach was
% discounted should be documented here.
%
% Remember:
% Appropriate design notation (e.g. UML diagrams) can be very helpful in conveying different aspects of a design.
% It is vital that your design not be carried out in a vacuum.
% Your design should be motivated very clearly by your goals and specifications.
% Make sure that it is clear why you took the decisions that you did.
% Do not give the impression that you settled on a design because it “felt right”
% or that you tinkered around until you found something that worked. 

Since the Whiley compiler has multiple passes over the abstract syntax tree
(AST)
the current design adds a new pass. The compiler passes runs checks of properties
of the program such as the definite assignment of variables and sound usage of
types. This project adds an additional pass that searches for loop patterns and
inserts generated code into the AST.
This has been called the Loop Invariant Generator (LIG) pass.
This is akin to a modern macro system operating directly onto the AST.

The overall design of the Loop Invariant Generator is to collect the required
information before a loop to be used to infer loop invariants.
This is using forward propagation to determine if the
information collected is safe to use. Once the LIG reaches a loop
it tests the loop to see if it fits a loop pattern. This has been
designed using the strategy pattern, where each loop pattern is encoded
into a single strategy and produces a single invariant.
The invariant is known to be valid since they are built up via language
semantics.

This is in contrast with other state of the art loop invariant generators.
The general design of other loop invariant generators are to infer a set
of possible invariants and iterate through them trying to prove which
invariants are valid \cite{infer-dynamic} \cite{infer-postconditions}.
However, these loop invariant generators have the design to be used as a tool
to use with the source code where this project aims to integrate with the
compiler itself. Being apart of the compiler imposes multiple constraints.
This includes a constraint on keeping the impact on compile time to a
reasonable level.
Another constraint to consider is to not fail compilation for otherwise valid
code to ensure backwards compatibility and reduce surprise to the user.
Conversely should an error occur the resulting error message should not be
confusing, such as referring to generated code.


\section{Considerations}

\subsection{Usability}

\section{Limitations}

\section{}

