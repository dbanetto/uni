\documentclass[11pt]{article}


\usepackage[margin=1in]{geometry}  % set the margins to 1in on all sides
\usepackage{graphicx}              % to include figures
\usepackage{mathtools}
\usepackage{array}
\usepackage{epstopdf}
\usepackage{float}
\floatstyle{boxed}
\restylefloat{figure}

\author{David Barnett (ID:300313764)}
\title{NWEN243 Lab 4 -- Routing -- Report}
\date{}

\begin{document}

\maketitle


\section*{Introduction}

This experiment aims to compare the efficiency of the Distance Vector
routing algorithms against a range of other routing methods over a small
variety of topologies. This is composed of directly comparing the effectiveness
of the routing algorithm to deliver packets in a network.

\section*{Method}

The method used to test the multiple routing algorithms over a variety of
networks was to run them using a network simulator by the name of CNET by
The University of Western Australia. The algorithm used were: flooding1, flooding2,
flooding3  and Distance Vector (DV). Flooding1 is a blind relay that sends
all packets received onto all available links. Flooding2 is similar but does
not send the packet back down the link it was received on. Flooding3 goes a
different route with building a table of routes collecting data from all packets
passing it, using the table to do some simple routing, if it cannot use the table
follows flooding2's method. Distance Vector builds a table of nodes and the cost
to route to them using a given link, the best route to each node is shared with neighbours

\paragraph{}
The networks used to test varied from some modeled examples, such as a New Zealand
network and Australia and New Zealand network, and some extreme cases such as
K5, a fully connected graph of 5 nodes, and a star where two sets of fully
connected nodes are connected by a single node. All of the networks are
static in both number of nodes and the cost between the nodes and has a 100\%
reliable connection between every node in the network. The settings used in the
simulation was to emulate a network for 10 minutes sampling every 10 seconds
for statistics of the network's metrics such as percentage efficiency of packets
received divided by packets sent.

\section*{Results}

\begin{center}
\includegraphics[width=0.7\textwidth]{../res/K5-MAP-plot.png}
\includegraphics[width=0.7\textwidth]{../res/STAR-MAP-plot.png}
\includegraphics[width=0.7\textwidth]{../res/NEWZEALAND-MAP-plot.png}
\includegraphics[width=0.7\textwidth]{../res/NZ-AUS-MAP-plot.png}
\end{center}

\section*{Discussion}

% intro of sort
The results show in general the Distance Vector algorithm performs better or
in all networks tested with a varying magnitude of improvement. With each
network bringing out the strengths and weakness of the flooding series of
algorithms the Distance Vector routing stays the most efficient through all
four networks. There is also a distinct inverse logarithmic curve during the
initial portion of the simulation. This is also present in flooding3 which in
both cases would be caused by the setup time of building the routing table
with flooding1 and flooding2 both have more of an instantaneous response.

% K5 Map
\paragraph{}
The \emph{K5} network is designed to test how each routing method will handle
being full connected with every other node, though it is an unlikely extreme for
networks it is still shows how each method would handle being the center of a
star-like structure. Distance Vector Routing and flooding3 thrived in this
layout due to their use of tables so they do not send unneeded packets that
would result in a congest the network such as flooding1 and flooding2 had
achieved with relaying every packet.

% Star map
\paragraph{}
The \emph{Star} network is made to test how each routing method would behave
with separate ``communities'' of nodes connected by only one central node.
Relative to the spamming methods of flooding1 and flooding2 Distance Vector and
flooding3 is by magnitudes more efficient. Distance Vector being even more
efficient than flooding3 in particular during the beginning of the simulation
with Distance Vector's decision to not send packets unless it knows how to
routes them while flooding3 would spam and build a table.

% NZ Map
\paragraph{}

The \emph{New Zealand} network is a model of a real network and is similar to a
linked list configuration. The spamming methods increased in efficiency as they
had less chance to fire the packet in the wrong direction but with so many less
connections between nodes the few connections left become congested and drop
packets. This is shown with flooding3 and Distance Vector falling low relative
to other networks like \emph{K5}.

% NZ AUS Map
\paragraph{}

The \emph{New Zealand and Australia} map is also a model of a networks using
the network between major cities in Australia and New Zealand as the base. This
graph is similar to the \emph{Star} network but with not as large difference
between Distance Vector/flooding3 and flooding1/flooding2. This would due to
the structures on both sides of the modeled Tasman sea have another star-like
network to the west and a linked list to the east causing a compound of affects
seen in the \emph{New Zealand} network and \emph{Star} network.

% round up
\paragraph{}
Throughout all of the networks used to test Distance Vector algorithm, it out
performs all the  other methods over all the different networks used. This can
be attributed to Distance Vector's expanding horizon so it only allows connections
between known nodes with routes stopping any spamming seen in the other methods
thus reducing the overall congestion of the network.

\section*{Conclusion}

Through the experimentation of comparing the flooding methods against the
Distance Vector algorithm shows how communicating with neighbours about routes
and to build a table from that information can lead to greater efficiency.
The results show that overall that the Distance Vector out performs the other methods.


\end{document}

